\documentclass{docs}

%--- 基本パッケージ ---%
\usepackage[tmpdir]{graphviz} % フローチャートの作成
\usepackage{nicematrix}       % 複雑な表の作成
\usepackage{tabularx}         % 表の幅を調整
\usepackage{xpatch}           % コマンドの改造

%--- 文書情報 ---%
\title{内部設計書}

\makeatletter
	% GraphvizをLua経由で実行し、
	% 特殊な拡張子を持ったファイルを生成させ、
	% 日本語フォントを使用するよう改造
	% ※Lua経由で実行しないとファイルが生成されない
	% ※Latexmkでのビルド時に生成ファイルの変更が
	%  毎回検知されるため、無視するように設定するために
	%  特殊な拡張子を持たせている
	\renewcommand\@outext{pdflmi}
	\DeclareGraphicsRule{.pdflmi}{pdf}{*}{}
	\xpatchcmd\inputdigraph%
	{\immediate\write18{#3 -T\@outextspace -o \@tmpdir#2.\@outextspace \@tmpdir#2.dot}}%
	{
		\directlua{
			os.execute("#3 -Afontname='"..(os.getenv("GITHUB_ACTIONS")and"Hiragino Sans"or"BIZ UDPGothic").."' -Tpdf -o \@tmpdir#2.pdflmi \@tmpdir#2.dot")
		}
	}%
	{}{\errmessage{改造失敗}}
\makeatother

% モジュール定義の表を描画するマクロ
% \modtable{モジュールID}{名称}{概要}{処理手順}{入力}{出力}{補足}
% 処理手順はGraphvizのDOT言語で記述
\newcommand\modtable[7]{
	\sffamily
	\begingroup
		\setlength\arrayrulewidth{1pt}
		\begin{NiceTabularX}\textwidth{llX}[hvlines]
			\CodeBefore
			\rectanglecolor{lightgray}{1-1}{1-3}
			\rectanglecolor{lightgray}{2-1}{6-2}
			\rectanglecolor{lightgray}{7-1}{7-3}
			\rectanglecolor{lightgray}{9-1}{9-3}
			\rectanglecolor{lightgray}{11-1}{11-3}
			\rectanglecolor{lightgray}{13-1}{13-3}
			\Body
			\Block{1-3}{\emph{モジュール定義}} \\
			\Block{2-1}{\emph{管理情報}} & \emph{システム名} & SceneTrip \\
			& \emph{工程名} & 内部設計 \\
			\Block{3-1}{\emph{基本情報}} & \emph{モジュールID} & #1 \\
			& \emph{名称} & #2 \\
			& \emph{概要} & #3 \\
			\Block{1-3}{\emph{処理手順}} \\
			\Block{1-3}{
				\digraph[scale=.45]{#1}{
					#4
				}
			}
			\\
			\Block{1-3}{\emph{入力}} \\
			\Block[l]{1-3}{#5} \\
			\Block{1-3}{\emph{出力}} \\
			\Block[l]{1-3}{#6} \\
			\Block{1-3}{\emph{補足}} \\
			\Block[l]{1-3}{#7} \\
		\end{NiceTabularX}
	\endgroup
}

\begin{document}

% ====================================
\section{はじめに}
% 【記述内容】本設計書の目的、対象システムの概要、および外部設計書からの変更点を記述。

\section{システム概要}
本書は「聖地巡礼サポートシステムSceneTrip 外部設計書 第2.0版」に基づき、
システムのモジュール構成、データ処理ロジック、およびインターフェースの詳細を
定義する。
\subsection{共通機能}
\subsubsection{アカウント作成機能}

\subsubsection{ログイン・ログアウト機能}

\subsection{利用者(ログイン不要)}
\subsubsection{観光地検索}

\subsection{利用者(ログイン必須)}

\subsubsection{経路作成}

\subsubsection{Google マップ転送機能}

\subsubsection{評価機能}

\subsubsection{写真投稿・共有機能}

\subsubsection{チェックイン機能}

\subsubsection{クーポンの受け取り・利用機能}

\subsection{事業者}
\subsubsection{サブスクリプション登録機能}

\subsubsection{スポット作成機能}

\subsubsection{事業者情報変更機能}

\subsubsection{観光データ分析API}

\subsubsection{クーポン発行機能}

\subsubsection{請求書ダウンロード機能}

\subsection{管理者}
\subsubsection{観光地・施設情報管理}

\subsubsection{ユーザー管理}

\subsubsection{UGCの監視・管理機能}

\subsubsection{データ閲覧・分析機能}

\subsubsection{サブスクリプション承認・解除機能}

\section{適用範囲と制約}
外部設計書で指定されたパッケージ構成および環境設定(開発言語、セキュリティ、
データベースの構成など)を変更せずに作成する。

% ====================================

\section{開発環境}
% 【記述内容】動作環境と同様に開発環境について書く

\section{動作環境}
% 【記述内容】OS、Webサーバー、データベース、ネットワークなど、システムが稼働する環境を具体的に記述
本システムは以下の環境で動作させる。
\begin{table}[H]
	\centering
	\caption{ハードウェア構成}\label{tab:hardware}
	\begin{tabularx}{0.9\textwidth}{|l|p{9\zw}|X|p{10\zw}|}
		\hline
		\thead{項目} & \thead{種類} & \thead{数量} & \thead{備考} \\ \hline
		メインサーバ & OCI Compute & 1台 & Always Freeサービス(無料枠)\\ \hline
		データベースサーバ & OCI Compute & 1台 & Always Freeサービス(無料枠)\\ \hline
		管理者端末 & PCおよびスマートフォン & 管理者数 & \\ \hline
		利用者端末 & PCおよびスマートフォン & 利用者数 & \\ \hline
	\end{tabularx}
\end{table}

\begin{table}[H]
	\centering
	\caption{ソフトウェア構成}\label{tab:software}
	\begin{tabularx}{0.9\textwidth}{|l|X|l|}
		\hline
		\thead{項目} & \thead{ソフトウェア} & \thead{備考} \\ \hline
		Webサーバ & Nginx & \\ \hline
		データベース & MariaDB & \\ \hline
		バックエンド & PHP(Laravel) & \\ \hline
		フロントエンド & TypeScript & \\ \hline
		端末OS & Android、Linux、Windows、iOS、macOS & \\ \hline
		端末ブラウザ & Firefox、Google Chrome、Safari & \\ \hline
		LLM & Gemini 2.5 Flash-Lite & Google AI Studioを使用 \\ \hline
		地図 & OpenStreetMap & \\ \hline
		交通情報 & 国土交通省交通量API & \\ \hline
	\end{tabularx}
\end{table}

\section{開発言語・フレームワーク}
% 【記述内容】使用するプログラミング言語、フレームワーク、主要ライブラリを記述。
\begin{itemize}
	\item
\end{itemize}

\section{コーディング規約}
% 【記述内容】命名規則(クラス名、変数名)、インデント、コメントの書き方、セキュリティ上の規約などを定義。
\begin{itemize}
	\item
\end{itemize}

% ====================================

\section{パッケージ構成およびディレクトリ構造}
% 【記述内容】ソースコード全体を構成するパッケージ(モジュール)の論理的・物理的な配置を定義

\subsection{パッケージ構成図}
% 【記述内容】システム全体の主要モジュールの関係を図示

\subsection{物理ディレクトリ構造}
\begin{verbatim}

\end{verbatim}

% ====================================

\section{モジュール詳細設計}
% 【記述内容】外部設計書の機能要件を基に、各機能の入出力、責務、および内部処理フローを詳細に定義

\subsection{共通モジュール}
% 【記述内容】ログイン/ログアウト、アカウント作成とかを記述?
\newpage
\subsubsection{ログイン画面の表示}
\begin{table}[H]
  \centering
  \caption{ログイン画面の表示}\label{tab:mod-create-login}
  \modtable{M01}{loginController.php}%
  {SceneTripのログインを行う。入力値の検証、ログイン名の重複チェック}%
  {
    node [shape=box,style=rounded];%
    b, h, k, l, r [shape=diamond];%
    a [label="モジュール呼び出し"];%
    b [label="ボタン選択"];%
    c [label="「アカウントをお持ちでない方はこちら」選択"];%
    d [label="アカウント作成モジュールへ遷移"];%
    e [label="入力項目へ入力"];%
    f [label="ログインボタン選択"];%
    g [label="空白検知、特殊文字検知モジュールへ遷移"];%
    h [label="入力内容問題があるか"];%
    i [label="エラーメッセージ表示"];%
    j [label="セッション生成"];%
    k [label="ログイン確認"];%
    l [label="一般利用者か事業者か管理者か"];%
    m [label="一般利用者トップモジュールへ"];%
    n [label="事業者トップモジュールへ"];%
    o [label="管理者トップモジュールへ"];%
    p [label="Google認証選択"];%
    q [label="OAuthを利用して認証を行い、GoogleIDとユーザー名を取得"];%
    r [label="取得したGoogleIDが既に登録されているか"];%
    s [label="新規アカウント作成"];%
    t [label="usersテーブルにGoogleIDとユーザー名を登録"];%
    a->b;
    b->c->d;%
    b->e->f->g->h;%
    h->i [label="Yes"];%
    i->a;%
    h->j [label="No"];%
    j->k;%
    k->l [label="Yes"];%
    k->a [label="No"];%
    l->m [label="一般利用者"];%
    l->n [label="事業者"];%
    l->o [label="管理者"];%
    b->p [label="Google認証選択"];%
    p->q->r;%
    r->k [label="Yes"];%
    r->s [label="No"];%
    s->t->k;%
  }%
  {}%
  {ウェブページ}%
  {}
\end{table}

\newpage
\subsection{ログアウト画面の表示}
\begin{table}[H]
  \centering
  \caption{ログアウト画面の表示}\label{tab:mod-logout}
  \modtable{M02}{AccountCreater.php}%
  {SceneTripのログアウトを行う。}%
  {
    node [shape=box,style=rounded];
    b [shape=diamond];
    a [label="ボタン選択"];
    b [label="続行を選択"];
    c [label="ログアウト確認"];
    d [label="ログアウト処理"];
    e [label="ログアウトのポップアップ表示"];
    f [label="キャンセル選択"];
    g [label="前のモジュールに戻る"];
    a->b;
    b->c [label="ログアウト"];
    c->d->e;
    b->f [label="キャンセル"];
    f->g;
  }%
  {}%
  {ウェブページ}%
  {}
\end{table}

\newpage
\subsubsection{検索画面の表示}
\begin{table}[H]
	\centering
	\caption{検索画面の表示}\label{tab:mod-search-display}
	\modtable{M00}{RootController.php}%
	{検索画面を表示するモジュール}%
	{
		node [shape=diamond] c;
		node [shape=box,style=rounded] a;
		a [label="/にアクセスされる"];
		b [label="検索画面を生成"];
		c [label="ログインしているか"];
		d [label="ログイン時のUIを生成"];
		e [label="未ログイン時のUIを生成"];
		a->b;
		b->c;
		c->d [label="Yes"];
		c->e [label="No"];
	}%
	{利用者から受け取ったHTTPリクエスト}%
	{ウェブページ}%
	{}
\end{table}
\subsection{利用者モジュール}
% 【記述内容】各ステップのロジックを説明になるんかな?。
\subsubsection{観光地・施設検索処理}

\begin{table}[H]
	\centering
	\caption{観光地・施設検索処理}\label{tab:mod-search}
	\modtable{M11}{RootController.php}%
	{キーワード・カテゴリに基づくスポット検索}%
	{
		node [shape=diamond] c;
		node [shape=box,style=rounded] a;
		a [label="検索開始", shape=];
		b [label="キーワード・カテゴリ受信"];
		c [label="入力値検証", shape=diamond];
		c1 [label="入力エラー表示"];
		d [label="DB検索実行", shape=diamond];
		d1 [label="結果件数>0"];
		d2 [label="「該当無し」表示"];
		e [label="結果表示"];
		f [label="終了"];
		a->b->c;
		c->c1 [label="NG"];
		c->d [label="OK"];
		d->d2 [label="No"];
		d->e [label=Yes]
		d2->f;
		e->f;
	}%
	{}%
	{検索結果リスト または メッセージ}%
	{入力値が空の場合は、仕様に合わせて実装}
\end{table}

\subsubsection{経路作成}

\subsubsection{AIによる観光地・施設推薦}
\begin{table}[H]
    \centering
    \caption{AI観光地推薦}\label{tab:mod-ai}
    \modtable{M13}{AiService.php}
	{ユーザーのチャット入力に基づく観光地推薦}%
    {
        node [shape=diamond] c;
		node [shape=box,style=rounded] a;
        start [label="開始", shape=oval];
        input [label="チャット受信"];
        checkempty [label="空文字チェック", shape=diamond];
        fetchctx [label="周辺情報取得"];
        prompt [label="プロンプト構築"];
        apicall [label="Gemini API送信"];
        checkapi [label="API応答成功?", shape=diamond];
        parse [label="JSON解析"];
        search [label="推薦スポット取得"];
        checkhit [label="スポットあり?", shape=diamond];
        resok [label="推薦回答"];
        resng [label="エラー回答", shape=oval];
        end [label="終了", shape=oval];
        start -> input -> checkempty;
        checkempty -> resng [label="空"];
        checkempty -> fetchctx [label="OK"];
        fetchctx -> prompt -> apicall -> checkapi;
        checkapi -> resng [label="失敗"];
        checkapi -> parse [label="成功"];
        parse -> search -> checkhit;
        checkhit -> resng [label="0件"];
        checkhit -> resok [label="1件以上"];
        resok -> end;
    }%
    {}%
    {}%
    {}
\end{table}

\subsubsection{チェックイン}
\begin{table}[H]
    \centering
    \caption{チェックイン判定}\label{tab:mod-checkin}
    \modtable{M14}{CheckinService.php}{QRコードと位置情報によるスタンプ付与}
    {
		node [shape=diamond] c;
		node [shape=box,style=rounded] a;
		start [label="開始", shape=oval];
        input [label="QRデータ・GPS受信"];
        decodeqr [label="QR検証(SpotID)", shape=diamond];
        getspot [label="Spot座標取得"];
        calcdist [label="距離計算"];
        checkdist [label="距離 <= 50m?", shape=diamond];
        savestamp [label="スタンプ保存"];
        checkcoup [label="クーポン条件?", shape=diamond];
        savecoup [label="クーポン付与"];
        msgsuccess [label="成功画面"];
        msgerror [label="エラー画面", shape=oval];
        end [label="終了", shape=oval];
        start -> input -> decodeqr;
        decodeqr -> msgerror [label="不正"];
        decodeqr -> getspot [label="正"];
        getspot -> calcdist -> checkdist;
        checkdist -> msgerror [label="範囲外"];
        checkdist -> savestamp [label="範囲内"];
        savestamp -> checkcoup;
        checkcoup -> msgsuccess [label="未達"];
        checkcoup -> savecoup [label="達成"];
        savecoup -> msgsuccess -> end;
    }%
    {}%
    {スタンプ取得結果, クーポン(条件達成時)}
    {}
\end{table}

\subsubsection{評価・写真投稿}
\begin{table}[H]
    \centering
    \caption{評価・写真投稿}\label{tab:mod-review-photo}
    \modtable{M15}{ReviewService.php}{評価と写真をシステムに登録する}%
    {
		node [shape=diamond] c;
		node [shape=box,style=rounded] a;
        start [label="開始", shape=oval];
        input [label="投稿データ受信"];
        validate [label="入力チェック", shape=diamond];
        hasimg [label="画像あり?", shape=diamond];
        upload [label="画像アップロード"];
        checkup [label="アップロード成功?", shape=diamond];
        convert [label="座標変換"];
        savedb [label="DB保存"];
        success [label="完了表示"];
        error [label="エラー表示", shape=oval];
        end [label="終了", shape=oval];
        start -> input -> validate;
        validate -> error [label="NG"];
        validate -> hasimg [label="OK"];
        hasimg -> savedb [label="No"];
        hasimg -> upload [label="Yes"];
        upload -> checkup;
        checkup -> error [label="失敗"];
        checkup -> convert [label="成功"];
        convert -> savedb;
        savedb -> success -> end;
    }%
    {}%
    {投稿完了メッセージ}%
    {}
\end{table}
\subsubsection{クーポン受け取り}
\subsubsection{クーポン受け取り処理}
\begin{table}[H]
    \centering
    \caption{クーポン受け取り処理}\label{tab:mod-coupon-receive}
    \modtable{M16}{CouponService.php}{スタンプ獲得状況に基づくクーポン付与}%
    {
		node [shape=diamond] c;
		node [shape=box,style=rounded] a;
        start [label="開始", shape=oval];
        input [label="獲得スタンプID受信"];
        fetchcond [label="関連クーポン検索(condspotid)"];
        checkhit [label="対象クーポンあり?", shape=diamond];
        checkduplicate [label="既に獲得済み?", shape=diamond];
        checkexpire [label="有効期限内?", shape=diamond];
        savedb [label="usercoupons保存(利用可能状態で作成)"];
        ressuccess [label="獲得成功通知"];
        resnone [label="獲得なし通知", shape=oval];
        end [label="終了", shape=oval];
        start -> input -> fetchcond -> checkhit;
        checkhit -> resnone [label="なし"];
        checkhit -> checkduplicate [label="あり"];
        checkduplicate -> resnone [label="Yes"];
        checkduplicate -> checkexpire [label="No"];
        checkexpire -> resnone [label="期限切"];
        checkexpire -> savedb [label="有効"];
        savedb -> ressuccess -> end;
    }%
    {}%
    {}%
    {スタンプ獲得トリガーで呼び出される想定。ユーザーが同じクーポンを重複して受け取らないようにする}
\end{table}


\subsection{事業者モジュール}
% 【記述内容】各ステップのロジックを説明になるんかな?。

\subsection{管理者モジュール}
% 【記述内容】各ステップのロジックを説明になるんかな?。
% ====================================
\newpage
\subsubsection{観光地・施設情報追加画面の表示}
\begin{table}[H]
	\centering
	\caption{観光地・施設情報追加画面の表示}\label{tab:mod-search}
	\modtable{M00}{RootController.php}%
	{観光地・施設情報追加画面を表示するモジュール}%
	{
		node [shape=diamond] c;
		node [shape=box,style=rounded] a;
		a [label="追加タブを選択"];
		b [label="情報追加画面を生成"];
		c [label="必要情報を入力"]
		d [label="入力値を検証"];
		e [label="SQLクエリを生成"];
		f [label="情報追加できない旨のUIを生成"];
		g [label="情報追加処理を実行"];
		h [label="情報追加完了画面を生成・表示"];
		a->b;
		b->c;
		c->d [label="Yes"];
		c->e [label="No"];
		d->f;
		f->g;
		g->h;
	}%
	{利用者から受け取ったHTTPリクエスト}%
	{ウェブページ}%
	{}
\end{table}

\newpage
\subsubsection{観光地・施設情報編集画面の表示}
\begin{table}[H]
	\centering
	\caption{観光地・施設情報編集画面の表示}\label{tab:mod-search}
	\modtable{M00}{RootController.php}%
	{観光地・施設情報編集画面を表示するモジュール}%
	{
		node [shape=diamond] c;
		node [shape=box,style=rounded] a;
		a [label="編集タブを選択"];
		b [label="情報編集画面を生成"];
		c [label="必要情報を入力"]
		d [label="入力値を検証"];
		e [label="SQLクエリを生成"];
		f [label="情報編集できない旨のUIを生成"];
		g [label="情報編集処理を実行"];
		h [label="情報編集完了画面を生成・表示"];
		a->b;
		b->c;
		c->d;
		d->e [label="Yes"];
		d->f [label="No"];
		f->g;
		g->h;
	}%
	{利用者から受け取ったHTTPリクエスト}%
	{ウェブページ}%
	{}
\end{table}

\newpage
\subsubsection{観光地・施設情報削除画面の表示}
\begin{table}[H]
	\centering
	\caption{観光地・施設情報削除画面の表示}\label{tab:mod-search}
	\modtable{M00}{RootController.php}%
	{観光地・施設情報削除画面を表示するモジュール}%
	{
		node [shape=diamond] c;
		node [shape=box,style=rounded] a;
		a [label="削除タブを選択"];
		b [label="情報削除画面を生成"];
		c [label="操作の確認画面の表示"];
		d [label="SQLクエリを生成"];
		e [label="元の画面を表示"];
		f [label="情報削除処理を実行"];
		g [label="情報削除完了画面を生成・表示"];
		a->b;
		b->c;
		c->d [label="Yes"];
		c->e [label="No"];
		d->f;
		f->g;
	}%
	{利用者から受け取ったHTTPリクエスト}%
	{ウェブページ}%
	{}
\end{table}


\section{データベースアクセス設計}
% 【記述内容】データベースへのアクセス方法、外部設計書で指定された特殊なデータ形式やロジック(ID生成など)の詳細を定義

\subsection{ }
\begin{itemize}
	\item
\end{itemize}

\subsection{}
\begin{itemize}
	\item
\end{itemize}

% ====================================
\section{外部インターフェース設計}
% 【記述内容】外部サービスとの連携におけるリクエスト・レスポンスの形式、認証方法を定義

\subsection{「Gemini API」連携仕様}
\begin{itemize}
    \item
\end{itemize}

\subsection{「OpenStreetMap」連携仕様}
\begin{itemize}
    \item
\end{itemize}

\subsection{「Google マップ」連携仕様}

\subsection{「国土交通省交通量API」連携仕様}
\begin{itemize}
    \item
\end{itemize}


\end{document}
