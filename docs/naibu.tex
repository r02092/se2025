\documentclass{docs}

%--- 基本パッケージ ---%
\usepackage{tikz} % 図の作成

%--- TikZライブラリ ---%
\usetikzlibrary{arrows.meta} % 矢印
\usetikzlibrary{shapes.geometric} % 幾何学図形

%--- 文書情報 ---%
\title{内部設計書}

% アクティビティ図を描画するマクロ
% \adiagram{パーティション名}{縦の長さ}{流れ}{追加する描画命令}
% パーティション名: 上部に表示されるパーティション名を","で区切って記述
% 縦の長さ    : アクションごとに縦に進む長さ
% 本筋のフロー  : パーティション番号
%           0:アクション or 1:分岐
%           アクション名 or 分岐の名前
%           上記の3つを"/"で区切り、それを","で区切って記述
%           作成される各アクションのnodeは
%           "action<通し番号>"という名前で
%           (パーティション番号,-通し番号)の座標に配置される
%           本筋のアクションしか描画できないため、それ以外は
%           以下の引数に描画命令を渡して手動で描く必要がある
% 追加する描画命令: 追加で描画するTikZの形式の命令
%           条件分岐で別れた本筋以外のフローの描画などに使用
\newcommand\adiagram[4]{
	\foreach\t[count=\i from 0]in{#1}{
		\xdef\nps{\i}
	}
	\begin{tikzpicture}[
		action/.style={draw,thick,rounded corners,font=\sffamily,align=center},
		decision/.style={draw,thick,diamond,inner sep=8pt},
		arrow/.style={-Stealth},
		label/.style={fill=white,inner sep=0pt,font=\sffamily,align=center},
		xscale=(\textwidth-1.6pt)/(\nps cm+1cm)
	]
		\foreach\t[count=\i from 0]in{#1}
			\node[font=\bfseries\sffamily]at(\i,1.5){\t};
		\begin{scope}[yscale=#2]
			\node[fill,inner sep=8pt,shape=circle](action0)at(0,0){};
			\xdef\ot{}
			\xdef\ox{0}
			\xdef\oi{0}
			\foreach\x/\e/\t[count=\i from 1]in{#3}{
				\ifnum\e=0
					\node[action](action\i)at(\x,-\i){\t};
				\else
					\node[decision](action\i)at(\x,.25-\i){};
				\fi
				\draw[arrow](action\oi)
				\if"\ot"
					\ifnum\x=\ox--\else-|\fi(action\i);
				\else
					\ifnum\x=\ox--node[label]{\ot}\else-|node[label]{\ot}\fi(action\i);
					\xdef\ot{}
				\fi
				\ifnum\e=1\xdef\ot{\t}\fi
				\xdef\ox{\x}
				\xdef\oi{\i}
			}
			\node[draw,double distance=2pt,thick,fill,inner sep=8pt,shape=circle](end)at(\ox,-1-\oi){};
			\draw[arrow](action\oi)--(end);
		\end{scope}
		\foreach\i in{-.5,...,\nps.5}
			\draw[ultra thick](\i,1)--(\i,-2.5-\oi*#2);
		\begin{scope}[yscale=#2]
			#4
		\end{scope}
	\end{tikzpicture}
}

\begin{document}

% ====================================
\section{はじめに}
% 【記述内容】本設計書の目的、対象システムの概要、および外部設計書からの変更点を記述。

\subsection{システム概要}
本書は、外部設計書(SceneTrip 第2.0版)に基づき、システムのモジュール構成、データ処理ロジック、およびインターフェースの詳細を定義する。

\subsection{適用範囲と制約}
本設計は、外部設計書で指定されたパッケージ構成および環境設定(PHP、Argon2id、特定のデータベースアクセスロジックなど)を変更せずに作成する。

% ====================================

\section{環境・コーディング規約}
% 【記述内容】動作環境、開発言語、ライブラリ、およびプロジェクト全体のコーディングルールを定義

\subsection{動作環境}
% 【記述内容】OS、Webサーバー、データベース、ネットワークなど、システムが稼働する環境を具体的に記述
\begin{itemize}

\end{itemize}

\subsection{開発言語・フレームワーク}
% 【記述内容】使用するプログラミング言語、フレームワーク、主要ライブラリを記述。
\begin{itemize}

\end{itemize}

\subsection{コーディング規約}
% 【記述内容】命名規則(クラス名、変数名)、インデント、コメントの書き方、セキュリティ上の規約などを定義。
\begin{itemize}

\end{itemize}

% ====================================

\section{パッケージ構成およびディレクトリ構造}
% 【記述内容】ソースコード全体を構成するパッケージ(モジュール)の論理的・物理的な配置を定義

\subsection{パッケージ構成図}
% 【記述内容】システム全体の主要モジュールの関係を図示

\subsection{物理ディレクトリ構造}
\begin{verbatim}

\end{verbatim}

% ====================================

\section{モジュール詳細設計}
% 【記述内容】外部設計書の機能要件を基に、各機能の入出力、責務、および内部処理フローを詳細に定義

\subsection{認証モジュール}
% 【記述内容】ログイン/ログアウト、アカウント作成とかを記述?

\subsection{}
% 【記述内容】各ステップのロジックを説明になるんかな?。

% ====================================

\section{データベースアクセス設計}
% 【記述内容】データベースへのアクセス方法、外部設計書で指定された特殊なデータ形式やロジック(ID生成など)の詳細を定義

\subsection{ }
\begin{itemize}

\end{itemize}

\subsection{}
\begin{itemize}

\end{itemize}

% ====================================
\section{外部インターフェース設計}
% 【記述内容】外部サービスとの連携におけるリクエスト・レスポンスの形式、認証方法を定義

\subsection{}
\begin{itemize}
    \item
\end{itemize}

\end{document}
