\documentclass{docs}

%--- 基本パッケージ ---%
\usepackage[tmpdir]{graphviz} % フローチャートの作成
\usepackage[verb]{bxghost}    % 和欧文間のアキを調整
\usepackage{dirtree}          % ディレクトリ構造の図を作成
\usepackage{mboxfill}         % 特定の文字で埋める
\usepackage{nicematrix}       % 複雑な表の作成
\usepackage{tabularx}         % 表の幅を調整
\usepackage{xpatch}           % コマンドの改造

%--- 文書情報 ---%
\title{内部設計書}

% \dotfillで埋める文字をGitHub Actions(ヒラギノ角ゴ)の場合に変更
\if"\directlua{tex.print(os.getenv("GITHUB_ACTIONS"))}"{}\else
	\renewcommand\dotfill{\mboxfill[3.333333pt]{\CID{9778}}}
\fi
\setlength\DTbaselineskip{9pt}

\makeatletter
	% GraphvizをLua経由で実行し、
	% 特殊な拡張子を持ったファイルを生成させ、
	% 日本語フォントを使用するよう改造
	% ※Lua経由で実行しないとファイルが生成されない
	% ※Latexmkでのビルド時に生成ファイルの変更が
	%  毎回検知されるため、無視するように設定するために
	%  特殊な拡張子を持たせている
	\renewcommand\@outext{pdflmi}
	\DeclareGraphicsRule{.pdflmi}{pdf}{*}{}
	\xpatchcmd\inputdigraph%
	{\immediate\write18{#3 -T\@outextspace -o \@tmpdir#2.\@outextspace \@tmpdir#2.dot}}%
	{
		\directlua{
			os.execute("#3 -Afontname='"..(os.getenv("GITHUB_ACTIONS")and"Hiragino Sans"or"BIZ UDPGothic").."' -Tpdf -o \@tmpdir#2.pdflmi \@tmpdir#2.dot")
		}
	}%
	{}{\errmessage{改造失敗}}
\makeatother

% モジュール定義の表を描画するマクロ
% \modtable{モジュールID}{名称}{概要}{処理手順}{入力}{出力}{補足}
% 処理手順はGraphvizのDOT言語で記述
\newcommand\modtable[7]{
	\sffamily
	\begingroup
		\setlength\arrayrulewidth{1pt}
		\begin{NiceTabularX}\textwidth{llX}[hvlines]
			\CodeBefore
			\rectanglecolor{lightgray}{1-1}{1-3}
			\rectanglecolor{lightgray}{2-1}{6-2}
			\rectanglecolor{lightgray}{7-1}{7-3}
			\rectanglecolor{lightgray}{9-1}{9-3}
			\rectanglecolor{lightgray}{11-1}{11-3}
			\rectanglecolor{lightgray}{13-1}{13-3}
			\Body
			\Block{1-3}{\emph{モジュール定義}} \\
			\Block{2-1}{\emph{管理情報}} & \emph{システム名} & SceneTrip \\
			& \emph{工程名} & 内部設計 \\
			\Block{3-1}{\emph{基本情報}} & \emph{モジュールID} & #1 \\
			& \emph{名称} & #2 \\
			& \emph{概要} & #3 \\
			\Block{1-3}{\emph{処理手順}} \\
			\Block{1-3}{
				\digraph[scale=.3]{#1}{
					#4
				}
			}
			\\
			\Block{1-3}{\emph{入力}} \\
			\Block[l]{1-3}{#5} \\
			\Block{1-3}{\emph{出力}} \\
			\Block[l]{1-3}{#6} \\
			\Block{1-3}{\emph{補足}} \\
			\Block[l]{1-3}{#7} \\
		\end{NiceTabularX}
	\endgroup
}

\begin{document}

% ====================================
\section{はじめに}
% 【記述内容】本設計書の目的、対象システムの概要、および外部設計書からの変更点を記述。

\section{システム概要}
本書は「聖地巡礼サポートシステムSceneTrip 外部設計書 第2.0版」に基づき、
システムのモジュール構成、データ処理ロジック、およびインターフェースの詳細を
定義する。
\subsection{共通機能}
\subsubsection{アカウント作成機能}
システムを利用するためにはアカウントを作成する必要がある。
アカウント作成時にはアカウント作成画面から
「ユーザー名」、「ログイン名」、「パスワード」、「確認用パスワード」
の4つの文字列を受け取り、文字列の検証とアカウントの作成を行う。
登録したアカウント情報(プロフィール)はログイン名を除き、プロフィール画面で編集可能とする。

\subsubsection{ログイン・ログアウト機能}
本システムにはアカウントの作成に伴い、ログイン及びログアウトの機能がある。
ログインを行うことで、ユーザーは本システムの全ての機能を利用できる。
一方ログインを行わない場合、利用できる機能は観光地検索機能に制限される。
ログインには本システムで作成するアカウントを使用する方法とGoogle アカウントを使用する方法がある。
本システムで作成したアカウントを使用する方法では、
システムはログイン用画面でログイン名とパスワードを受け取り、検証を行う。
Google アカウントを使用する方法では、OAuthを用いて認証を行い、
GoogleのIDとユーザー名から検証を行う。
また、RFC 6238に準拠したTOTPによる二要素認証を受け付けており、
プロフィール画面から設定を行えるものとする。

\subsection{利用者(ログイン不要)}
\subsubsection{観光地・施設検索}
検索処理は、キーワードとカテゴリを入力として受け取り、入力値の検証を行う。
検索結果には、観光地や施設の詳細情報を表示する。

\subsection{利用者(ログイン必須)}

\subsubsection{経路作成}
ユーザーが入力した出発地・到着地となる観光地・施設から経路を生成する。
また、Gemini APIを利用して、Geminiから観光地・施設の提案を行う。
提案した観光地・施設は出発地から到着地の中継地点として経路に含める。
本システムでは経路をノードとエッジを用いたグラフとして表す。

\subsubsection{Google マップ転送機能}
ユーザーとGeminiからの入力によって作成した経路から
Google マップ上で表示できるURIを生成する。
ユーザーはURIを基にGoogle マップへと移動する。

\subsubsection{評価機能}
観光地・施設に5段階の評価とコメントを投稿できる。
システムはそれらをスポットに関連付ける。

\subsubsection{写真投稿・共有機能}
場所に関連付けて地図上に表示される写真及びコメントを投稿できる。
システムは写真とコメントを該当する経度、緯度と関連付ける。

\subsubsection{チェックイン機能}
訪れた観光地・施設のスタンプを現地まで赴いた実績として入手できる機能である。
システムはユーザーのカメラ映像から二次元コードを読み取り、
ユーザーの位置情報を基に観光地・施設から半径50\,mの範囲にユーザーが位置しているか検証を行う。

\subsubsection{クーポンの受け取り・利用機能}
本システム上で提供されるクーポンの受け取り、並びにそのクーポンの利用ができる。
クーポンは観光地・施設等での割引や特典に利用できる。
クーポンは、利用可能・利用中・利用済み・期限切れの4状態で表示する。
クーポンの「利用中」とは、クーポンを提示して店舗等で利用手続きを行った状態を指し、
翌日5時には「利用済み」となる。
ユーザーは画面上でクーポンが「利用可能」なのか、「利用中」なのかを確認できるものとする。

\subsection{事業者}
\subsubsection{サブスクリプション登録機能}
本システムのスタンダードプランまたはプレミアムプランへの登録の申請を行う機能である。
登録されたプランは、各プランの契約数としてアカウントに紐づく。
また、申請時に郵便番号と住所を受け取り、アカウントに関連付ける。

\subsubsection{スポット(観光地・施設)情報管理機能}
スポットの追加、情報の編集、削除を行う機能である。
システムはスポットを追加する際にスポットの名前、種別、所在地を取得し、スポットに関連付ける。
追加したスポットの情報は後で編集、削除できる。

\subsubsection{事業者情報変更機能}
登録した店舗情報(店舗名、所在地、店舗写真)を変更する機能である。

\subsubsection{観光データ分析API}
システムは逐一、ユーザーの観光地・施設検索の検索内容やユーザーのIDなどを収集し、
観光データ分析に生かせるAPIの機能を提供する。

\subsubsection{クーポン発行機能}
システム上でクーポンを発行できる。
これはサブスクリプションに登録したアカウントのみで行うことができるものとする。
作成時に使用できるスポット、クーポン名、行く必要のあるスポット、期限を入力する必要がある。

\subsubsection{請求書ダウンロード機能}
受領した請求書をPDF形式でダウンロードできるものとする。

\subsection{管理者}

\subsubsection{ユーザー管理}
アカウントの作成、ユーザー情報の編集、削除を行う機能である。

\subsubsection{UGCの監視・管理機能}
UGCの監視、投稿の削除ができる機能である。
システムは管理者が投稿された口コミやフォトスポット投稿について、
利用規約違反があると判断した場合、
投稿の削除を行えるものとする。

\subsubsection{データ閲覧・分析機能}
観光分析APIのデータやダッシュボードを監視する機能である。

\subsubsection{サブスクリプション承認・解除機能}
事業者からのサブスクリプション登録の申請の承認または却下、
サブスクリプションの解除または停止を行う機能である。
システムは管理者が事業者の未払いや、規約違反があると判断した場合、
強制的にサブスクリプションを解除または停止できるものとする。

\section{適用範囲と制約}
外部設計書で指定されたパッケージ構成および環境設定(開発言語、セキュリティ、
データベースの構成など)を変更せずに作成する。

% ====================================

\section{技術スタック}
% 【記述内容】使用するプログラミング言語、フレームワーク、主要ライブラリを記述。
本システムは\cref{tab:tech}に示す技術を使用する。
\begin{table}[H]
	\centering
	\caption{技術スタック}\label{tab:tech}
	\begin{tabularx}{0.9\textwidth}{|p{13\zw}|X|l|}
		\hline
		\thead{項目} & \thead{ソフトウェア} & \thead{備考} \\ \hline
		Webサーバ & Nginx & \\ \hline
		データベース & MariaDB & \\ \hline
		バックエンド開発言語 & PHP & \\ \hline
		バックエンドパッケージマネージャ & Composer & \\ \hline
		バックエンドフレームワーク & Laravel & \\ \hline
		バックエンド単体テスト & PHPUnit & \\ \hline
		ソーシャルログイン & Laravel Socialite & \\ \hline
		フロントエンド開発言語 & TypeScript & \\ \hline
		フロントエンドランタイム & Node.js & \\ \hline
		Node.jsバージョン管理 & Node Version Manager & \\ \hline
		フロントエンドパッケージマネージャ & pnpm & \\ \hline
		フロントエンドバンドラ & Vite & \\ \hline
		フロントエンドテスト & Vitest & \\ \hline
		E2Eテスト & Playwright & \\ \hline
		PDFファイルの作成 & tFPDF & \\ \hline
		PDFファイルの加工 & FPDI & \\ \hline
		二次元コードの作成 & Bacon/BaconQrCode & \\ \hline
		二次元コードの読み取り & nimiq/qr-scanner & \\ \hline
		地図表示 & MapLibre GL JS & \\ \hline
		LLM & Gemini 2.5 Flash-Lite & Google AI Studioを使用 \\ \hline
		地図 & OpenStreetMap & \\ \hline
		交通情報 & 国土交通省交通量API & \\ \hline
	\end{tabularx}
\end{table}

\section{開発環境}
% 【記述内容】動作環境と同様に開発環境について書く
本システムは、GitHubでソースコードを管理し、GitHubに
搭載されたCI/CDツールであるGitHub Actionsを開発に用いる。
本文書を含めたシステムに関わる文書を管理するリポジトリと共通のリポジトリを
用いる。
また、コードフォーマッタであるPrettierおよび
リンタであるESLintとLarastanを用いる。

GNU/LinuxシステムまたはWindowsで動作するPCを開発用PCとして用いる。
開発用PCには、Node Version Managerをインストールし、
それを用いてNode.jsをインストールし、それに含まれるnpmを用いてpnpmを
インストールするものとする。
また、開発用PCにコンテナを用いて\cref{tab:dev-con}に示す環境を作成し、開発をおこなう。
コンテナ管理ツールとしてはPodmanまたはDockerを用いる。
\begin{table}[H]
	\centering
	\caption{開発環境}\label{tab:dev-con}
	\begin{tabularx}{0.9\textwidth}{|l|X|p{10\zw}|}
		\hline
		\thead{項目} & \thead{ソフトウェア} & \thead{備考} \\ \hline
		OS & Alpine 3.23 & \\ \hline
		PHP実行環境 & PHP 8.3 & \\ \hline
		データベース & MariaDB 10.11.14 & \\ \hline
		自動レビュ & Reviewdog v0.18.1 & \\ \hline
	\end{tabularx}
\end{table}
GitHub Actionsでは、以下の作業を自動でおこなう。
\begin{itemize}
	\item \emph{\LaTeX 文書のコンパイル}
	\begin{itemize}
		\item 特定の形式のGitタグがプッシュされた際に動作
		\item 文書に用いるフォントの都合上、macOSのイメージ上で動作
		\item Mac\TeX の環境の構築に時間がかかるため、6日毎に
		環境のキャッシュを作成
	\end{itemize}
	\item \emph{Pull Requestの自動確認}
	\begin{itemize}
		\item フォーマッタによる整形を自動でコミット
		\item リンタによる指摘箇所をReviewdogでコメント
		\item リンタによる指摘内容が英語の場合、日本語に自動翻訳
		\item PHPUnitによるバックエンドの自動での単体テスト
	\end{itemize}
	\item \emph{Google App Scriptのデプロイ}
	\begin{itemize}
		\item 翻訳APIの呼び出しに使用
	\end{itemize}
	\item \emph{システム全体のデプロイ}
\end{itemize}
\cref{tab:dev-latex}および\cref{tab:dev-gha}に、GitHub Actionsワークフローで
用いる環境を示す。
この表に書いていない細かなソフトウェアは、
パッケージ管理システム(UbuntuであればAPT、macOSで
あればHomebrew)で導入されるバージョンを使用する。
\begin{table}[H]
	\centering
	\caption{\LaTeX 文書をコンパイルするGitHub Actionsワークフローで用いる環境}\label{tab:dev-latex}
	\begin{tabularx}{0.9\textwidth}{|l|X|p{10\zw}|}
		\hline
		\thead{項目} & \thead{ソフトウェア} & \thead{備考} \\ \hline
		OS & macOS Sequoia & \\ \hline
		\TeX ディストリビューション & Mac\TeX{} 2025 & \\ \hline
	\end{tabularx}
\end{table}
\begin{table}[H]
	\centering
	\caption{その他のGitHub Actionsワークフローで用いる環境}\label{tab:dev-gha}
	\begin{tabularx}{0.9\textwidth}{|l|X|p{10\zw}|}
		\hline
		\thead{項目} & \thead{ソフトウェア} & \thead{備考} \\ \hline
		OS & Ubuntu 24.04.3 & \\ \hline
		PHP実行環境 & PHP 8.3 & \\ \hline
		データベース & MariaDB 10.11.13 & \\ \hline
		自動レビュ & Reviewdog v0.18.1 & \\ \hline
	\end{tabularx}
\end{table}

また、リポジトリには依存関係のアップデートをおこない、Pull Requestを
自動で作成するbotであるRenovateを導入する。
これにより、Composerおよびpnpmで管理する依存関係やGitHub Actionsの
依存関係、Node.jsのアップデートを自動でおこない、最新の状態に保つ。

\section{動作環境}
% 【記述内容】OS、Webサーバー、データベース、ネットワークなど、システムが稼働する環境を具体的に記述
本システムは\cref{tab:honban}に示す環境で動作させる。
\begin{table}[H]
	\centering
	\caption{動作環境}\label{tab:honban}
	\begin{NiceTabularX}{0.9\textwidth}{p{6\zw}Xp{4\zw}p{10\zw}}[hvlines]
		\thead{項目} & \thead{種類} & \thead{数量} & \thead{備考} \\
		メインサーバ & OCI Compute & 1台 & Always Freeサービス(無料枠)\\
		データベースサーバ & OCI Compute & 1台 & Always Freeサービス(無料枠)\\
		管理者端末 & PCおよびスマートフォン & 管理者数 & \\
		利用者端末 & PCおよびスマートフォン & 利用者数 & \\
		端末OS & Android、Linux、Windows、iOS、macOS & \diagbox{}{} & \\
		端末ブラウザ & Firefox、Google Chrome、Safari & \diagbox{}{} & \\
	\end{NiceTabularX}
\end{table}

\section{コーディング規約}
% 【記述内容】命名規則(クラス名、変数名)、インデント、コメントの書き方、セキュリティ上の規約などを定義。
\begin{itemize}
	\item \emph{共通}
	\begin{itemize}
		\item 文字コードは、BOMを付与しないUTF-8を用いる。
		\item 改行コードは、LFを用いる。
		\item 各ファイルの末尾は、改行とする。
		\item 各行の末尾に1つ以上のタブおよびスペースの
		連続が存在しないようにする。
		\item 原則として、各行の文字数は最大で80文字程度と
		する。この規則において、全角文字は2文字、
		タブは4文字として数える。
		\item 何らかの要素の列挙について、
		最後の要素の後の末尾のカンマは、
		列挙が改行を伴いかつ挿入が可能であれば挿入し、
		そうでない場合は挿入しないものとする。
		\item 可能な場合、Prettierを用いてコードの
		自動整形をおこなう。
		設定は、ここに記述する規約を可能な限り
		適用するものとする。
		\item 環境変数の命名には、アッパースネークケースを用いる。
		\item ここに記述する規約は、依存関係などの\verb|.gitignore|ファイルで無視するファイルには
		適用しないものとする。
	\end{itemize}
	\item \emph{PHPコード}
	\begin{itemize}
		\item インデントには、スペース4つを用いる。
		\item 文字列には、優先的にシングルクォートを用いる。
		ダブルクォートよりシングルクォートを多く含む文字列や
		変数展開を用いる場合など、ダブルクォートを
		用いることが好ましい場合は、ダブルクォートを用いる。
		\item コードブロックを構成する括弧は、
		インデントとそれのみを含む単一の行に配置する。
		\item PHPStanの指摘が起きないようにする。
		設定は、デフォルトのルールを用い、
		必要に応じて開発中に変更するものとする。
		\item クラスおよびトレイト、名前空間の命名には、アッパーキャメルケースを用いる。
		\item 関数および変数の命名には、ローワーキャメルケースを用いる。
		\item 定数の命名には、アッパースネークケースを用いる。
		\item データベースのテーブルおよびカラムの命名には、
		ローワースネークケースを用いる。
		\item \verb|App\Http\Controllers\Controller|クラスを継承するクラスの名前の
		末尾は、\verb|Controller|とする。
		\item \verb|App\Models\|名前空間に存在するクラスの名前は、対応するデータベースのテーブル名に
		基づくものとする。
		\item \verb|App\Http\Controllers\Controller|クラスを継承するクラスから
		呼び出される共通処理は、\verb|App\Traits\|名前空間に
		属するトレイトとする。
		\item \verb|App\Traits\|名前空間に属するトレイトの
		名前の末尾は、\verb|Trait|とする。
		\item クラスと対応するファイルの場所は、PSR-4に準拠したオートローダで読み込み可能な場所とする。
		ただし、名前空間とディレクトリには\cref{tab:coding-psr4}のような対応付けがあるものとする。
		\begin{table}[H]
			\centering
			\caption{名前空間とディレクトリの対応付け}\label{tab:coding-psr4}
			\begin{tabularx}{0.9\textwidth}{|X|X|}
				\hline
				\thead{名前空間} & \thead{ディレクトリ} \\ \hline
				\texttt{App\textbackslash} & \texttt{app/} \\ \hline
				\texttt{Database\textbackslash Factories\textbackslash} & \texttt{database/factories/} \\ \hline
				\texttt{Database\textbackslash Seeders\textbackslash} & \texttt{database/seeders/} \\ \hline
				\texttt{Tests\textbackslash} & \texttt{tests/} \\ \hline
			\end{tabularx}
		\end{table}
		\item クラスと対応しないファイルは、\cref{sec:dir}で定めたディレクトリに配置するものとする。
		\item クラスと対応しないファイルの名前は、\verb|artisan|ファイルを除き、
		ローワースネークケースを用いた名前の末尾に“\verb|.php|”を付与するものとする。
	\end{itemize}
	\item \emph{Bladeテンプレート}
	\begin{itemize}
		\item インデントには、タブを用いる。
		\item HTMLタグが改行を伴って記述される場合、
		タグの閉じ括弧は、インデントとそれのみを含む
		単一の行に配置する。
		\item 要素の名前は、半角英数字列とする。
		\item カスタム属性の名前は、ケバブケースを用いた名前の先頭に“\verb|data-|”を付与するものとする。
		\item IDの命名には、ローワースネークケースを用いる。
		\item クラスの命名には、ケバブケースを用いる。
		\item ファイルは、\verb|resources/views/|ディレクトリに配置するものとする。
		\item ファイル名は、ローワースネークケースを用いた名前の末尾に“\verb|.blade.php|”を
		付与するものとする。
	\end{itemize}
	\item \emph{CSSスタイルシート}
	\begin{itemize}
		\item インデントには、タブを用いる。
		\item 0には単位を付与しないものとする。
		\item 要素やカスタム属性、ID、クラスの
		命名規則は、Bladeテンプレートのものに従う。
		\item ルールセット内で、関連する内容の
		プロパティ(同じプレフィックスをもつものなど)が
		存在する場合、それらを連続させるような順序で
		記述する。
		\item ファイルは、\verb|resources/css/|ディレクトリ配下に配置するものとする。
		\item ファイル名は、ローワースネークケースを用いた名前の末尾に“\verb|.css|”を付与するものとする。
	\end{itemize}
	\item \emph{TypeScriptコード}
	\begin{itemize}
		\item インデントには、タブを用いる。
		\item 文は、セミコロンで区切る。
		\item 文字列には、優先的にダブルクォートを用いる。
		シングルクォートよりダブルクォートを多く含む
		文字列など、シングルクォートを用いることが
		好ましい場合は、シングルクォートを用いる。
		\item オブジェクトのプロパティ名は、必要な場合のみ
		クォートで囲むものとする。
		\item オブジェクト内の要素と括弧の間には、
		スペースを挿入しないものとする。
		\item アロー関数の引数が1つの場合、引数を囲う括弧は
		省略する。
		\item ESLintの指摘が起きないようにする。
		設定は、デフォルトのルールを用い、
		必要に応じて開発中に変更するものとする。
		\item クラスおよび名前空間の命名には、アッパーキャメルケースを用いる。
		\item 関数および変数、定数の命名には、ローワーキャメルケースを用いる。
		\item ファイルは、\verb|resources/ts/|ディレクトリ配下に配置するものとする。
		\item ファイル名は、ローワースネークケースを用いた名前の末尾に“\verb|.ts|”を付与するものとする。
	\end{itemize}
\end{itemize}

% ====================================

\section{ディレクトリ構造}\label{sec:dir}
ディレクトリ構造を示す図を\cref{fig:dir}に示す。
\begin{figure}[H]
	{
		\footnotesize
		\dirtree{%
			.1 se2025\textrm{\DTcomment{リポジトリのルート}}.
			.2 .github\textrm{\DTcomment{リモートリポジトリ関連の設定}}.
			.3 workflows\textrm{\DTcomment{GitHub Actionsのワークフロー}}.
			.2 .vscode\textrm{\DTcomment{Visual Studio Codeの設定}}.
			.2 app\textrm{\DTcomment{システムの主な処理を行うプログラム}}.
			.3 Http\textrm{\DTcomment{HTTPリクエストがあった場合に実行されるプログラム}}.
			.4 Controllers\textrm{\DTcomment{ルーティングから呼び出されるコントローラとなるプログラム}}.
			.3 Models\textrm{\DTcomment{データベースのテーブルが表現する実体に対応したクラスを定義するプログラム}}.
			.3 Providers\textrm{\DTcomment{サービスプロバイダとなるプログラム}}.
			.3 Traits\textrm{\DTcomment{コントローラから呼び出される共通処理となるプログラム}}.
			.2 bootstrap\textrm{\DTcomment{システムの起動時の処理を行うプログラム}}.
			.3 cache.
			.2 config\textrm{\DTcomment{システムの設定を記述するファイル}}.
			.2 database\textrm{\DTcomment{データベース関連のプログラム}}.
			.3 factories\textrm{\DTcomment{ダミーのレコードの定義をおこなうプログラム}}.
			.3 migrations\textrm{\DTcomment{テーブルの作成などをおこなうプログラム}}.
			.3 seeders\textrm{\DTcomment{ダミーのレコードの挿入をおこなうプログラム}}.
			.2 dev\textrm{\DTcomment{開発用コンテナ関連のファイル}}.
			.3 db\textrm{\DTcomment{開発用コンテナ内のデータベースのマウントポイント}}.
			.2 docs\textrm{\DTcomment{システム提案書や外部設計書、内部設計書のような文書}}.
			.3 figure\textrm{\DTcomment{文書に使用する図}}.
			.3 tmp\textrm{\DTcomment{文書に使用する自動生成された図}}.
			.2 gas\textrm{\DTcomment{Google App Scriptのデプロイに用いるファイル}}.
			.3 dist\textrm{\DTcomment{ViteによってビルドされたGoogle App Script}}.
			.2 node\_modules\textrm{\DTcomment{Node.jsのパッケージ}}.
			.2 public\textrm{\DTcomment{Webサーバのドキュメントルート}}.
			.3 build\textrm{\DTcomment{Viteによってビルドされたフロントエンド関連のファイル}}.
			.4 assets\textrm{\DTcomment{CSSスタイルシートとJavaScriptコード}}.
			.2 resources\textrm{\DTcomment{フロントエンド関連のファイル}}.
			.3 css\textrm{\DTcomment{CSSスタイルシート}}.
			.3 ts\textrm{\DTcomment{TypeScriptコード}}.
			.4 ci\textrm{\DTcomment{CIで使用するスクリプト}}.
			.4 e2e\textrm{\DTcomment{E2Eテスト用のテストスクリプト}}.
			.3 views\textrm{\DTcomment{Bladeテンプレート}}.
			.2 routes\textrm{\DTcomment{ルーティング関連のファイル}}.
			.2 storage\textrm{\DTcomment{Laravelが生成・使用するファイル}}.
			.3 app.
			.4 private.
			.4 public.
			.3 framework.
			.4 cache.
			.5 data.
			.4 sessions.
			.4 testing.
			.4 views.
			.3 logs.
			.2 tests\textrm{\DTcomment{PHPUnitのテストスクリプト}}.
			.3 Feature\textrm{\DTcomment{機能テストのテストスクリプト}}.
			.3 Unit\textrm{\DTcomment{単体テストのテストスクリプト}}.
			.2 vendor\textrm{\DTcomment{Composerのパッケージ}}.
		}
	}
	\caption{主なディレクトリの一覧}\label{fig:dir}
\end{figure}

% ====================================
\newpage
\section{モジュール詳細設計}
% 【記述内容】外部設計書の機能要件を基に、各機能の入出力、責務、および内部処理フローを詳細に定義

\subsection{共通モジュール}
% 【記述内容】ログイン/ログアウト、アカウント作成とかを記述?
\subsubsection{画面の共通部分出力モジュール}
\begin{table}[H]
	\centering
	\caption{画面の共通部分出力モジュール}\label{tab:mod-view-mf}
	\modtable{MV00}{resources/views/mf.blade.php}%
	{画面の共通部分をテンプレートとして出力}%
	{
		a [label="画面の共通部分を出力",shape=box,style=rounded];
	}%
	{}%
	{画面テンプレート}%
	{}
\end{table}
\subsubsection{ログイン画面出力モジュール}
\begin{table}[H]
	\centering
	\caption{ログイン画面出力モジュール}\label{tab:mod-view-login}
	\modtable{MV01}{resources/views/login.blade.php}%
	{ログイン画面を構成する}%
	{
		node [shape=box,style=rounded];
		a [label="画面の共通部分モジュールを呼び出し"];
		b [label="ログイン画面を構成"];
		a -> b;
	}%
	{}%
	{Webページ}%
	{}
\end{table}
構成する画面が異なるのみであり、
画面の共通部分モジュールを呼び出してそれを基に画面を構成する
部分は同じであるため省略するが、原則このモジュールと同様に
各画面に対応したモジュールを作成するものとする。
\subsubsection{ログイン処理モジュール}
\begin{table}[H]
	\centering
	\caption{ログイン処理モジュール}\label{tab:mod-create-login}
	\modtable{MC01}{app/Http/Controllers/LoginController.php}%
	{SceneTripのログインを行う。入力値の検証、ログイン名の重複チェック}%
	{
		node [shape=box,style=rounded];%
		b, h, r, k [shape=diamond];
		{ rank=source; b; }
		b [label="ログイン方式は何か?"];
		g [label="空白検知モジュールを呼び出し"];%
		h [label="入力内容が問題あるか?"];
		i [label="エラーメッセージを含むページを応答"];
		j [label="セッション生成"];%
		k [label="二要素認証は必要か?"];
		l [label="二要素認証画面生成モジュールを呼び出し"];
		m [label="応答"];
		n [label="トップページへのリダイレクトを含む応答"];
		q [label="OAuthを利用して認証を行い、GoogleのIDとユーザー名を取得"];%
		r [label="取得したGoogleのIDが既に登録されているか"];%
		t [label="usersテーブルにGoogleのIDとユーザー名を登録"];%
		b->g [label="SceneTripでログイン"];
		g->h;
		h->i [label="Yes"];%
		h->j [label="No"];%
		j->k;%
		k->l [label="Yes"];%
		k->n [label="No"];
		l->m;
		b->q [label="Googleでログイン"];
		q->r;
		r->k [label="Yes"];%
		r->t [label="No"];
		t->k;
	}%
	{利用者から受け取ったログインに関するHTTPリクエスト}%
	{Webページまたはリダイレクト応答となるHTTPレスポンス}%
	{}
\end{table}

\newpage
\subsubsection{アカウント作成モジュール}
\begin{table}[H]
	\centering
	\caption{アカウント作成モジュール}\label{tab:mod-create-account}
	\modtable{MC02}{app/Http/Controllers/AccountCreateController.php}%
	{SceneTripのアカウント作成を行う。入力値の検証、ログイン名の重複チェック}%
	{
		node [shape=box,style=rounded];
		h [shape=diamond];
		{ rank=source; d; }
		d [label="アカウント作成ボタン選択"];
		e [label="空白検知、特殊文字検知モジュールを呼び出し"];
		f [label="ログイン名重複チェックモジュールを呼び出し"];
		h [label="入力内容が問題あるか"];
		i [label="エラーメッセージを含むページを応答"];
		j [label="usersテーブルにアカウント情報を登録"];
		l [label="ログイン画面へ遷移"];
		d->e->f->h;
		h->i [label="Yes"];
		h->j [label="No"];
		j->l;
		}%
	{利用者から受け取ったアカウント情報を含むHTTPリクエスト}%
	{Webページ}%
	{}
\end{table}

\newpage
\subsubsection{利用規約確認モジュール}
\begin{table}[H]
	\centering
	\caption{利用規約確認モジュール}\label{tab:mod-terms}
	\modtable{MC03}{resources/ts/terms.ts}%
	{SceneTripの利用規約確認を行う。}%
	{
		node [shape=box,style=rounded];
		{ rank=source; c; }
		c [label="「同意する」ボタンの押下を検知"];
		d [label="アカウント作成画面へ遷移"];
		c->d;
	}%
	{}%
	{}%
	{}
\end{table}

\newpage
\subsubsection{空白検知モジュール}
\begin{table}[H]
	\centering
	\caption{空白検知モジュール}\label{tab:mod-blank}
	\modtable{MC04}{app/Traits/BlankCheckTrait.php}%
	{入力値に空白が含まれていないか検査する。}%
	{
		node [shape=box,style=rounded];
		b [shape=diamond];
		{ rank=source; b; }
		b [label="空白や特殊文字が含まれているか"];
		d [label="エラーメッセージを返す"];
		e [label="含まれていない旨を返す"];
		b->d [label="含まれている"];
		b->e [label="含まれていない"];
	}%
	{}%
	{エラーメッセージまたは正常終了の旨}%
	{}
\end{table}

\newpage
\subsubsection{特殊文字検知モジュール}
\begin{table}[H]
	\centering
	\caption{特殊文字検知モジュール}\label{tab:mod-special}
	\modtable{MC05}{app/Traits/SpecialCharCheckTrait.php}%
	{入力値に特殊文字が含まれていないか検査
する。}%
	{
		node [shape=box,style=rounded];
		b [shape=diamond];
		{ rank=source; b; }
		b [label="特殊文字が含まれているか"];
		d [label="エラーメッセージを返す"];
		e [label="含まれていない旨を返す"];
		b->d [label="含まれている"];
		b->e [label="含まれていない"];
	}%
	{}%
	{エラーメッセージまたは含まれていない旨}%
	{}
\end{table}

\newpage
\subsubsection{ログイン名重複チェックモジュール}
\begin{table}[H]
	\centering
	\caption{ログイン名重複チェックモジュール}\label{tab:mod-dup}
	\modtable{MC06}{app/Traits/DupLoginNameCheckTrait.php}%
	{ログイン名が重複していないか検査する。}%
	{
		node [shape=box,style=rounded];
		b [shape=diamond];
		{ rank=source; b; }
		b [label="ログイン名が重複しているか"];
		d [label="エラーメッセージを返す"];
		e [label="重複していない旨を返す"];
		b->d [label="重複している"];
		b->e [label="重複していない"];
	}%
	{}%
	{エラーメッセージまたは重複していない旨}%
	{}
\end{table}

\newpage
\subsubsection{ログアウト確認モジュール}
\begin{table}[H]
	\centering
	\caption{ログアウト確認モジュール}\label{tab:mod-logout}
	\modtable{MC07}{resources/ts/logout-confirm.ts}%
	{ログアウトしても良いか確認する。}%
	{
		node [shape=box,style=rounded];
		b [shape=diamond];
		b [label="ボタンの押下を検知"];
		c [label="ログアウトを行うページに遷移"];
		g [label="前のページに戻る"];
		b->c [label="ログアウト"];
		b->g [label="キャンセル"];
	}%
	{}%
	{}%
	{}
\end{table}

\newpage
\subsubsection{プロフィール画面構成モジュール}
\begin{table}[H]
	\centering
	\caption{プロフィール画面構成モジュール}\label{tab:mod-profile}
	\modtable{MC08}{app/Http/Controllers/ProfileController.php}%
	{SceneTripのプロフィール画面を構成する。}%
	{
		node [shape=box,style=rounded];
		a [label="プロフィールを利用者モジュールを用いて読み込み"];
		b [label="プロフィール画面出力モジュールを呼び出し"];
		a->b;
	}%
	{利用者から受け取ったHTTPリクエスト}%
	{Webページ}%
	{}
\end{table}

\newpage
\subsubsection{プロフィール編集画面構成モジュール}
\begin{table}[H]
	\centering
	\caption{プロフィール編集画面構成モジュール}\label{tab:mod-profile-edit}
	\modtable{MC09}{app/Http/Controllers/ProfileEditController.php}%
	{SceneTripのプロフィール編集画面を表示する。}%
	{
		node [shape=box,style=rounded];
		br1, br2, br3 [shape=diamond];
		{ rank=source; br1; }
		br1 [label="何を編集するのか?"];
		e [label="空白検知モジュールを呼び出し"];
		f [label="ログイン名重複チェックモジュールを呼び出し"];
		br2 [label="入力内容が問題あるか"];
		g [label="エラーメッセージを含むページを応答"];
		j [label="プロフィール更新処理を実行"];
		k [label="プロフィール更新完了画面出力モジュールを呼び出し"];
		br3 [label="入力内容が問題あるか"];
		m [label="エラーメッセージを含むページを応答"];
		n [label="画像アップロード処理を実行"];
		o [label="画像アップロード完了画面出力モジュールを呼び出し"];
		br1->e [label="プロフィール情報"];
		e->f->br2;
		br2->g [label="Yes"];
		br2->j [label="No"];
		j->k;
		br1->br3 [label="アイコン"];
		br3->m [label="Yes"];
		br3->n [label="No"];
		n->o;
	}%
	{利用者から受け取ったアカウント情報を含むHTTPリクエスト}%
	{Webページ}%
	{}
\end{table}



\newpage
\subsubsection{検索画面構成モジュール}
\begin{table}[H]
	\centering
	\caption{検索画面構成モジュール}\label{tab:mod-search-display}
	\modtable{MC00}{app/Http/Controllers/SearchController.php}%
	{検索画面を表示するモジュール}%
	{
		node [shape=box,style=rounded];
		c [shape=diamond];
		b [label="検索画面を生成"];
		c [label="ログインしているか?"];
		d [label="ログイン時のUIを生成"];
		e [label="未ログイン時のUIを生成"];
		b->c;
		c->d [label="Yes"];
		c->e [label="No"];
	}%
	{利用者から受け取ったHTTPリクエスト}%
	{Webページ}%
	{}
\end{table}
\subsection{利用者モジュール}
% 【記述内容】各ステップのロジックを説明になるんかな?。
\subsubsection{観光地・施設検索処理モジュール}

\begin{table}[H]
	\centering
	\caption{観光地・施設検索処理モジュール}\label{tab:mod-search}
	\modtable{MU11}{app/Http/Controllers/SearchApiController.php}%
	{キーワード・カテゴリに基づくスポット検索}%
	{
		node [shape=box,style=rounded];
		c [label="入力内容が問題あるか?", shape=diamond];
		c1 [label="エラーメッセージを応答"];
		d [label="検索実行"];
		d1 [label="結果件数が存在するか?", shape=diamond];
		d2 [label="該当するものがない旨を応答"];
		e [label="結果を応答"];
		c->c1 [label="No"];
		c->d [label="Yes"];
		d->d1;
		d1->d2 [label="No"];
		d1->e [label=Yes];
	}%
	{利用者から受け取った検索クエリを含むHTTPリクエスト}%
	{エラーメッセージまたは検索結果リスト}%
	{}
\end{table}

\subsubsection{AI観光地推薦処理モジュール}
\begin{table}[H]
	\centering
	\caption{AI観光地推薦処理モジュール}\label{tab:mod-ai}
	\modtable{MU13}{app/Http/Controllers/AiApiController.php}
	{ユーザーのチャット入力に基づく観光地推薦}%
	{
		node [shape=box,style=rounded];
		fetchctx [label="周辺情報取得"];
		prompt [label="プロンプト構築"];
		apicall [label="Gemini APIへの送信"];
		checkapi [label="APIの応答が成功であるか?", shape=diamond];
		parse [label="JSON解析"];
		search [label="推薦スポット取得"];
		checkhit [label="結果が問題あるか?", shape=diamond];
		resok [label="推薦文を応答"];
		resng [label="エラーメッセージを応答"];
		fetchctx -> prompt -> apicall -> checkapi;
		checkapi -> resng [label="No"];
		checkapi -> parse [label="Yes"];
		parse -> search -> checkhit;
		checkhit -> resng [label="Yes"];
		checkhit -> resok [label="No"];
	}%
	{利用者から受け取ったチャット入力を含むHTTPリクエスト}%
	{エラーメッセージまたはLLMの出力した推薦文}%
	{}
\end{table}

\subsubsection{チェックイン処理モジュール}
\begin{table}[H]
	\centering
	\caption{チェックイン処理モジュール}\label{tab:mod-checkin}
	\modtable{MU14}{app/Http/Controllers/CheckinApiController.php}{送信された二次元コードと位置情報を基にスタンプを付与}
	{
		node [shape=box,style=rounded];
		decodeqr [label="入力内容が問題あるか?", shape=diamond];
		getspot [label="スポットの情報を取得"];
		calcdist [label="スポットとの距離を計算"];
		checkdist [label="距離が50m以下か?", shape=diamond];
		savestamp [label="スタンプモジュールを呼び出しスタンプを記録"];
		checkcoup [label="クーポン受け取りモジュールを呼び出し"];
		msgerror [label="エラーメッセージを応答"];
		decodeqr -> msgerror [label="YeS"];
		decodeqr -> getspot [label="No"];
		getspot -> calcdist -> checkdist;
		checkdist -> msgerror [label="No"];
		checkdist -> savestamp [label="Yes"];
		savestamp -> checkcoup;
	}%
	{利用者から受け取った二次元コード読み取り結果と位置情報を含むHTTPリクエスト}%
	{エラーメッセージまたはスタンプ・クーポン取得結果}%
	{}
\end{table}

\subsubsection{口コミ画面構成モジュール}
\begin{table}[H]
	\centering
	\caption{口コミ画面構成モジュール}\label{tab:mod-review-photo}
	\modtable{MU15}{app/Http/Controllers/ReviewController.php}{口コミをシステムに登録する}%
	{
		node [shape=box,style=rounded];
		validate [label="入力内容が問題あるか?", shape=diamond];
		hasimg [label="画像があるか?", shape=diamond];
		upload [label="画像を保存"];
		checkup [label="保存に成功したか?", shape=diamond];
		savedb [label="口コミモジュールを用いて口コミを記録"];
		success [label="口コミ画面へのリダイレクトを含む応答"];
		error [label="エラーメッセージを含むページを応答"];
		validate -> error [label="Yes"];
		validate -> hasimg [label="No"];
		hasimg -> savedb [label="No"];
		hasimg -> upload [label="Yes"];
		upload -> checkup;
		checkup -> error [label="失敗"];
		checkup -> savedb [label="成功"];
		savedb -> success;
	}%
	{利用者から受け取った口コミを含むHTTPリクエスト}%
	{Webページまたはリダイレクト応答となるHTTPレスポンス}%
	{}
\end{table}
\subsubsection{クーポン受け取り処理モジュール}
\begin{table}[H]
	\centering
	\caption{クーポン受け取り処理モジュール}\label{tab:mod-coupon-receive}
	\modtable{MU16}{app/Traits/CouponTrait.php}{スタンプ獲得状況に基づくクーポン付与}%
	{
		node [shape=box,style=rounded];
		fetchcond [label="クーポンモジュールを用いて関連クーポン検索"];
		checkhit [label="対象クーポンがあるか?", shape=diamond];
		checkexpire [label="有効期限内であるか?", shape=diamond];
		savedb [label="利用者クーポンモジュールを用いてデータベースに記録"];
		ressuccess [label="獲得に成功した旨を返す"];
		resnone [label="獲得に失敗した旨を返す"];
		fetchcond -> checkhit;
		checkhit -> resnone [label="No"];
		checkhit -> checkexpire [label="Yes"];
		checkexpire -> resnone [label="No"];
		checkexpire -> savedb [label="Yes"];
		savedb -> ressuccess;
	}%
	{チェックインしたスポット}%
	{獲得に成功したかどうか}%
	{\parbox[t]{\dimexpr\textwidth-2\tabcolsep\relax}{スタンプ獲得トリガーで呼び出される想定。ユーザーが同じクーポンを重複して受け取らないようにする}\vspace{.35\zw}}
\end{table}
\newpage
\subsection{管理者モジュール}
% 【記述内容】各ステップのロジックを説明になるんかな?。
% ====================================
\subsubsection{スポット追加画面構成モジュール}
\begin{table}[H]
	\centering
	\caption{スポット追加画面構成モジュール}\label{tab:mod-add-spot}
	\modtable{MA01}{app/Http/Controllers/InsertSpotController.php}%
	{スポット追加画面を構成するモジュール}%
	{
		node [shape=box,style=rounded];
		c [label="入力内容が問題あるか?",shape=diamond];
		f [label="エラーメッセージを含むページを応答"];
		g [label="スポットモジュールを用いてスポット追加処理を実行"];
		h [label="スポット一覧画面へのリダイレクトを含む応答"];
		c->f [label="Yes"];
		c->g [label="No"];
		g->h;
	}%
	{利用者から受け取ったスポット情報を含むHTTPリクエスト}%
	{Webページまたはリダイレクト応答となるHTTPレスポンス}%
	{}
\end{table}

\newpage
\subsubsection{スポット編集画面構成モジュール}
\begin{table}[H]
	\centering
	\caption{スポット編集画面構成モジュール}\label{tab:mod-edit-spot}
	\modtable{MA02}{app/Http/Controllers/EditSpotController.php}%
	{スポット編集画面を構成するモジュール}%
	{
		node [shape=box,style=rounded];
		c [label="入力内容が問題あるか?",shape=diamond];
		f [label="エラーメッセージを含むページを応答"];
		g [label="スポットモジュールを用いてスポット編集処理を実行"];
		h [label="スポット一覧画面へのリダイレクトを含む応答"];
		c->f [label="Yes"];
		c->g [label="No"];
		g->h;
	}%
	{利用者から受け取ったスポット情報を含むHTTPリクエスト}%
	{Webページまたはリダイレクト応答となるHTTPレスポンス}%
	{}
\end{table}

\newpage
\subsubsection{スポット削除画面構成モジュール}
\begin{table}[H]
	\centering
	\caption{スポット削除画面構成モジュール}\label{tab:mod-delete-spot}
	\modtable{MA03}{app/Http/Controllers/DeleteSpotController.php}%
	{スポット削除画面を構成するモジュール}%
	{
		node [shape=box,style=rounded];
		c [label="入力内容が問題あるか?",shape=diamond];
		f [label="エラーメッセージを含むページを応答"];
		g [label="スポットモジュールを用いてスポット削除処理を実行"];
		h [label="スポット一覧画面へのリダイレクトを含む応答"];
		c->f [label="Yes"];
		c->g [label="No"];
		g->h;
	}%
	{利用者から受け取ったスポットのIDを含むHTTPリクエスト}%
	{Webページまたはリダイレクト応答となるHTTPレスポンス}%
	{}
\end{table}

\newpage
\subsubsection{利用者追加画面構成モジュール}
\begin{table}[H]
	\centering
	\caption{利用者追加画面構成モジュール}\label{tab:mod-add-user}
	\modtable{MA04}{app/Http/Controllers/InsertUserController.php}%
	{利用者追加画面を構成するモジュール}%
	{
		node [shape=box,style=rounded];
		c [label="入力内容が問題あるか?",shape=diamond];
		f [label="エラーメッセージを含むページを応答"];
		g [label="利用者モジュールを用いて利用者追加処理を実行"];
		h [label="利用者一覧画面へのリダイレクトを含む応答"];
		c->f [label="Yes"];
		c->g [label="No"];
		g->h;
	}%
	{利用者から受け取った利用者情報を含むHTTPリクエスト}%
	{Webページまたはリダイレクト応答となるHTTPレスポンス}%
	{}
\end{table}

\newpage
\subsubsection{利用者編集画面構成モジュール}
\begin{table}[H]
	\centering
	\caption{利用者編集画面構成モジュール}\label{tab:mod-edit-user}
	\modtable{MA05}{app/Http/Controllers/EditUserController.php}%
	{利用者編集画面を構成するモジュール}%
	{
		node [shape=box,style=rounded];
		c [label="入力内容が問題あるか?",shape=diamond];
		f [label="エラーメッセージを含むページを応答"];
		g [label="利用者モジュールを用いて利用者編集処理を実行"];
		h [label="利用者一覧画面へのリダイレクトを含む応答"];
		c->f [label="Yes"];
		c->g [label="No"];
		g->h;
	}%
	{利用者から受け取った利用者情報を含むHTTPリクエスト}%
	{Webページまたはリダイレクト応答となるHTTPレスポンス}%
	{}
\end{table}

\newpage
\subsubsection{利用者削除画面構成モジュール}
\begin{table}[H]
	\centering
	\caption{利用者削除画面構成モジュール}\label{tab:mod-delete-user}
	\modtable{MA06}{app/Http/Controllers/DeleteUserController.php}%
	{利用者削除画面を構成するモジュール}%
	{
		node [shape=box,style=rounded];
		c [label="入力内容が問題あるか?",shape=diamond];
		f [label="エラーメッセージを含むページを応答"];
		g [label="利用者モジュールを用いて利用者削除処理を実行"];
		h [label="利用者一覧画面へのリダイレクトを含む応答"];
		c->f [label="Yes"];
		c->g [label="No"];
		g->h;
	}%
	{利用者から受け取った利用者のIDを含むHTTPリクエスト}%
	{Webページまたはリダイレクト応答となるHTTPレスポンス}%
	{}
\end{table}

\newpage
\subsubsection{UGC監視・管理画面構成モジュール}
\begin{table}[H]
	\centering
	\caption{UGC監視・管理画面構成モジュール}\label{tab:mod-admin-ugc}
	\modtable{MA07}{app/Http/Controllers/AdminUgcController.php}%
	{UGC監視・管理画面を構成するモジュール}%
	{
		node [shape=box,style=rounded];
		c [label="入力内容が問題あるか?",shape=diamond];
		f [label="エラーメッセージを含むページを応答"];
		g [label="口コミモジュールまたは写真モジュールを用いて削除処理を実行"];
		h [label="UGC監視・管理画面へのリダイレクトを含む応答"];
		c->f [label="Yes"];
		c->g [label="No"];
		g->h;
	}%
	{利用者から受け取ったUGCの種別とIDを含むHTTPリクエスト}%
	{Webページまたはリダイレクト応答となるHTTPレスポンス}%
	{}
\end{table}

\newpage
\subsubsection{サブスクリプション承認画面構成モジュール}
\begin{table}[H]
	\centering
	\caption{サブスクリプション承認画面構成モジュール}\label{tab:mod-approve-subs}
	\modtable{MA08}{app/Http/Controllers/SubscriptionApproveController.php}%
	{サブスクリプション承認画面を表示するモジュール}%
	{
		node [shape=box,style=rounded];
		c [label="入力内容が問題あるか?",shape=diamond];
		f [label="エラーメッセージを含むページを応答"];
		g [label="利用者モジュールを用いて承認処理を実行"];
		h [label="承認要求一覧画面へのリダイレクトを含む応答"];
		c->f [label="Yes"];
		c->g [label="No"];
		g->h;
	}%
	{利用者から受け取った利用者のIDを含むHTTPリクエスト}%
	{Webページまたはリダイレクト応答となるHTTPレスポンス}%
	{}
\end{table}

\section{データベースアクセス設計}
% 【記述内容】データベースへのアクセス方法、外部設計書で指定された特殊なデータ形式やロジック(ID生成など)の詳細を定義

\subsection{ }
\begin{itemize}
	\item
\end{itemize}

\subsection{}
\begin{itemize}
	\item
\end{itemize}

\end{document}
