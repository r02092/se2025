\documentclass{docs}

%--- 基本パッケージ ---%
\usepackage[tmpdir]{graphviz} % フローチャートの作成
\usepackage{nicematrix}       % 複雑な表の作成
\usepackage{tabularx}         % 表の幅を調整
\usepackage{xpatch}           % コマンドの改造

%--- 文書情報 ---%
\title{内部設計書}

\makeatletter
	% GraphvizをLua経由で実行し、
	% 特殊な拡張子を持ったファイルを生成させ、
	% 日本語フォントを使用するよう改造
	% ※Lua経由で実行しないとファイルが生成されない
	% ※Latexmkでのビルド時に生成ファイルの変更が
	%  毎回検知されるため、無視するように設定するために
	%  特殊な拡張子を持たせている
	\renewcommand\@outext{pdflmi}
	\DeclareGraphicsRule{.pdflmi}{pdf}{*}{}
	\xpatchcmd\inputdigraph%
	{\immediate\write18{#3 -T\@outextspace -o \@tmpdir#2.\@outextspace \@tmpdir#2.dot}}%
	{
		\directlua{
			os.execute("#3 -Afontname='"..(os.getenv("GITHUB_ACTIONS")and"Hiragino Sans"or"BIZ UDPGothic").."' -Tpdf -o \@tmpdir#2.pdflmi \@tmpdir#2.dot")
		}
	}%
	{}{\errmessage{改造失敗}}
\makeatother

% モジュール定義の表を描画するマクロ
% \modtable{モジュールID}{名称}{概要}{処理手順}{入力}{出力}{補足}
% 処理手順はGraphvizのDOT言語で記述
\newcommand\modtable[7]{
	\sffamily
	\begingroup
		\setlength\arrayrulewidth{1pt}
		\begin{NiceTabularX}\textwidth{llX}[hvlines]
			\CodeBefore
			\rectanglecolor{lightgray}{1-1}{1-3}
			\rectanglecolor{lightgray}{2-1}{6-2}
			\rectanglecolor{lightgray}{7-1}{7-3}
			\rectanglecolor{lightgray}{9-1}{9-3}
			\rectanglecolor{lightgray}{11-1}{11-3}
			\rectanglecolor{lightgray}{13-1}{13-3}
			\Body
			\Block{1-3}{\emph{モジュール定義}} \\
			\Block{2-1}{\emph{管理情報}} & \emph{システム名} & SceneTrip \\
			& \emph{工程名} & 内部設計 \\
			\Block{3-1}{\emph{基本情報}} & \emph{モジュールID} & #1 \\
			& \emph{名称} & #2 \\
			& \emph{概要} & #3 \\
			\Block{1-3}{\emph{処理手順}} \\
			\Block{1-3}{
				\digraph[scale=.9]{#1}{
					#4
				}
			}
			\\
			\Block{1-3}{\emph{入力}} \\
			\Block[l]{1-3}{#5} \\
			\Block{1-3}{\emph{出力}} \\
			\Block[l]{1-3}{#6} \\
			\Block{1-3}{\emph{補足}} \\
			\Block[l]{1-3}{#7} \\
		\end{NiceTabularX}
	\endgroup
}

\begin{document}

% ====================================
\section{はじめに}
% 【記述内容】本設計書の目的、対象システムの概要、および外部設計書からの変更点を記述。

\section{システム概要}
本書は「聖地巡礼サポートシステムSceneTrip 外部設計書 第2.0版」に基づき、
システムのモジュール構成、データ処理ロジック、およびインターフェースの詳細を
定義する。
\subsection{共通機能}
\subsubsection{アカウント作成機能}

\subsubsection{ログイン・ログアウト機能}

\subsection{利用者(ログイン不要)}
\subsubsection{観光地検索}

\subsection{利用者(ログイン必須)}

\subsubsection{経路作成}

\subsubsection{Google マップ転送機能}

\subsubsection{評価機能}

\subsubsection{写真投稿・共有機能}

\subsubsection{チェックイン機能}

\subsubsection{クーポンの受け取り・利用機能}

\subsection{事業者}
\subsubsection{サブスクリプション登録機能}

\subsubsection{スポット作成機能}

\subsubsection{事業者情報変更機能}

\subsubsection{観光データ分析API}

\subsubsection{クーポン発行機能}

\subsubsection{請求書ダウンロード機能}

\subsection{管理者}
\subsubsection{観光地・施設情報管理}

\subsubsection{ユーザー管理}

\subsubsection{UGCの監視・管理機能}

\subsubsection{データ閲覧・分析機能}

\subsubsection{サブスクリプション承認・解除機能}

\section{適用範囲と制約}
外部設計書で指定されたパッケージ構成および環境設定(開発言語、セキュリティ、
データベースの構成など)を変更せずに作成する。

% ====================================

\section{開発環境}
% 【記述内容】動作環境と同様に開発環境について書く

\section{動作環境}
% 【記述内容】OS、Webサーバー、データベース、ネットワークなど、システムが稼働する環境を具体的に記述
本システムは以下の環境で動作させる。
\begin{table}[H]
	\centering
	\caption{ハードウェア構成}\label{tab:hardware}
	\begin{tabularx}{0.9\textwidth}{|l|p{9\zw}|X|p{10\zw}|}
		\hline
		\thead{項目} & \thead{種類} & \thead{数量} & \thead{備考} \\ \hline
		メインサーバ & OCI Compute & 1台 & Always Freeサービス(無料枠)\\ \hline
		データベースサーバ & OCI Compute & 1台 & Always Freeサービス(無料枠)\\ \hline
		管理者端末 & PCおよびスマートフォン & 管理者数 & \\ \hline
		利用者端末 & PCおよびスマートフォン & 利用者数 & \\ \hline
	\end{tabularx}
\end{table}

\begin{table}[H]
	\centering
	\caption{ソフトウェア構成}\label{tab:software}
	\begin{tabularx}{0.9\textwidth}{|l|X|l|}
		\hline
		\thead{項目} & \thead{ソフトウェア} & \thead{備考} \\ \hline
		Webサーバ & Nginx & \\ \hline
		データベース & MariaDB & \\ \hline
		バックエンド & PHP(Laravel) & \\ \hline
		フロントエンド & TypeScript & \\ \hline
		端末OS & Android、Linux、Windows、iOS、macOS & \\ \hline
		端末ブラウザ & Firefox、Google Chrome、Safari & \\ \hline
		LLM & Gemini 2.5 Flash-Lite & Google AI Studioを使用 \\ \hline
		地図 & OpenStreetMap & \\ \hline
		交通情報 & 国土交通省交通量API & \\ \hline
	\end{tabularx}
\end{table}

\section{開発言語・フレームワーク}
% 【記述内容】使用するプログラミング言語、フレームワーク、主要ライブラリを記述。
\begin{itemize}
	\item
\end{itemize}

\section{コーディング規約}
% 【記述内容】命名規則(クラス名、変数名)、インデント、コメントの書き方、セキュリティ上の規約などを定義。
\begin{itemize}
	\item
\end{itemize}

% ====================================

\section{パッケージ構成およびディレクトリ構造}
% 【記述内容】ソースコード全体を構成するパッケージ(モジュール)の論理的・物理的な配置を定義

\subsection{パッケージ構成図}
% 【記述内容】システム全体の主要モジュールの関係を図示

\subsection{物理ディレクトリ構造}
\begin{verbatim}

\end{verbatim}

% ====================================

\section{モジュール詳細設計}
% 【記述内容】外部設計書の機能要件を基に、各機能の入出力、責務、および内部処理フローを詳細に定義

\subsection{共通モジュール}
% 【記述内容】ログイン/ログアウト、アカウント作成とかを記述?
\newpage
\subsubsection{検索画面の表示}
\begin{table}[H]
	\centering
	\caption{検索画面の表示}\label{tab:mod-search-display}
	\modtable{M00}{RootController.php}%
	{検索画面を表示するモジュール}%
	{
		node [shape=diamond] c;
		node [shape=box,style=rounded] a;
		a [label="/にアクセスされる"];
		b [label="検索画面を生成"];
		c [label="ログインしているか"];
		d [label="ログイン時のUIを生成"];
		e [label="未ログイン時のUIを生成"];
		a->b;
		b->c;
		c->d [label="Yes"];
		c->e [label="No"];
	}%
	{利用者から受け取ったHTTPリクエスト}%
	{ウェブページ}%
	{}
\end{table}
\subsection{利用者モジュール}
% 【記述内容】各ステップのロジックを説明になるんかな?。
\subsubsection{観光地・施設検索処理}

\begin{table}[H]
	\centering
	\caption{観光地・施設検索処理}\label{tab:mod-search}
	\modtable{M11}{RootController.php}%
	{キーワード・カテゴリに基づくスポット検索}%
	{
		node [shape=diamond] c;
		node [shape=box,style=rounded] a;

		a [label="検索開始", shape=];
		b [label="キーワード・カテゴリ受信"];
		c [label="入力値検証", shape=diamond];
		c1 [label="入力エラー表示"];
		d [label="DB検索実行", shape=diamond];
		d1 [label="結果件数>0"];
		d2 [label="「該当無し」表示"];
		e [label="結果表示"];
		f [label="終了"];

		a->b->c;

		c->c1 [label="NG"];
		c->d [label="OK"];

		d->d2 [label="No"];
		d->e [label=Yes]

		d2->f;
		e->f;
	}%
	{}%
	{検索結果リスト または メッセージ}%
	{入力値が空の場合は、仕様に合わせて実装}
\end{table}

\subsubsection{経路作成}

\subsubsection{AIによる観光地・施設推薦}
\begin{table}[H]
    \centering
    \caption{AI観光地推薦}\label{tab:mod-ai}
    \modtable{M14}{ユーザーのチャット入力に基づく観光地推薦}%
    {
        node [shape=diamond] c;
		node [shape=box,style=rounded] a;

        start [label="開始", shape=oval];
        input [label="チャット受信"];
        check_empty [label="空文字チェック", shape=diamond];
        fetch_ctx [label="周辺情報取得"];
        prompt [label="プロンプト構築"];
        api_call [label="Gemini API送信"];
        check_api [label="API応答成功?", shape=diamond];
        parse [label="JSON解析"];
        search [label="推薦スポット取得"];
        check_hit [label="スポットあり?", shape=diamond];
        res_ok [label="推薦回答"];
        res_ng [label="エラー回答", shape=oval];
        end [label="終了", shape=oval];

        start -> input -> check_empty;
        check_empty -> res_ng [label="空"];
        check_empty -> fetch_ctx [label="OK"];
        fetch_ctx -> prompt -> api_call -> check_api;
        check_api -> res_ng [label="失敗"];
        check_api -> parse [label="成功"];
        parse -> search -> check_hit;
        check_hit -> res_ng [label="0件"];
        check_hit -> res_ok [label="1件以上"];
        res_ok -> end;
    }%
    {}%
    {}%
    {}
\end{table}

\subsubsection{チェックイン}
\begin{table}[H]
    \centering
    \caption{チェックイン判定}\label{tab:mod-checkin}
    \modtable{M15}{QRコードと位置情報によるスタンプ付与}
    {
		node [shape=diamond] c;
		node [shape=box,style=rounded] a;

		start [label="開始", shape=oval];
        input [label="QRデータ・GPS受信"];
        decode_qr [label="QR検証(SpotID)", shape=diamond];
        get_spot [label="Spot座標取得"];
        calc_dist [label="距離計算"];
        check_dist [label="距離 <= 50m?", shape=diamond];
        save_stamp [label="スタンプ保存"];
        check_coup [label="クーポン条件?", shape=diamond];
        save_coup [label="クーポン付与"];
        msg_success [label="成功画面"];
        msg_error [label="エラー画面", shape=oval];
        end [label="終了", shape=oval];

        start -> input -> decode_qr;
        decode_qr -> msg_error [label="不正"];
        decode_qr -> get_spot [label="正"];
        get_spot -> calc_dist -> check_dist;
        check_dist -> msg_error [label="範囲外"];
        check_dist -> save_stamp [label="範囲内"];
        save_stamp -> check_coup;
        check_coup -> msg_success [label="未達"];
        check_coup -> save_coup [label="達成"];
        save_coup -> msg_success -> end;
    }%
    {}%
    {スタンプ取得結果, クーポン(条件達成時)}
    {}
\end{table}

\subsubsection{評価・写真投稿}
\begin{table}[H]
    \centering
    \caption{評価・写真投稿}\label{tab:mod-review-photo}
    \modtable{M16}{評価と写真をシステムに登録する}%
    {
		node [shape=diamond] c;
		node [shape=box,style=rounded] a;

        start [label="開始", shape=oval];
        input [label="投稿データ受信"];
        validate [label="入力チェック", shape=diamond];
        has_img [label="画像あり?", shape=diamond];
        upload [label="画像アップロード"];
        check_up [label="アップロード成功?", shape=diamond];
        convert [label="座標変換"];
        save_db [label="DB保存"];
        success [label="完了表示"];
        error [label="エラー表示", shape=oval];
        end [label="終了", shape=oval];

        start -> input -> validate;
        validate -> error [label="NG"];
        validate -> has_img [label="OK"];
        has_img -> save_db [label="No"];
        has_img -> upload [label="Yes"];
        upload -> check_up;
        check_up -> error [label="失敗"];
        check_up -> convert [label="成功"];
        convert -> save_db;
        save_db -> success -> end;
    }%
    {}%
    {投稿完了メッセージ}%
    {}
\end{table}
\subsubsection{クーポン受け取り}
\subsubsection{クーポン受け取り処理}
\begin{table}[H]
    \centering
    \caption{クーポン受け取り処理}\label{tab:mod-coupon-receive}
    \modtable{M-USR-06}{スタンプ獲得状況に基づくクーポン付与}%
    {
		node [shape=diamond] c;
		node [shape=box,style=rounded] a;

        start [label="開始", shape=oval];
        input [label="獲得スタンプID受信"];
        fetch_cond [label="関連クーポン検索\n(cond_spot_id)"];
        check_hit [label="対象クーポンあり?", shape=diamond];
        check_duplicate [label="既に獲得済み?", shape=diamond];
        check_expire [label="有効期限内?", shape=diamond];
        save_db [label="user_coupons保存\n(利用可能状態で作成)"];
        res_success [label="獲得成功通知"];
        res_none [label="獲得なし通知", shape=oval];
        end [label="終了", shape=oval];

        start -> input -> fetch_cond -> check_hit;

        % クーポン存在チェック
        check_hit -> res_none [label="なし"];
        check_hit -> check_duplicate [label="あり"];

        % 重複チェック
        check_duplicate -> res_none [label="Yes"];
        check_duplicate -> check_expire [label="No"];

        % 期限チェック
        check_expire -> res_none [label="期限切"];
        check_expire -> save_db [label="有効"];

        save_db -> res_success -> end;
    }%
    {}%
    {}%
    {スタンプ獲得トリガーで呼び出される想定。ユーザーが同じクーポンを重複して受け取らないよう、必ず\texttt{user\_coupons}を確認するロジックを含める。}
\end{table}


\subsection{事業者モジュール}
% 【記述内容】各ステップのロジックを説明になるんかな?。

\subsection{管理者モジュール}
% 【記述内容】各ステップのロジックを説明になるんかな?。
% ====================================

\section{データベースアクセス設計}
% 【記述内容】データベースへのアクセス方法、外部設計書で指定された特殊なデータ形式やロジック(ID生成など)の詳細を定義

\subsection{ }
\begin{itemize}
	\item
\end{itemize}

\subsection{}
\begin{itemize}
	\item
\end{itemize}

% ====================================
\section{外部インターフェース設計}
% 【記述内容】外部サービスとの連携におけるリクエスト・レスポンスの形式、認証方法を定義

\subsection{「Gemini API」連携仕様}
\begin{itemize}
    \item
\end{itemize}

\subsection{「OpenStreetMap」連携仕様}
\begin{itemize}
    \item
\end{itemize}

\subsection{「Google マップ」連携仕様}

\subsection{「国土交通省交通量API」連携仕様}
\begin{itemize}
    \item
\end{itemize}


\end{document}
