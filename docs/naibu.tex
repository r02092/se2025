\documentclass{docs}

%--- 基本パッケージ ---%
\usepackage[tmpdir]{graphviz} % フローチャートの作成
\usepackage{nicematrix}       % 複雑な表の作成
\usepackage{xpatch}           % コマンドの改造

%--- 文書情報 ---%
\title{内部設計書}

\makeatletter
	% GraphvizをLua経由で実行し、
	% 特殊な拡張子を持ったファイルを生成させ、
	% 日本語フォントを使用するよう改造
	% ※Lua経由で実行しないとファイルが生成されない
	% ※Latexmkでのビルド時に生成ファイルの変更が
	%  毎回検知されるため、無視するように設定するために
	%  特殊な拡張子を持たせている
	\renewcommand\@outext{pdflmi}
	\DeclareGraphicsRule{.pdflmi}{pdf}{*}{}
	\xpatchcmd\inputdigraph%
	{\immediate\write18{#3 -T\@outextspace -o \@tmpdir#2.\@outextspace \@tmpdir#2.dot}}%
	{
		\directlua{
			os.execute("#3 -Afontname='"..(os.getenv("GITHUB_ACTIONS")and"Hiragino Sans"or"BIZ UDPGothic").."' -Tpdf -o \@tmpdir#2.pdflmi \@tmpdir#2.dot")
		}
	}%
	{}{\errmessage{改造失敗}}
\makeatother
\setlength\arrayrulewidth{1pt}

% モジュール定義の表を描画するマクロ
% \modtable{モジュールID}{概要}{処理手順}{入力}{出力}{補足}
% 処理手順はGraphvizのDOT言語で記述
\newcommand\modtable[6]{
	\sffamily
	\begin{NiceTabularX}\textwidth{llX}[hvlines]
		\CodeBefore
		\rectanglecolor{lightgray}{1-1}{1-3}
		\rectanglecolor{lightgray}{2-1}{5-2}
		\rectanglecolor{lightgray}{6-1}{6-3}
		\rectanglecolor{lightgray}{8-1}{8-3}
		\rectanglecolor{lightgray}{10-1}{10-3}
		\rectanglecolor{lightgray}{12-1}{12-3}
		\Body
		\Block{1-3}{\emph{モジュール定義}} \\
		\Block{2-1}{\emph{管理情報}} & \emph{システム名} & SceneTrip \\
		& \emph{工程名} & 内部設計 \\
		\Block{2-1}{\emph{基本情報}} & \emph{モジュールID} & #1 \\
		& \emph{概要} & #2 \\
		\Block{1-3}{\emph{処理手順}} \\
		\Block{1-3}{
			\digraph[scale=.9]{#1}{
				#3
			}
		}
		\\
		\Block{1-3}{\emph{入力}} \\
		\Block[l]{1-3}{#4} \\
		\Block{1-3}{\emph{出力}} \\
		\Block[l]{1-3}{#5} \\
		\Block{1-3}{\emph{補足}} \\
		\Block[l]{1-3}{#6} \\
	\end{NiceTabularX}
}

\begin{document}

% ====================================
\section{はじめに}
% 【記述内容】本設計書の目的、対象システムの概要、および外部設計書からの変更点を記述。

\subsection{システム概要}
本書は、外部設計書(SceneTrip 第2.0版)に基づき、システムのモジュール構成、データ処理ロジック、およびインターフェースの詳細を定義する。

\subsection{適用範囲と制約}
本設計は、外部設計書で指定されたパッケージ構成および環境設定(PHP、Argon2id、特定のデータベースアクセスロジックなど)を変更せずに作成する。

% ====================================

\section{開発環境}
% 【記述内容】動作環境と同様に開発環境について書く

\section{動作環境}
% 【記述内容】OS、Webサーバー、データベース、ネットワークなど、システムが稼働する環境を具体的に記述
本アプリケーションSceneTripは以下の環境で動作することを想定している。
\begin{table}[H]
	\centering
	\caption{ハードウェア構成}\label{tab:hardware}
	\begin{tabularx}{0.9\textwidth}{|l|p{9\zw}|X|p{10\zw}|}
		\hline
		\thead{項目} & \thead{種類} & \thead{数量} & \thead{備考} \\ \hline
		メインサーバ & OCI Compute & 1台 & Always Freeサービス(無料枠)\\ \hline
		データベースサーバ & OCI Compute & 1台 & Always Freeサービス(無料枠)\\ \hline
		管理者端末 & PCおよびスマートフォン & 管理者数 & \\ \hline
		利用者端末 & PCおよびスマートフォン & 利用者数 & \\ \hline
	\end{tabularx}
\end{table}

\begin{table}[H]
	\centering
	\caption{ソフトウェア構成}\label{tab:software}
	\begin{tabularx}{0.9\textwidth}{|l|X|l|}
		\hline
		\thead{項目} & \thead{ソフトウェア} & \thead{備考} \\ \hline
		Webサーバ & Nginx & \\ \hline
		データベース & MariaDB & \\ \hline
		バックエンド & PHP(Laravel) & \\ \hline
		フロントエンド & TypeScript & \\ \hline
		端末OS & Android、Linux、Windows、iOS、macOS & \\ \hline
		端末ブラウザ & Firefox、Google Chrome、Safari & \\ \hline
		LLM & Gemini 2.5 Flash-Lite & Google AI Studioを使用 \\ \hline
		地図 & OpenStreetMap & \\ \hline
		交通情報 & 国土交通省交通量API & \\ \hline
	\end{tabularx}
\end{table}

\section{開発言語・フレームワーク}
% 【記述内容】使用するプログラミング言語、フレームワーク、主要ライブラリを記述。
\begin{itemize}
	\item
\end{itemize}

\section{コーディング規約}
% 【記述内容】命名規則(クラス名、変数名)、インデント、コメントの書き方、セキュリティ上の規約などを定義。
\begin{itemize}
	\item
\end{itemize}

% ====================================

\section{パッケージ構成およびディレクトリ構造}
% 【記述内容】ソースコード全体を構成するパッケージ(モジュール)の論理的・物理的な配置を定義

\subsection{パッケージ構成図}
% 【記述内容】システム全体の主要モジュールの関係を図示

\subsection{物理ディレクトリ構造}
\begin{verbatim}

\end{verbatim}

% ====================================

\section{モジュール詳細設計}
% 【記述内容】外部設計書の機能要件を基に、各機能の入出力、責務、および内部処理フローを詳細に定義

\subsection{共通モジュール}
% 【記述内容】ログイン/ログアウト、アカウント作成とかを記述?
\newpage
\subsubsection{検索画面の表示}
\begin{table}[H]
	\centering
	\caption{検索画面の表示}\label{tab:mod-search}
	\modtable{M00}{検索画面を表示するモジュール}%
	{
		node [shape=diamond] c;
		node [shape=box,style=rounded] a;
		a [label="/にアクセスされる"];
		b [label="検索画面を生成"];
		c [label="ログインしているか"];
		d [label="ログイン時のUIを生成"];
		e [label="未ログイン時のUIを生成"];
		a->b;
		b->c;
		c->d [label="Yes"];
		c->e [label="No"];
	}%
	{利用者から受け取ったHTTPリクエスト}%
	{ウェブページ}%
	{}
\end{table}
\subsection{利用者モジュール}
% 【記述内容】各ステップのロジックを説明になるんかな?。

\subsection{事業者モジュール}
% 【記述内容】各ステップのロジックを説明になるんかな?。

\subsection{管理者モジュール}
% 【記述内容】各ステップのロジックを説明になるんかな?。
% ====================================

\section{データベースアクセス設計}
% 【記述内容】データベースへのアクセス方法、外部設計書で指定された特殊なデータ形式やロジック(ID生成など)の詳細を定義

\subsection{ }
\begin{itemize}
	\item
\end{itemize}

\subsection{}
\begin{itemize}
	\item
\end{itemize}

% ====================================
\section{外部インターフェース設計}
% 【記述内容】外部サービスとの連携におけるリクエスト・レスポンスの形式、認証方法を定義

\subsection{Gemini API 連携仕様}
\begin{itemize}
    \item
\end{itemize}

\subsection{OpenStreetMap / Google Maps 連携仕様}
\begin{itemize}
    \item
\end{itemize}

\subsection{国土交通省 交通量API 連携仕様}
\begin{itemize}
    \item
\end{itemize}


\end{document}
