\documentclass{docs}

%--- 基本パッケージ ---%
\usepackage[tmpdir]{graphviz} % フローチャートの作成
\usepackage{nicematrix}       % 複雑な表の作成
\usepackage{tabularx}         % 表の幅を調整
\usepackage{xpatch}           % コマンドの改造

%--- 文書情報 ---%
\title{内部設計書}

\makeatletter
	% GraphvizをLua経由で実行し、
	% 特殊な拡張子を持ったファイルを生成させ、
	% 日本語フォントを使用するよう改造
	% ※Lua経由で実行しないとファイルが生成されない
	% ※Latexmkでのビルド時に生成ファイルの変更が
	%  毎回検知されるため、無視するように設定するために
	%  特殊な拡張子を持たせている
	\renewcommand\@outext{pdflmi}
	\DeclareGraphicsRule{.pdflmi}{pdf}{*}{}
	\xpatchcmd\inputdigraph%
	{\immediate\write18{#3 -T\@outextspace -o \@tmpdir#2.\@outextspace \@tmpdir#2.dot}}%
	{
		\directlua{
			os.execute("#3 -Afontname='"..(os.getenv("GITHUB_ACTIONS")and"Hiragino Sans"or"BIZ UDPGothic").."' -Tpdf -o \@tmpdir#2.pdflmi \@tmpdir#2.dot")
		}
	}%
	{}{\errmessage{改造失敗}}
\makeatother

% モジュール定義の表を描画するマクロ
% \modtable{モジュールID}{名称}{概要}{処理手順}{入力}{出力}{補足}
% 処理手順はGraphvizのDOT言語で記述
\newcommand\modtable[7]{
	\sffamily
	\begingroup
		\setlength\arrayrulewidth{1pt}
		\begin{NiceTabularX}\textwidth{llX}[hvlines]
			\CodeBefore
			\rectanglecolor{lightgray}{1-1}{1-3}
			\rectanglecolor{lightgray}{2-1}{6-2}
			\rectanglecolor{lightgray}{7-1}{7-3}
			\rectanglecolor{lightgray}{9-1}{9-3}
			\rectanglecolor{lightgray}{11-1}{11-3}
			\rectanglecolor{lightgray}{13-1}{13-3}
			\Body
			\Block{1-3}{\emph{モジュール定義}} \\
			\Block{2-1}{\emph{管理情報}} & \emph{システム名} & SceneTrip \\
			& \emph{工程名} & 内部設計 \\
			\Block{3-1}{\emph{基本情報}} & \emph{モジュールID} & #1 \\
			& \emph{名称} & #2 \\
			& \emph{概要} & #3 \\
			\Block{1-3}{\emph{処理手順}} \\
			\Block{1-3}{
				\digraph[scale=.3]{#1}{
					#4
				}
			}
			\\
			\Block{1-3}{\emph{入力}} \\
			\Block[l]{1-3}{#5} \\
			\Block{1-3}{\emph{出力}} \\
			\Block[l]{1-3}{#6} \\
			\Block{1-3}{\emph{補足}} \\
			\Block[l]{1-3}{#7} \\
		\end{NiceTabularX}
	\endgroup
}

\begin{document}

% ====================================
\section{はじめに}
% 【記述内容】本設計書の目的、対象システムの概要、および外部設計書からの変更点を記述。

\section{システム概要}
本書は「聖地巡礼サポートシステムSceneTrip 外部設計書 第2.0版」に基づき、
システムのモジュール構成、データ処理ロジック、およびインターフェースの詳細を
定義する。
\subsection{共通機能}
\subsubsection{アカウント作成機能}
システムを利用するためにはアカウントを作成する必要がある。
アカウント作成時にはアカウント作成画面から
「ユーザー名」、「ログイン名」、「パスワード」、「確認用パスワード」
の4つの文字列を受け取り、文字列の検証とアカウントの作成を行う。
登録したアカウント情報(プロフィール)はログイン名を除き、プロフィール画面で編集可能とする。

\subsubsection{ログイン・ログアウト機能}
本システムにはアカウントの作成に伴い、ログイン及びログアウトの機能がある。
ログインを行うことで、ユーザは本システムの全ての機能を利用できる。
一方ログインを行わない場合、利用できる機能は観光地検索機能に制限される。
ログインには本システムで作成するアカウントを使用する方法とGoogleアカウントを使用する方法がある。
本システムで作成したアカウントを使用する方法では、
システムはログイン用画面でログイン名とパスワードを受け取り、検証を行う。
Googleアカウントを使用する方法では、OAuthを通して認証を行い、
GoogleのIDとユーザ名から検証を行う。
また、RFC6238に準拠したTOTPによる二要素認証を受け付けており、
プロフィール画面から設定を行えるとする。

\subsection{利用者(ログイン不要)}
\subsubsection{観光地・施設検索}
検索処理は、キーワードとカテゴリを入力として受け取り、入力値の検証を行う。
検索結果には、観光地や施設の詳細情報を表示する。

\subsection{利用者(ログイン必須)}

\subsubsection{経路作成}
ユーザが入力した出発地・到着地となる観光地・施設から経路を生成する。
また、Gemini APIを利用して、Geminiから観光地・施設の提案を行う。
提案した観光地・施設は出発地から到着地の中継地点として経路に含める。
本システムでは経路をノードとエッジを用いたグラフとして表す。

\subsubsection{Google マップ転送機能}
ユーザとGeminiからの入力によって作成した経路から
Google マップ上で表示できるURIを生成する。
ユーザはURIを基にGoogle マップへと移動する。

\subsubsection{評価機能}
観光地・施設に5段階の評価とコメントを投稿することができる。
システムはそれらをスポットに関連付ける。

\subsubsection{写真投稿・共有機能}
場所に関連付けて地図上に表示される写真及びコメントを投稿することができる。
システムは写真とコメントを該当する経度、緯度と関連付ける。

\subsubsection{チェックイン機能}
訪れた観光地・施設のスタンプを現地まで赴いた実績として入手できる機能である。
システムはユーザのカメラ映像から二次元コードを読み取り、
ユーザの位置情報を基に観光地・施設から半径50mの範囲にユーザが位置しているか検証を行う。

\subsubsection{クーポンの受け取り・利用機能}
本システム上で提供されるクーポンの受け取り、並びにそのクーポンの利用ができる。
クーポンは観光地・施設等での割引や特典に利用できる。
クーポンは、利用可能・利用中・利用済み・期限切れの4状態で管理する。
クーポンの「利用中」とは、クーポンを提示して店舗等で利用手続きを行った状態を指し、
翌日5時には「利用済み」となる。
ユーザーは画面上でクーポンが「利用可能」なのか、「利用中」なのかを確認できるものとする。

\subsection{事業者}
\subsubsection{サブスクリプション登録機能}
本システムのスタンダードプランまたはプレミアムプランへの登録の申請を行う機能である。
登録されたプランは、各プランの契約数としてアカウントに紐づく。
また、申請時に郵便番号と住所を受け取り、アカウントに関連付ける。

\subsubsection{スポット(観光地・施設)情報管理機能}
スポットの追加、情報の編集、削除を行う機能である。
システムはスポットを追加する際にスポットの名前、種別、所在地を取得し、スポットに関連付ける。
追加したスポットの情報は後で編集、削除することができる。

\subsubsection{事業者情報変更機能}
登録した店舗情報(店舗名、所在地、営業時間、紹介文、店舗写真)を変更する機能である。

\subsubsection{観光データ分析API}
システムは逐一、ユーザの観光地・施設検索の検索内容やユーザのIDなどを収集し、
観光データ分析に生かせるAPIの機能を提供する。

\subsubsection{クーポン発行機能}
システム上でクーポンを発行できる。
これはサブスクリプションに登録したアカウントのみで行うことができるものとする。
作成時に使用できるスポット、クーポン名、行く必要のあるスポット、期限を入力する必要がある。

\subsubsection{請求書ダウンロード機能}
受領した請求書をPDF形式でダウンロードできるものとする。

\subsection{管理者}

\subsubsection{ユーザー管理}
アカウントの作成、ユーザー情報の編集、削除を行う機能である。

\subsubsection{UGCの監視・管理機能}
UGCの監視、投稿の削除ができる機能である。
システムは管理者が投稿された口コミやフォトスポット投稿について、
利用規約違反があると判断した場合、
投稿の削除を行えるものとする。

\subsubsection{データ閲覧・分析機能}

\subsubsection{サブスクリプション承認・解除機能}

\section{適用範囲と制約}
外部設計書で指定されたパッケージ構成および環境設定(開発言語、セキュリティ、
データベースの構成など)を変更せずに作成する。

% ====================================

\section{技術スタック}
% 【記述内容】使用するプログラミング言語、フレームワーク、主要ライブラリを記述。
本システムは以下の技術を使用する。
\begin{table}[H]
	\centering
	\caption{技術スタック}\label{tab:tech}
	\begin{tabularx}{0.9\textwidth}{|p{13\zw}|X|l|}
		\hline
		\thead{項目} & \thead{ソフトウェア} & \thead{備考} \\ \hline
		Webサーバ & Nginx & \\ \hline
		データベース & MariaDB & \\ \hline
		バックエンド開発言語 & PHP & \\ \hline
		バックエンドパッケージマネージャ & Composer & \\ \hline
		バックエンドフレームワーク & Laravel & \\ \hline
		バックエンド単体テスト & PHPUnit & \\ \hline
		ソーシャルログイン & Laravel Socialite & \\ \hline
		フロントエンド開発言語 & TypeScript & \\ \hline
		フロントエンドランタイム & Node.js & \\ \hline
		Node.jsバージョン管理 & Node Version Manager & \\ \hline
		フロントエンドパッケージマネージャ & pnpm & \\ \hline
		フロントエンドバンドラ & Vite & \\ \hline
		フロントエンドテスト & Vitest & \\ \hline
		ブラウザテスト & Playwright & \\ \hline
		地図表示 & MapLibre GL JS & \\ \hline
		LLM & Gemini 2.5 Flash-Lite & Google AI Studioを使用 \\ \hline
		地図 & OpenStreetMap & \\ \hline
		交通情報 & 国土交通省交通量API & \\ \hline
	\end{tabularx}
\end{table}

\section{開発環境}
% 【記述内容】動作環境と同様に開発環境について書く
本システムは、GitHubでソースコードを管理し、GitHubに
搭載されたCI/CDツールであるGitHub Actionsを開発に用いる。
本文書を含めたシステムに関わる文書を管理するリポジトリと共通のリポジトリを
用いる。
また、コードフォーマッタであるPrettierおよび
リンタであるESLintとLarastanを用いる。

GNU/LinuxシステムまたはWindowsで動作するPCを開発用PCとして用いる。
開発用PCには、Node Version Managerをインストールし、
それを用いてNode.jsをインストールし、それに含まれるnpmを用いてpnpmを
インストールするものとする。
また、開発用PCにコンテナを用いて以下の環境を作成し、開発をおこなう。
コンテナ管理ツールとしてはPodmanまたはDockerを用いる。
\begin{table}[H]
	\centering
	\caption{開発環境}\label{tab:dev-con}
	\begin{tabularx}{0.9\textwidth}{|l|X|p{10\zw}|}
		\hline
		\thead{項目} & \thead{ソフトウェア} & \thead{備考} \\ \hline
		OS & Alpine 3.23 & \\ \hline
		PHP実行環境 & PHP 8.3 & \\ \hline
		データベース & MariaDB 10.11.14 & \\ \hline
		自動レビュ & Reviewdog v0.18.1 & \\ \hline
	\end{tabularx}
\end{table}
GitHub Actionsでは、以下の作業を自動でおこなう。
\begin{itemize}
	\item \emph{\LaTeX 文書のコンパイル}
	\begin{itemize}
		\item 特定の形式のGitタグがプッシュされた際に動作
		\item 文書に用いるフォントの都合上、macOSのイメージ上で動作
		\item Mac\TeX の環境の構築に時間がかかるため、6日毎に
		環境のキャッシュを作成
	\end{itemize}
	\item \emph{Pull Requestの自動確認}
	\begin{itemize}
		\item フォーマッタによる整形を自動でコミット
		\item リンタによる指摘箇所をReviewdogでコメント
		\item リンタによる指摘内容が英語の場合、日本語に自動翻訳
		\item PHPUnitによるバックエンドの自動での単体テスト
	\end{itemize}
	\item \emph{Google App Scriptのデプロイ}
	\begin{itemize}
		\item 翻訳APIの呼び出しに使用
	\end{itemize}
	\item \emph{システム全体のデプロイ}
\end{itemize}
以下に、GitHub Actionsワークフローで用いる環境を示す。
この表に書いていない細かなソフトウェアは、
パッケージ管理システム(UbuntuであればAPT、macOSで
あればHomebrew)で導入されるバージョンを使用する。
\begin{table}[H]
	\centering
	\caption{\LaTeX 文書をコンパイルするGitHub Actionsワークフローで用いる環境}\label{tab:dev-latex}
	\begin{tabularx}{0.9\textwidth}{|l|X|p{10\zw}|}
		\hline
		\thead{項目} & \thead{ソフトウェア} & \thead{備考} \\ \hline
		OS & macOS Sequoia & \\ \hline
		\TeX ディストリビューション & Mac\TeX{} 2025 & \\ \hline
	\end{tabularx}
\end{table}
\begin{table}[H]
	\centering
	\caption{その他のGitHub Actionsワークフローで用いる環境}\label{tab:dev-gha}
	\begin{tabularx}{0.9\textwidth}{|l|X|p{10\zw}|}
		\hline
		\thead{項目} & \thead{ソフトウェア} & \thead{備考} \\ \hline
		OS & Ubuntu 24.04.3 & \\ \hline
		PHP実行環境 & PHP 8.3 & \\ \hline
		データベース & MariaDB 10.11.13 & \\ \hline
		自動レビュ & Reviewdog v0.18.1 & \\ \hline
	\end{tabularx}
\end{table}

また、リポジトリには依存関係のアップデートをおこない、Pull Requestを
自動で作成するbotであるRenovateを導入する。
これにより、Composerおよびpnpmで管理する依存関係やGitHub Actionsの
依存関係、Node.jsのアップデートを自動でおこない、最新の状態に保つ。

\section{動作環境}
% 【記述内容】OS、Webサーバー、データベース、ネットワークなど、システムが稼働する環境を具体的に記述
本システムは以下の環境で動作させる。
\begin{table}[H]
	\centering
	\caption{動作環境}\label{tab:honban}
	\begin{tabularx}{0.9\textwidth}{|p{6\zw}|X|p{4\zw}|p{10\zw}|}
		\hline
		\thead{項目} & \thead{種類} & \thead{数量} & \thead{備考} \\ \hline
		メインサーバ & OCI Compute & 1台 & Always Freeサービス(無料枠)\\ \hline
		データベースサーバ & OCI Compute & 1台 & Always Freeサービス(無料枠)\\ \hline
		管理者端末 & PCおよびスマートフォン & 管理者数 & \\ \hline
		利用者端末 & PCおよびスマートフォン & 利用者数 & \\ \hline
		端末OS & Android、Linux、Windows、iOS、macOS & & \\ \hline
		端末ブラウザ & Firefox、Google Chrome、Safari & & \\ \hline
	\end{tabularx}
\end{table}

\section{コーディング規約}
% 【記述内容】命名規則(クラス名、変数名)、インデント、コメントの書き方、セキュリティ上の規約などを定義。
\begin{itemize}
	\item
\end{itemize}

% ====================================

\section{パッケージ構成およびディレクトリ構造}
% 【記述内容】ソースコード全体を構成するパッケージ(モジュール)の論理的・物理的な配置を定義

\subsection{パッケージ構成図}
% 【記述内容】システム全体の主要モジュールの関係を図示

\subsection{物理ディレクトリ構造}
\begin{verbatim}

\end{verbatim}

% ====================================

\section{モジュール詳細設計}
% 【記述内容】外部設計書の機能要件を基に、各機能の入出力、責務、および内部処理フローを詳細に定義

\subsection{共通モジュール}
% 【記述内容】ログイン/ログアウト、アカウント作成とかを記述?
\newpage
\subsubsection{ログイン画面の表示}
\begin{table}[H]
	\centering
	\caption{ログイン画面の表示}\label{tab:mod-create-login}
	\modtable{M01}{loginController.php}%
	{SceneTripのログインを行う。入力値の検証、ログイン名の重複チェック}%
	{
		node [shape=box,style=rounded];%
		b, h, k, l, r [shape=diamond];%
		{ rank=source; a; }
		a [label="モジュール呼び出し"];%
		b [label="ボタン選択"];%
		c [label="「アカウントをお持ちでない方はこちら」選択"];%
		d [label="アカウント作成モジュールへ遷移"];%
		e [label="入力項目へ入力"];%
		f [label="ログインボタン選択"];%
		g [label="空白検知モジュールへ遷移"];%
		h [label="入力内容問題があるか"];%
		i [label="エラーメッセージ表示"];%
		j [label="セッション生成"];%
		k [label="ログイン確認"];%
		l [label="一般利用者か事業者か管理者か"];%
		m [label="一般利用者トップモジュールへ"];%
		n [label="事業者トップモジュールへ"];%
		o [label="管理者トップモジュールへ"];%
		p [label="Google認証選択"];%
		q [label="OAuthを利用して認証を行い、GoogleIDとユーザー名を取得"];%
		r [label="取得したGoogleIDが既に登録されているか"];%
		s [label="新規アカウント作成"];%
		t [label="usersテーブルにGoogleIDとユーザー名を登録"];%
		a->b;
		b->c->d;%
		b->e->f->g->h;%
		h->i [label="Yes"];%
		i->a;%
		h->j [label="No"];%
		j->k;%
		k->l [label="Yes"];%
		k->a [label="No"];%
		l->m [label="一般利用者"];%
		l->n [label="事業者"];%
		l->o [label="管理者"];%
		b->p [label="Google認証選択"];%
		p->q->r;%
		r->k [label="Yes"];%
		r->s [label="No"];%
		s->t->k;%
	}%
	{}%
	{ウェブページ}%
	{}
\end{table}

\newpage
\subsubsection{アカウント作成モジュール}
\begin{table}[H]
	\centering
	\caption{アカウント作成モジュール}\label{tab:mod-create-account}
	\modtable{M02}{AccountCreater.php}%
	{SceneTripのアカウント作成を行う。入力値の検証、ログイン名の重複チェック}%
	{
		node [shape=box,style=rounded];
		b [shape=diamond];
		{ rank=source; a; }
		a [label="モジュール呼び出し"];
		aEx [label="利用規約確認モジュールへ遷移"];
		b [label="ボタン選択"];
		c [label="入力項目へ入力"];
		d [label="アカウント作成ボタン選択"];
		e [label="空白検知、特殊文字検知モジュールへ遷移"];
		f [label="ログイン名重複チェックモジュールへ遷移"];
		g [label="パスワード確認モジュールへ遷移"];
		h [label="入力内容問題があるか"];
		i [label="エラーメッセージ表示"];
		j [label="usersテーブルにアカウント情報を登録"];
		k [label="登録成功か"];
		l [label="ログイン画面へ遷移"];
		m [label="「すでにアカウントをお持ちの方はこちら」を選択"];
		a->aEx->b;
		b->c [label="アカウント作成"];
		c->d->e->f->g->h;
		h->i [label="Yes"];
		i->a;
		h->j [label="No"];
		j->k->l;
		b->m [label="すでにアカウントをお持ちの方"];
		m->l;
		}%
	{}%
	{ウェブページ}%
	{}
\end{table}

\newpage
\subsubsection{利用規約確認モジュール}
\begin{table}[H]
	\centering
	\caption{利用規約確認モジュール}\label{tab:mod-terms}
	\modtable{M04}{Terms.php}%
	{SceneTripの利用規約確認を行う。}%
	{
		node [shape=box,style=rounded];
		c [shape=diamond];
		{ rank=source; a; }
		a [label="モジュール呼び出し"];
		b [label="利用規約を表示"];
		c [label="「同意する」選択"];
		d [label="アカウント作成画面へ遷移"];
		a->b;
		b->c;
		c->d [label="同意する"];
		c->b [label="同意しない"];
	}%
	{}%
	{ウェブページ}%
	{}
\end{table}

\newpage
\subsubsection{空白検知モジュール}
\begin{table}[H]
	\centering
	\caption{空白検知モジュール}\label{tab:mod-blank}
	\modtable{M05}{BlankChecker.php}%
	{SceneTripの入力値に空白が含まれていないか検査する。}%
	{
		node [shape=box,style=rounded];
		b [shape=diamond];
		{ rank=source; a; }
		a [label="モジュール呼び出し"];
		c [label="入力値を検査"];
		b [label="空白や特殊文字が含まれているか"];
		d [label="エラーメッセージを返す"];
		e [label="正常終了を返す"];
		a->c->b;
		b->d [label="含まれている"];
		b->e [label="含まれていない"];
	}%
	{}%
	{エラーメッセージ または 正常終了}%
	{}
\end{table}

\newpage
\subsubsection{特殊文字検知モジュール}
\begin{table}[H]
	\centering
	\caption{特殊文字検知モジュール}\label{tab:mod-special}
	\modtable{M06}{SpecialCharChecker.php}%
	{SceneTripの入力値に特殊文字が含まれていないか検査
する。}%
	{
		node [shape=box,style=rounded];
		b [shape=diamond];
		{ rank=source; a; }
		a [label="モジュール呼び出し"];
		c [label="入力値を検査"];
		b [label="特殊文字が含まれているか"];
		d [label="エラーメッセージを返す"];
		e [label="正常終了を返す"];
		a->c->b;
		b->d [label="含まれている"];
		b->e [label="含まれていない"];
	}%
	{}%
	{エラーメッセージ または 正常終了}%
	{}
\end{table}

\newpage
\subsubsection{ログイン名重複チェックモジュール}
\begin{table}[H]
	\centering
	\caption{ログイン名重複チェックモジュール}\label{tab:mod-dup}
	\modtable{M07}{DupLoginNameChecker.php}%
	{SceneTripのログイン名が重複していないか検査する。}%
	{
		node [shape=box,style=rounded];
		b [shape=diamond];
		{ rank=source; a; }
		a [label="モジュール呼び出し"];
		c [label="入力値を検査"];
		b [label="ログイン名が重複しているか"];
		d [label="エラーメッセージを返す"];
		e [label="正常終了を返す"];
		a->c->b;
		b->d [label="重複している"];
		b->e [label="重複していない"];
	}%
	{}%
	{エラーメッセージ または 正常終了}%
	{}
\end{table}

\newpage
\subsection{ログアウト画面の表示}
\begin{table}[H]
	\centering
	\caption{ログアウト画面の表示}\label{tab:mod-logout}
	\modtable{M03}{AccountCreater.php}%
	{SceneTripのログアウトを行う。}%
	{
		node [shape=box,style=rounded];
		b [shape=diamond];
		a [label="ボタン選択"];
		b [label="続行を選択"];
		c [label="ログアウト確認"];
		d [label="ログアウト処理"];
		e [label="ログアウトのポップアップ表示"];
		f [label="ホーム画面へ遷移"];
		g [label="キャンセル選択"];
		h [label="前のモジュールに戻る"];
		a->b;
		b->c [label="ログアウト"];
		c->d->e->f;
		b->g [label="キャンセル"];
		g->h;
	}%
	{}%
	{ウェブページ}%
	{}
\end{table}

\newpage
\subsubsection{プロフィール画面の表示}
\begin{table}[H]
	\centering
	\caption{プロフィール画面の表示}\label{tab:mod-profile}
	\modtable{M08}{ProfileController.php}%
	{SceneTripのプロフィール画面を表示する。}%
	{
		node [shape=diamond];
		node [shape=box,style=rounded] a;
		a [label="モジュール呼び出し"];
		b [label="プロフィール画面を生成"];
		c [label="ボタン選択"];
		d [label="プロフィール編集画面へ遷移"];
		e [label="事業者申込画面へ遷移"];
		a->b->c;
		c->d [label="プロフィール編集を選択"];
		c->e [label="登録を選択"];
	}%
	{}%
	{ウェブページ}%
	{}
\end{table}

\newpage
\subsubsection{プロフィール編集画面の表示}
\begin{table}[H]
	\centering
	\caption{プロフィール編集画面の表示}\label{tab:mod-profile-edit}
	\modtable{M09}{ProfileController.php}%
	{SceneTripのプロフィール編集画面を表示する。}%
	{
		node [shape=box,style=rounded];
		br1, br2, br3 [shape=diamond];
		{ rank=source; a; }
		a [label="編集ボタン選択"];
		b [label="プロフィール編集画面を生成"];
		br1 [label="ボタン選択"];
		c [label="必要情報を入力"];
		d [label="保存ボタン選択"];
		e [label="空白検知モジュールへ遷移"];
		f [label="ログイン名重複チェックモジュールへ遷移"];
		br2 [label="入力内容問題があるか"];
		g [label="エラーメッセージ表示"];
		h [label="SQLクエリを生成"];
		i [label="プロフィール更新できない旨ポップアップを表示"];
		j [label="プロフィール更新処理を実行"];
		k [label="プロフィール更新完了画面を生成・表示"];
		l [label="画像アップロードモジュールへ遷移"];
		br3 [label="入力内容に問題があるか"];
		m [label="エラーメッセージ表示"];
		n [label="画像アップロード処理を実行"];
		o [label="画像アップロード完了画面を生成・表示"];
		p [label="二要素検証設定モジュールへ遷移"];
		q [label="?"];
		a->b->br1;
		br1->c [label="編集"];
		c->d->e->f->br2;
		br2->g [label="Yes"];
		br2->i [label="No"];
		i->br1;
		g->h->j->k;
		br1->l [label="画像アップロードを選択"];
		l->br3;
		br3->m [label="Yes"];
		br3->n [label="No"];
		n->o;
		br1->p [label="二要素認証設定を選択"];
		p->q;
	}%
	{利用者から受け取ったHTTPリクエスト}%
	{ウェブページ}%
	{}
\end{table}



\newpage
\subsubsection{検索画面の表示}
\begin{table}[H]
	\centering
	\caption{検索画面の表示}\label{tab:mod-search-display}
	\modtable{M00}{RootController.php}%
	{検索画面を表示するモジュール}%
	{
		node [shape=diamond] c;
		node [shape=box,style=rounded] a;
		a [label="/にアクセスされる"];
		b [label="検索画面を生成"];
		c [label="ログインしているか"];
		d [label="ログイン時のUIを生成"];
		e [label="未ログイン時のUIを生成"];
		a->b;
		b->c;
		c->d [label="Yes"];
		c->e [label="No"];
	}%
	{利用者から受け取ったHTTPリクエスト}%
	{ウェブページ}%
	{}
\end{table}
\subsection{利用者モジュール}
% 【記述内容】各ステップのロジックを説明になるんかな?。
\subsubsection{観光地・施設検索処理}

\begin{table}[H]
	\centering
	\caption{観光地・施設検索処理}\label{tab:mod-search}
	\modtable{M11}{RootController.php}%
	{キーワード・カテゴリに基づくスポット検索}%
	{
		node [shape=diamond] c;
		node [shape=box,style=rounded] a;
		a [label="検索開始", shape=];
		b [label="キーワード・カテゴリ受信"];
		c [label="入力値検証", shape=diamond];
		c1 [label="入力エラー表示"];
		d [label="DB検索実行", shape=diamond];
		d1 [label="結果件数>0"];
		d2 [label="「該当無し」表示"];
		e [label="結果表示"];
		f [label="終了"];
		a->b->c;
		c->c1 [label="NG"];
		c->d [label="OK"];
		d->d2 [label="No"];
		d->e [label=Yes]
		d2->f;
		e->f;
	}%
	{}%
	{検索結果リスト または メッセージ}%
	{入力値が空の場合は、仕様に合わせて実装}
\end{table}

\subsubsection{経路作成}

\subsubsection{AIによる観光地・施設推薦}
\begin{table}[H]
	\centering
	\caption{AI観光地推薦}\label{tab:mod-ai}
	\modtable{M13}{AiService.php}
	{ユーザーのチャット入力に基づく観光地推薦}%
	{
		node [shape=diamond] c;
		node [shape=box,style=rounded] a;
		start [label="開始", shape=oval];
		input [label="チャット受信"];
		checkempty [label="空文字チェック", shape=diamond];
		fetchctx [label="周辺情報取得"];
		prompt [label="プロンプト構築"];
		apicall [label="Gemini API送信"];
		checkapi [label="API応答成功?", shape=diamond];
		parse [label="JSON解析"];
		search [label="推薦スポット取得"];
		checkhit [label="スポットあり?", shape=diamond];
		resok [label="推薦回答"];
		resng [label="エラー回答", shape=oval];
		end [label="終了", shape=oval];
		start -> input -> checkempty;
		checkempty -> resng [label="空"];
		checkempty -> fetchctx [label="OK"];
		fetchctx -> prompt -> apicall -> checkapi;
		checkapi -> resng [label="失敗"];
		checkapi -> parse [label="成功"];
		parse -> search -> checkhit;
		checkhit -> resng [label="0件"];
		checkhit -> resok [label="1件以上"];
		resok -> end;
	}%
	{}%
	{}%
	{}
\end{table}

\subsubsection{チェックイン}
\begin{table}[H]
	\centering
	\caption{チェックイン判定}\label{tab:mod-checkin}
	\modtable{M14}{CheckinService.php}{QRコードと位置情報によるスタンプ付与}
	{
		node [shape=diamond] c;
		node [shape=box,style=rounded] a;
		start [label="開始", shape=oval];
		input [label="QRデータ・GPS受信"];
		decodeqr [label="QR検証(SpotID)", shape=diamond];
		getspot [label="Spot座標取得"];
		calcdist [label="距離計算"];
		checkdist [label="距離 <= 50m?", shape=diamond];
		savestamp [label="スタンプ保存"];
		checkcoup [label="クーポン条件?", shape=diamond];
		savecoup [label="クーポン付与"];
		msgsuccess [label="成功画面"];
		msgerror [label="エラー画面", shape=oval];
		end [label="終了", shape=oval];
		start -> input -> decodeqr;
		decodeqr -> msgerror [label="不正"];
		decodeqr -> getspot [label="正"];
		getspot -> calcdist -> checkdist;
		checkdist -> msgerror [label="範囲外"];
		checkdist -> savestamp [label="範囲内"];
		savestamp -> checkcoup;
		checkcoup -> msgsuccess [label="未達"];
		checkcoup -> savecoup [label="達成"];
		savecoup -> msgsuccess -> end;
	}%
	{}%
	{スタンプ取得結果, クーポン(条件達成時)}
	{}
\end{table}

\subsubsection{評価・写真投稿}
\begin{table}[H]
	\centering
	\caption{評価・写真投稿}\label{tab:mod-review-photo}
	\modtable{M15}{ReviewService.php}{評価と写真をシステムに登録する}%
	{
		node [shape=diamond] c;
		node [shape=box,style=rounded] a;
		start [label="開始", shape=oval];
		input [label="投稿データ受信"];
		validate [label="入力チェック", shape=diamond];
		hasimg [label="画像あり?", shape=diamond];
		upload [label="画像アップロード"];
		checkup [label="アップロード成功?", shape=diamond];
		convert [label="座標変換"];
		savedb [label="DB保存"];
		success [label="完了表示"];
		error [label="エラー表示", shape=oval];
		end [label="終了", shape=oval];
		start -> input -> validate;
		validate -> error [label="NG"];
		validate -> hasimg [label="OK"];
		hasimg -> savedb [label="No"];
		hasimg -> upload [label="Yes"];
		upload -> checkup;
		checkup -> error [label="失敗"];
		checkup -> convert [label="成功"];
		convert -> savedb;
		savedb -> success -> end;
	}%
	{}%
	{投稿完了メッセージ}%
	{}
\end{table}
\subsubsection{クーポン受け取り}
\subsubsection{クーポン受け取り処理}
\begin{table}[H]
	\centering
	\caption{クーポン受け取り処理}\label{tab:mod-coupon-receive}
	\modtable{M16}{CouponService.php}{スタンプ獲得状況に基づくクーポン付与}%
	{
		node [shape=diamond] c;
		node [shape=box,style=rounded] a;
		start [label="開始", shape=oval];
		input [label="獲得スタンプID受信"];
		fetchcond [label="関連クーポン検索(condspotid)"];
		checkhit [label="対象クーポンあり?", shape=diamond];
		checkduplicate [label="既に獲得済み?", shape=diamond];
		checkexpire [label="有効期限内?", shape=diamond];
		savedb [label="usercoupons保存(利用可能状態で作成)"];
		ressuccess [label="獲得成功通知"];
		resnone [label="獲得なし通知", shape=oval];
		end [label="終了", shape=oval];
		start -> input -> fetchcond -> checkhit;
		checkhit -> resnone [label="なし"];
		checkhit -> checkduplicate [label="あり"];
		checkduplicate -> resnone [label="Yes"];
		checkduplicate -> checkexpire [label="No"];
		checkexpire -> resnone [label="期限切"];
		checkexpire -> savedb [label="有効"];
		savedb -> ressuccess -> end;
	}%
	{}%
	{}%
	{スタンプ獲得トリガーで呼び出される想定。ユーザーが同じクーポンを重複して受け取らないようにする}
\end{table}


\subsection{事業者モジュール}
% 【記述内容】各ステップのロジックを説明になるんかな?。

\subsection{管理者モジュール}
% 【記述内容】各ステップのロジックを説明になるんかな?。
% ====================================
\newpage
\subsubsection{観光地・施設情報追加画面の表示}
\begin{table}[H]
	\centering
	\caption{観光地・施設情報追加画面の表示}\label{tab:mod-search}
	\modtable{M00}{RootController.php}%
	{観光地・施設情報追加画面を表示するモジュール}%
	{
		node [shape=diamond] c;
		node [shape=box,style=rounded] a;
		a [label="追加タブを選択"];
		b [label="情報追加画面を生成"];
		c [label="必要情報を入力"]
		d [label="入力値を検証"];
		e [label="SQLクエリを生成"];
		f [label="情報追加できない旨のUIを生成"];
		g [label="情報追加処理を実行"];
		h [label="情報追加完了画面を生成・表示"];
		a->b;
		b->c;
		c->d [label="Yes"];
		c->e [label="No"];
		d->f;
		f->g;
		g->h;
	}%
	{利用者から受け取ったHTTPリクエスト}%
	{ウェブページ}%
	{}
\end{table}

\newpage
\subsubsection{観光地・施設情報編集画面の表示}
\begin{table}[H]
	\centering
	\caption{観光地・施設情報編集画面の表示}\label{tab:mod-search}
	\modtable{M00}{RootController.php}%
	{観光地・施設情報編集画面を表示するモジュール}%
	{
		node [shape=diamond] c;
		node [shape=box,style=rounded] a;
		a [label="編集タブを選択"];
		b [label="情報編集画面を生成"];
		c [label="必要情報を入力"]
		d [label="入力値を検証"];
		e [label="SQLクエリを生成"];
		f [label="情報編集できない旨のUIを生成"];
		g [label="情報編集処理を実行"];
		h [label="情報編集完了画面を生成・表示"];
		a->b;
		b->c;
		c->d;
		d->e [label="Yes"];
		d->f [label="No"];
		f->g;
		g->h;
	}%
	{利用者から受け取ったHTTPリクエスト}%
	{ウェブページ}%
	{}
\end{table}

\newpage
\subsubsection{観光地・施設情報削除画面の表示}
\begin{table}[H]
	\centering
	\caption{観光地・施設情報削除画面の表示}\label{tab:mod-search}
	\modtable{M00}{RootController.php}%
	{観光地・施設情報削除画面を表示するモジュール}%
	{
		node [shape=diamond] c;
		node [shape=box,style=rounded] a;
		a [label="削除タブを選択"];
		b [label="情報削除画面を生成"];
		c [label="操作の確認画面の表示"];
		d [label="SQLクエリを生成"];
		e [label="元の画面を表示"];
		f [label="情報削除処理を実行"];
		g [label="情報削除完了画面を生成・表示"];
		a->b;
		b->c;
		c->d [label="Yes"];
		c->e [label="No"];
		d->f;
		f->g;
	}%
	{利用者から受け取ったHTTPリクエスト}%
	{ウェブページ}%
	{}
\end{table}

\newpage
\subsubsection{ユーザー情報追加画面の表示}
\begin{table}[H]
	\centering
	\caption{ユーザー情報追加画面の表示}\label{tab:mod-search}
	\modtable{M00}{RootController.php}%
	{ユーザー情報追加画面を表示するモジュール}%
	{
		node [shape=diamond] c;
		node [shape=box,style=rounded] a;
		a [label="追加タブを選択"];
		b [label="情報追加画面を生成"];
		c [label="必要情報を入力"]
		d [label="入力値を検証"];
		e [label="SQLクエリを生成"];
		f [label="情報追加できない旨のUIを生成"];
		g [label="情報追加処理を実行"];
		h [label="情報追加完了画面を生成・表示"];
		a->b;
		b->c;
		c->d [label="Yes"];
		c->e [label="No"];
		d->f;
		f->g;
		g->h;
	}%
	{利用者から受け取ったHTTPリクエスト}%
	{ウェブページ}%
	{}
\end{table}

\newpage
\subsubsection{ユーザー情報編集画面の表示}
\begin{table}[H]
	\centering
	\caption{ユーザー情報編集画面の表示}\label{tab:mod-search}
	\modtable{M00}{RootController.php}%
	{ユーザー情報編集画面を表示するモジュール}%
	{
		node [shape=diamond] c;
		node [shape=box,style=rounded] a;
		a [label="編集タブを選択"];
		b [label="情報編集画面を生成"];
		c [label="必要情報を入力"]
		d [label="入力値を検証"];
		e [label="SQLクエリを生成"];
		f [label="情報編集できない旨のUIを生成"];
		g [label="情報編集処理を実行"];
		h [label="情報編集完了画面を生成・表示"];
		a->b;
		b->c;
		c->d;
		d->e [label="Yes"];
		d->f [label="No"];
		f->g;
		g->h;
	}%
	{利用者から受け取ったHTTPリクエスト}%
	{ウェブページ}%
	{}
\end{table}

\newpage
\subsubsection{ユーザー情報削除画面の表示}
\begin{table}[H]
	\centering
	\caption{ユーザー情報削除画面の表示}\label{tab:mod-search}
	\modtable{M00}{RootController.php}%
	{ユーザー情報削除画面を表示するモジュール}%
	{
		node [shape=diamond] c;
		node [shape=box,style=rounded] a;
		a [label="削除タブを選択"];
		b [label="情報削除画面を生成"];
		c [label="操作の確認画面の表示"];
		d [label="SQLクエリを生成"];
		e [label="元の画面を表示"];
		f [label="情報削除処理を実行"];
		g [label="情報削除完了画面を生成・表示"];
		a->b;
		b->c;
		c->d [label="Yes"];
		c->e [label="No"];
		d->f;
		f->g;
	}%
	{利用者から受け取ったHTTPリクエスト}%
	{ウェブページ}%
	{}
\end{table}

\newpage
\subsubsection{UGC監視・管理画面の表示}
\begin{table}[H]
	\centering
	\caption{UGC監視・管理画面の表示}\label{tab:mod-search}
	\modtable{M00}{RootController.php}%
	{UGC監視・管理画面を表示するモジュール}%
	{
		node [shape=diamond] c;
		node [shape=box,style=rounded] a;
		a [label="投稿を選択"];
		b [label="投稿削除画面を生成"];
		c [label="操作の確認画面の表示"];
		d [label="SQLクエリを生成"];
		e [label="元の画面を表示"];
		f [label="情報削除処理を実行"];
		g [label="情報削除完了画面を生成・表示"];
		a->b;
		b->c;
		c->d [label="Yes"];
		c->e [label="No"];
		d->f;
		f->g;
	}%
	{利用者から受け取ったHTTPリクエスト}%
	{ウェブページ}%
	{}
\end{table}

\newpage
\subsubsection{サブスクリプション承認画面の表示}
\begin{table}[H]
	\centering
	\caption{サブスクリプション承認画面の表示}\label{tab:mod-search}
	\modtable{M00}{RootController.php}%
	{サブスクリプション承認画面を表示するモジュール}%
	{
		node [shape=diamond] c;
		node [shape=box,style=rounded] a;
		a [label="ユーザー管理画面を選択"];
		b [label="申請一覧画面を生成・選択"];
		c [label="承認ボタンを選択"]
		d [label="操作の確認画面の表示"];
		e [label="SQLクエリを生成"];
		f [label="元の画面を表示"];
		g [label="契約情報を更新"];
		h [label="情報削除完了画面を生成・表示"];
		a->b;
		b->c;
		c->d;
		d->e [label="Yes"];
		d->f [label="No"];
		d->f;
		f->g;
	}%
	{利用者から受け取ったHTTPリクエスト}%
	{ウェブページ}%
	{}
\end{table}

\newpage
\subsubsection{サブスクリプション解除画面の表示}
\begin{table}[H]
	\centering
	\caption{サブスクリプション解除画面の表示}\label{tab:mod-search}
	\modtable{M00}{RootController.php}%
	{サブスクリプション解除画面を表示するモジュール}%
	{
		node [shape=diamond] c;
		node [shape=box,style=rounded] a;
		a [label="ユーザー管理画面を選択"];
		b [label="契約一覧画面を生成・選択"];
		c [label="解除ボタンを選択"]
		d [label="操作の確認画面の表示"];
		e [label="SQLクエリを生成"];
		f [label="元の画面を表示"];
		g [label="契約情報を更新"];
		h [label="契約解除完了画面を生成・表示"];
		a->b;
		b->c;
		c->d;
		d->e [label="Yes"];
		d->f [label="No"];
		d->f;
		f->g;
	}%
	{利用者から受け取ったHTTPリクエスト}%
	{ウェブページ}%
	{}
\end{table}

\section{データベースアクセス設計}
% 【記述内容】データベースへのアクセス方法、外部設計書で指定された特殊なデータ形式やロジック(ID生成など)の詳細を定義

\subsection{ }
\begin{itemize}
	\item
\end{itemize}

\subsection{}
\begin{itemize}
	\item
\end{itemize}

% ====================================
\section{外部インターフェース設計}
% 【記述内容】外部サービスとの連携におけるリクエスト・レスポンスの形式、認証方法を定義

\subsection{「Gemini API」連携仕様}
\begin{itemize}
	\item
\end{itemize}

\subsection{「OpenStreetMap」連携仕様}
\begin{itemize}
	\item
\end{itemize}

\subsection{「Google マップ」連携仕様}

\subsection{「国土交通省交通量API」連携仕様}
\begin{itemize}
	\item
\end{itemize}


\end{document}
