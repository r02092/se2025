\documentclass{docs}

%--- 基本パッケージ ---%
\usepackage{tikz} % 図の作成

%--- TikZライブラリ ---%
\usetikzlibrary{arrows.meta} % 矢印
\usetikzlibrary{shapes.geometric} % 幾何学図形

%--- 文書情報 ---%
\title{外部設計書}

% アクティビティ図を描画するマクロ
% \adiagram{パーティション名}{縦の長さ}{流れ}{追加する描画命令}
% パーティション名: 上部に表示されるパーティション名を","で区切って記述
% 縦の長さ    : アクションごとに縦に進む長さ
% 本筋のフロー  : パーティション番号
%           0:アクション or 1:分岐
%           アクション名 or 分岐の名前
%           上記の3つを"/"で区切り、それを","で区切って記述
%           作成される各アクションのnodeは
%           "action<通し番号>"という名前で
%           (パーティション番号,-通し番号)の座標に配置される
%           本筋のアクションしか描画できないため、それ以外は
%           以下の引数に描画命令を渡して手動で描く必要がある
% 追加する描画命令: 追加で描画するTikZの形式の命令
%           条件分岐で別れた本筋以外のフローの描画などに使用
\newcommand\adiagram[4]{
	\begin{tikzpicture}[
		action/.style={draw,thick,rounded corners,font=\sffamily},
		decision/.style={draw,thick,diamond,inner sep=8pt},
		arrow/.style={-Stealth},
		label/.style={fill=white,inner sep=0pt,font=\sffamily},
		xscale=5
	]
		\foreach\t[count=\i from 0]in{#1}{
			\node[font=\bfseries\sffamily]at(\i,1.5){\t};
			\xdef\nps{\i}
		}
		\begin{scope}[yscale=#2]
			\node[fill,inner sep=8pt,shape=circle](action0)at(0,0){};
			\xdef\ot{}
			\xdef\ox{0}
			\xdef\oi{0}
			\foreach\x/\e/\t[count=\i from 1]in{#3}{
				\ifnum\e=0
					\node[action](action\i)at(\x,-\i){\t};
				\else
					\node[decision](action\i)at(\x,.25-\i){};
				\fi
				\draw[arrow](action\oi)
				\if"\ot"
					\ifnum\x=\ox--\else-|\fi(action\i);
				\else
					\ifnum\x=\ox--node[label]{\ot}\else-|node[label]{\ot}\fi(action\i);
					\xdef\ot{}
				\fi
				\ifnum\e=1\xdef\ot{\t}\fi
				\xdef\ox{\x}
				\xdef\oi{\i}
			}
			\node[draw,double distance=2pt,thick,fill,inner sep=8pt,shape=circle](end)at(\ox,-1-\oi){};
			\draw[arrow](action\oi)--(end);
		\end{scope}
		\foreach\i in{-.5,...,\nps.5}
			\draw[ultra thick](\i,1)--(\i,-2.5-\oi*1.5);
		\begin{scope}[yscale=#2]
			#4
		\end{scope}
	\end{tikzpicture}
}

\begin{document}
\section{はじめに}%%%%%%%%%%%%%%%%%%%%%%%%%%%%%%%%%%%%%%%%%%
本書では、我々がシステム提案書にて提案した「SceneTrip」の
詳細について記述します。まず、本システムの業務の流れを説明し、
次に機能設計、ユーザインタフェース設計、データベース設計、
ネットワーク設計について順に説明します。

\section{利用の流れ}
本システムは、高知県を訪れる観光客、
特にアニメなどの聖地巡礼を目的とした旅行者を対象として、
観光地や施設の検索、最適なルートの作成などの機能を提供するサービスです。
また、地域事業者向けには、観光データを分析できるAPIの提供や、
システム上での広告掲載機能の提供が可能です。
以下に、本システムの利用の流れを示します。
\begin{figure}[H]
  \centering
  \includegraphics[bb=0 0 612 792,clip,scale=.5]{gaibu-flow.pdf}
  \caption{利用の流れ}\label{fig:flow}
\end{figure}

\section{機能概要}
本システムは、以下の主要な機能を提供します。
\subsection{共通機能}
\subsubsection{利用規約}
システムを初めて利用する際に、ユーザーは利用規約に同意する必要があります。
利用規約に同意した後、ユーザーはシステムの各種機能を利用できます。

\subsubsection{ログイン・ログアウト機能}
ログイン・ログアウト機能は、本システムのすべてのユーザーが
ログインおよびログアウトを行うための機能です。
ユーザーは、ユーザ名とパスワードを用いてログインします。
ログインに成功すると、ユーザーはシステムの各種機能を利用できます。
ログアウトを行うと、ユーザーはシステムから安全に退出します。

\subsection{管理者}
\subsubsection{観光地・施設情報管理}
管理者は、システム上で観光地や施設の情報を管理できます。
具体的には、観光地や施設の追加、編集、削除などの操作が可能です。
これにより、システム内の観光地や施設の情報を最新の状態に保つことができます。

\subsubsection{ユーザー管理}
管理者は、システムのユーザー情報を管理できます。
具体的には、ユーザーの登録、編集、削除などの操作が可能です。
これにより、システムのユーザー情報を適切に管理できます。

\subsubsection{システム管理}
管理者は、システム全体の設定や運用を管理できます。
具体的には、システムのパフォーマンス監視、バックアップ、
セキュリティ設定などの操作が可能です。
これにより、システムの安定した運用を維持できます。

\subsection{観光客}
\subsubsection{観光地検索}
観光客は、システム上で観光地や施設を検索できます。
検索は、キーワード検索やカテゴリ検索などの方法で行うことができます。
検索結果には、観光地や施設の詳細情報が表示されます。

\subsubsection{経路作成}
観光客は、訪れたい観光地や施設を選択し、
最適な観光ルートを作成できます。
システムは、選択された観光地や施設を考慮して、
効率的なルートを提案します。

\subsubsection{外部地図アプリ連携機能}
観光客は、作成した観光ルートを外部の地図アプリに連携できます。
これにより、ナビゲーションやルート案内を外部アプリで利用できます。

\subsubsection{評価機能}
観光客は、訪れた観光地や施設に対して評価を行うことができます。
評価は、星評価やコメントなどの形式で行うことができます。
評価情報は、他の観光客が観光地や施設を選択する際の参考になります。

\subsubsection{写真投稿・共有機能}
観光客は、訪れた観光地や施設の写真をシステムに投稿・共有できます。
投稿された写真は、他の観光客が閲覧できるようになります。

\subsubsection{スタンプ機能}
観光客は、訪れた観光地や施設にスタンプをつけることができます。

\subsubsection{クーポンの受け取り機能}
観光客は、システム上で提供されるクーポンを受け取ることができます。
クーポンは、観光地や施設での割引や特典に利用できます。

\subsection{事業者}
\subsubsection{広告掲載機能}
地域事業者は、システム上で広告を掲載できます。
広告は、観光客がシステムを利用する際に表示されます。
広告掲載により、地域事業者は自社のサービスや商品を
観光客に効果的にアピールできます。

\subsubsection{観光データ分析API}
地域事業者は、システムが提供する観光データ分析APIを利用できます。
APIを通じて、観光客の行動データや評価データなどを取得し、
マーケティングやサービス改善に活用できます。

\subsubsection{クーポン発行機能}
地域事業者は、システム上でクーポンを発行できます。
発行されたクーポンは、観光客がシステムを通じて受け取ることができます。






\section{機能設計}%%%%%%%%%%%%%%%%%%%%%%%%%%%%%%%%%%%%%%%%%%

\subsection{サブシステム設計}

\begin{figure}[H]
	\centering
	\adiagram{利用者,システム,データベース}{1.5}{
		0/0/検索クエリを入力,
		0/0/検索ボタンをクリック,
		1/0/SQLクエリを生成,
		2/0/検索を実行,
		2/0/結果を返す,
		1/1/観光地・施設が存在する,
		1/0/結果を表示,
		0/0/結果を確認
	}{
		\node[action](decision0)at(1,-8){存在しない旨を表示};
		\draw[arrow](action6)-|(1.45,-8)node[label,pos=.57]{存在しない}|-(decision0);
		\draw[arrow](decision0)--(action8);
	}
	\caption{観光地・施設検索}\label{fig:act-spot_search}
\end{figure}

\section{ユーザインタフェース設計}

\section{データベース設計}
\begin{figure}[H]
	\centering
	\includegraphics[bb=16.652460 183.545994 824.542569 587.681982,clip,scale=.55]{gaibu-er.pdf}
	\caption{ER図}\label{fig:er}
\end{figure}

\section{ネットワーク設計}
\begin{figure}[H]
	\centering
	\includegraphics[bb=32.951601 147.563995 568.673983 794.321976,clip,scale=.8]{gaibu-network.pdf}
	\caption{ネットワーク構成図}\label{fig:nw}
\end{figure}

\end{document}
