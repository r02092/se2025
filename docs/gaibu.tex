\documentclass{docs}

%--- 基本パッケージ ---%
\usepackage{tikz} % 図の作成

%--- TikZライブラリ ---%
\usetikzlibrary{arrows.meta} % 矢印
\usetikzlibrary{shapes.geometric} % 幾何学図形

%--- 文書情報 ---%
\title{外部設計書}

% アクティビティ図を描画するマクロ
% \adiagram{パーティション名}{縦の長さ}{流れ}{追加する描画命令}
% パーティション名: 上部に表示されるパーティション名を","で区切って記述
% 縦の長さ    : アクションごとに縦に進む長さ
% 本筋のフロー  : パーティション番号
%           0:アクション or 1:分岐
%           アクション名 or 分岐の名前
%           上記の3つを"/"で区切り、それを","で区切って記述
%           作成される各アクションのnodeは
%           "action<通し番号>"という名前で
%           (パーティション番号,-通し番号)の座標に配置される
%           本筋のアクションしか描画できないため、それ以外は
%           以下の引数に描画命令を渡して手動で描く必要がある
% 追加する描画命令: 追加で描画するTikZの形式の命令
%           条件分岐で別れた本筋以外のフローの描画などに使用
\newcommand\adiagram[4]{
	\begin{tikzpicture}[
		action/.style={draw,thick,rounded corners,font=\sffamily},
		decision/.style={draw,thick,diamond,inner sep=8pt},
		arrow/.style={-Stealth},
		label/.style={fill=white,inner sep=0pt,font=\sffamily},
		xscale=5
	]
		\foreach\t[count=\i from 0]in{#1}{
			\node[font=\bfseries\sffamily]at(\i,1.5){\t};
			\xdef\nps{\i}
		}
		\begin{scope}[yscale=#2]
			\node[fill,inner sep=8pt,shape=circle](action0)at(0,0){};
			\xdef\ot{}
			\xdef\ox{0}
			\xdef\oi{0}
			\foreach\x/\e/\t[count=\i from 1]in{#3}{
				\ifnum\e=0
					\node[action](action\i)at(\x,-\i){\t};
				\else
					\node[decision](action\i)at(\x,.25-\i){};
				\fi
				\draw[arrow](action\oi)
				\if"\ot"
					\ifnum\x=\ox--\else-|\fi(action\i);
				\else
					\ifnum\x=\ox--node[label]{\ot}\else-|node[label]{\ot}\fi(action\i);
					\xdef\ot{}
				\fi
				\ifnum\e=1\xdef\ot{\t}\fi
				\xdef\ox{\x}
				\xdef\oi{\i}
			}
			\node[draw,double distance=2pt,thick,fill,inner sep=8pt,shape=circle](end)at(\ox,-1-\oi){};
			\draw[arrow](action\oi)--(end);
		\end{scope}
		\foreach\i in{-.5,...,\nps.5}
			\draw[ultra thick](\i,1)--(\i,-2.5-\oi*1.5);
		\begin{scope}[yscale=#2]
			#4
		\end{scope}
	\end{tikzpicture}
}

\begin{document}
\section{はじめに}%%%%%%%%%%%%%%%%%%%%%%%%%%%%%%%%%%%%%%%%%%
本書では、我々がシステム提案書にて提案した「SceneTrip」の
詳細について記述します。まず、本システムの業務の流れを説明し、
次に機能設計、ユーザインタフェース設計、データベース設計、
ネットワーク設計について順に説明します。

\section{利用の流れ}
本システムは、高知県を訪れる観光客、
特にアニメなどの聖地巡礼を目的とした旅行者を対象として、
観光地や施設の検索、最適なルートの作成などの機能を提供するサービスです。
また、地域事業者向けには、観光データを分析できるAPIの提供や、
システム上での広告掲載機能の提供が可能です。
以下に、本システムの利用の流れを示します。
\begin{figure}[H]
  \centering
  \includegraphics[bb=43.721999 420.785987 549.485983 778.409976,clip,scale=.8]{gaibu-flow.pdf}
  \caption{利用の流れ}\label{fig:flow}
\end{figure}

\section{機能設計}%%%%%%%%%%%%%%%%%%%%%%%%%%%%%%%%%%%%%%%%%%
\subsection{共通機能}
\subsubsection{利用規約}
システムを初めて利用する際に、ユーザーは利用規約に同意する必要があります。
利用規約に同意した後、ユーザーはシステムの各種機能を利用できます。

\subsubsection{ログイン・ログアウト機能}
ログイン・ログアウト機能は、本システムを利用するユーザーが
ログインおよびログアウトを行うための機能です。
ユーザーは、ユーザー名とパスワードを設定してアカウント登録をすることでログインができます。
また、パスワードは現在設定しているパスワードを再度入力することを条件に再度設定できます。
ゲストとして利用できる機能は、観光地・施設への経路検索のみになります。
ログインに成功すると、ユーザーはシステムの各種機能を利用できます。
ログアウトを行うと、ユーザーはシステムから安全に退出します。

\subsection{管理者}
\subsubsection{観光地・施設情報管理}
管理者は、システム上で観光地や施設の情報を管理できます。
具体的には、観光地や施設の追加、編集、削除などの操作が可能です。
これにより、システム内の観光地や施設の情報を最新の状態に保つことができます。

\subsubsection{ユーザー管理}
管理者は、システムのユーザー情報を管理できます。
具体的には、ユーザー情報の登録、編集、削除などの操作が可能です。
これにより、システムのユーザー情報を適切に管理できます。

\subsubsection{UGC監視・管理機能}
管理者は、UGC監視・管理できます。
具体的には、利用者が投稿した口コミやフォトスポット投稿を監視し、
利用規約違反が確認されたアカウントの利用停止や削除を行うことが可能です。
これにより、UGCを適切に管理できます。

\subsubsection{システム管理}
管理者は、システム全体の設定や運用を管理できます。
具体的には、システムのパフォーマンス監視、バックアップ、
セキュリティ設定などの操作が可能です。
これにより、システムの安定した運用を維持できます。

\subsubsection{データ分析・閲覧機能}
管理者はデータの閲覧・分析を行うことができます。
具体的には、行政や事業者に提供する観光データや、
ダッシュボードの表示内容を管理者として確認・監視
することが可能です。
これにより、正しいデータを事業者や行政に提供できます。

\subsubsection{問い合わせ対応機能}
管理者はユーザーからの問い合わせに対応できます。
具体的には、利用者や事業者からの問い合わせを一元管理し、
対応を記録することが可能です。

\subsection{利用者(ゲストログイン)}
\subsubsection{観光地検索}
利用者は、システム上で観光地や施設を検索できます。
検索は、キーワード検索やカテゴリ検索などの方法で行うことができます。
検索結果には、観光地や施設の詳細情報が表示されます。

\subsection{利用者(ログイン必須機能)}
\subsection{利用者アカウント作成機能}
利用者は、システムを利用するためのアカウントを作成できます。
作成の際には、ユーザー名、パスワード等の情報を登録します。

\subsubsection{経路作成}
利用者は、訪れたい観光地や施設を選択し、
最適な観光ルートを作成できます。
システムは、選択された観光地や施設を考慮して、
効率的なルートを提案します。

\subsubsection{外部地図アプリ連携機能}
利用者は、作成した観光ルートをGoogle Mapsに連携させて表示することができます。
これにより、ルート作成後のナビゲーションやルート案内もGoogle Mapsで利用できます。

\subsubsection{評価機能}
利用者は、訪れた観光地や施設に対して評価を行うことができます。
評価は、星評価やコメントなどの形式で行うことができます。
評価情報は、他の観光客が観光地や施設を選択する際の参考になります。

\subsubsection{写真投稿・共有機能}
利用者は、訪れた観光地や施設の写真をシステムに投稿・共有できます。
投稿された写真は、他の利用者が閲覧できるようになります。

\subsubsection{スタンプ機能}
利用者は、訪れた観光地や施設のスタンプをつけることができます。

\subsubsection{クーポンの受け取り機能}
利用者は、システム上で提供されるクーポンを受け取ることができます。
クーポンは、観光地や施設での割引や特典に利用できます。

\subsection{事業者}

\subsection{事業者アカウント作成機能}
店舗や施設などの事業者が、システムを利用するための
アカウントを作成できます。
作成の際には、店舗名、業種、所在地、メールアドレス、
パスワード等の情報を登録します。

%\subsubsection{広告掲載機能}
%地域事業者は、システム上で広告を掲載できます。
%広告は、観光客がシステムを利用する際に表示されます。
%広告掲載により、地域事業者は自社のサービスや商品を
%観光客に効果的にアピールできます。

\subsubsection{事業者情報変更機能}
登録した店舗情報(店舗名、所在地、営業時間、紹介文、店舗写真など)
を変更する機能。

\subsubsection{観光データ分析API}
地域事業者は、システムが提供する観光データ分析APIを利用できます。
APIを通じて、観光客の行動データや評価データなどを取得し、
マーケティングやサービス改善に活用できます。

\subsubsection{クーポン発行機能}
地域事業者は、システム上でクーポンを発行できます。
発行されたクーポンは、観光客がシステムを通じて受け取ることができます。



\subsection{アカウント}

\subsection{場所提供者}

\subsection{出店者}

\subsection{利用者}

\subsection{管理者}

\subsection{サブシステム設計}

\begin{figure}[H]
	\centering
	\adiagram{利用者,システム,データベース}{1.5}{
		0/0/検索クエリを入力,
		0/0/検索ボタンをクリック,
		1/0/SQLクエリを生成,
		2/0/検索を実行,
		2/0/結果を返す,
		1/1/観光地・施設が存在する,
		1/0/結果を表示,
		0/0/結果を確認
	}{
		\node[action](decision0)at(1,-8){存在しない旨を表示};
		\draw[arrow](action6)-|(1.45,-8)node[label,pos=.57]{存在しない}|-(decision0);
		\draw[arrow](decision0)--(action8);
	}
	\caption{観光地・施設検索}\label{fig:act-spot_search}
\end{figure}

\section{ユーザインタフェース設計}
\newpage
\begin{figure}[H]
	\centering
	\begin{tikzpicture}[scale=.0375]
		\draw(-210,297)--(210,297)--(210,-297)--(-210,-297)--cycle;
		\node[inner sep=0pt,outer sep=0pt]{\includegraphics[scale=.75]{invoice/invoice.pdf}};
	\end{tikzpicture}
	\caption{出力される請求書の様式}\label{fig:invoice}
\end{figure}
\section{データベース設計}
\subsection{ER図}
IE記法によるデータベースのER図を\cref{fig:er}に示します。
\begin{figure}[H]
	\centering
	\includegraphics[bb=16.65246 40.229999 824.542569 584.729982,clip,scale=.55]{gaibu-er.pdf}
	\caption{ER図}\label{fig:er}
\end{figure}
\subsection{テーブルとカラムの詳細}
各テーブルおよびカラムの詳細を示します。
IDについて特筆なきものは、AUTO INCREMENT属性を
指定することで自動で採番を行うものとします。
経緯度については、以下に示す式を用いて度単位の値を変換し、
得られた値を格納します。
\[
	\text{[データベースに格納する値]}=\frac{\text{[元の値]}-\text{[最小値]}}{\text{[最大値]}-\text{[最小値]}}(2^{32}-1)
\]

経路検索のために、システムは
鉄道(軌道を含む)や国道(高速道路を含む)および
それらを接続する主な道路を模したグラフをデータベースに
保持します。nodesテーブル(\cref{tb:db-nodes})は
グラフのノードを、edgesテーブル(\cref{tb:db-edges})は
グラフのエッジをそれぞれ保存するテーブルとなっています。
\begin{table}[H]
	\centering
	\caption{nodesテーブル(グラフのノード)}\label{tb:db-nodes}
	\begin{tabular}{|l|l|l|l|}
		\hline
		\thead{カラム名} & \thead{データ型} & \thead{制約} & \thead{説明} \\ \hline
		id & INT UNSIGNED & PK & ID \\ \hline
		type & TINYINT UNSIGNED & UK, NOT NULL & 種別 \\ \hline
		code & INT UNSIGNED & UK, NOT NULL & コード \\ \hline
		name & VARCHAR(255) & NOT NULL & 地点名 \\ \hline
		lng & INT UNSIGNED & NOT NULL & 経度 \\ \hline
		lat & INT UNSIGNED & NOT NULL & 緯度 \\ \hline
	\end{tabular}
\end{table}
nodesテーブルのIDは、typeカラムおよびcodeカラムの
値からSHA3-224で生成したハッシュ値の先頭4バイトを使用して
採番します。また、typeカラムの値の意味とtypeカラムの
値ごとのcodeカラムの意味を\cref{tb:db-nodes-type}に示します。
\begin{table}[H]
	\centering
	\caption{nodesテーブルのtypeカラムおよびcodeカラムについて}\label{tb:db-nodes-type}
	\begin{tabular}{|r|l|l|}
		\hline
		\thead{type} & \thead{意味} & \thead{codeの値の意味} \\ \hline
		0 & 交差点 &
		\begin{tabular}{l}
			$\text{[JIS X 0402:2020で定められた市区町村コード]}\times 1000$\\
			${}+\text{[独自に定めた市区町村内で一意の値]}$\\
			(市区町村コードはJIS X 0401:1973で定められた\\
			\quad 都道府県コードと併用するものとする)
		\end{tabular}
		\\ \hline
		1 & 鉄道(軌道を含む)の駅 &
		\begin{tabular}{l}
			「緯度経度付き全国沿線・駅データベース」で\\
			定められた駅コード
		\end{tabular}
		\\ \hline
		2 & 高速道路のIC &
		\begin{tabular}{l}
			$\text{[各路線に対し独自に定めた一意の値]}\times 10000$\\
			${}+\text{[IC番号]}\times 10+\text{[IC番号の枝番]}$
		\end{tabular}
		\\ \hline
		3 & 常時観測点 &
		\begin{tabular}{l}
			「国土交通省交通量API」で提供される\\
			常時観測点コード
		\end{tabular}
		\\ \hline
	\end{tabular}
\end{table}
\begin{table}[H]
	\centering
	\caption{edgesテーブル(グラフのエッジ)}\label{tb:db-edges}
	\begin{tabular}{|l|l|l|l|}
		\hline
		\thead{カラム名} & \thead{データ型} & \thead{制約} & \thead{説明} \\ \hline
		id & INT UNSIGNED & PK & ID \\ \hline
		node1\_id & INT UNSIGNED & FK, UK, NOT NULL & 1つ目のノード \\ \hline
		node2\_id & INT UNSIGNED & UK, NOT NULL & 2つ目のノード \\ \hline
		time & INT UNSIGNED & NOT NULL & ノード間の所要時間 \\ \hline
	\end{tabular}
\end{table}
このシステムでは、管理者や事業者を含む全ての利用者を
共通のusersテーブル(\cref{tb:db-users})で
扱い、\cref{tb:db-users-type}に示すようにtypeカラムの値により
それらを区別します。
事業者の場合、住所の登録を要するものとします。
住所は郵便番号およびJIS X 0402:2020で
定められた市区町村コード(JIS X 0401:1973で定められた
都道府県コードと併用するもの)と市区町村に続く住所の文字列の
組合せにより保存します。
また、ログイン時にRFC 6238に準拠したTOTPによる二要素認証を
行えるようにするためのカラムを設けています。
総当たり攻撃を防ぐため、直近の認証時の時刻と
認証回数を保持するカラムも設けており、
高頻度な認証操作を制限できます。OTPの更新頻度は30秒と
するため、直近の認証時の時刻は、UNIX時刻を30で
割った値として保存し、認証回数も30秒ごとに数え直すものと
します。
\begin{table}[H]
	\centering
	\caption{usersテーブル(管理者や事業者を含む利用者)}\label{tb:db-users}
	\begin{tabular}{|l|l|l|l|}
		\hline
		\thead{カラム名} & \thead{データ型} & \thead{制約} & \thead{説明} \\ \hline
		id & INT UNSIGNED & PK & ID \\ \hline
		login\_name & VARCHAR(255) & UK, NOT NULL & ログイン名 \\ \hline
		password & VARCHAR(255) & NOT NULL & ハッシュ化されたパスワード \\ \hline
		type & TINYINT UNSIGNED & NOT NULL & 種別 \\ \hline
		name & VARCHAR(255) & NOT NULL & 名前 \\ \hline
		points & INT UNSIGNED & NOT NULL & ポイント数 \\ \hline
		postal\_code & INT UNSIGNED & & 郵便番号 \\ \hline
		addr\_city & INT UNSIGNED & & 市区町村コード \\ \hline
		addr\_detail & VARCHAR(255) & & 市区町村名の後に続く住所 \\ \hline
		totp\_secret & BINARY(20) & & TOTPのシークレット \\ \hline
		totp\_iv & BINARY(12) & & TOTPの初期ベクトル \\ \hline
		totp\_tag & BINARY(16) & & TOTPの認証タグ \\ \hline
		totp\_last\_time & INT UNSIGNED & & TOTPの直近の認証時の時刻 \\ \hline
		totp\_counter & TINYINT UNSIGNED & & TOTPの30秒ごとの認証回数 \\ \hline
		created\_at & DATETIME & & 作成時刻 \\ \hline
		updated\_at & DATETIME & & 更新時刻 \\ \hline
		deleted\_at & DATETIME & & 削除時刻 \\ \hline
	\end{tabular}
\end{table}
\begin{table}[H]
	\centering
	\caption{usersテーブルのtypeカラムについて}\label{tb:db-users-type}
	\begin{tabular}{|r|l|}
		\hline
		\thead{type} & \thead{意味} \\ \hline
		0 & 管理者 \\ \hline
		1 & 一般の利用者 \\ \hline
		2 & スタンダードプランの事業者 \\ \hline
		3 & プレミアムプランの事業者 \\ \hline
	\end{tabular}
\end{table}
パスワードのハッシュ化にはPHPの\verb|password_hash()|関数を
使用し、暗号学的ハッシュ関数であるArgon2idを用い、かつ
ソルトを使用することで、安全にパスワードを保存します。

観光地や観光施設、店などのスポットを
保存するspotsテーブル(\cref{tb:db-spots})は、
近隣のグラフのノードに関連付けることで、
経路検索にスポットを使用できるようにします。
また、keywordsテーブル(\cref{tb:db-keywords})
スポットにはキーワードを関連付けることができます。
このキーワードはAI検索時のLLMのシステムプロンプトの構成などに
使用されます。
\begin{table}[H]
	\centering
	\caption{spotsテーブル(観光地や観光施設、店などのスポット)}\label{tb:db-spots}
	\begin{tabular}{|l|l|l|l|}
		\hline
		\thead{カラム名} & \thead{データ型} & \thead{制約} & \thead{説明} \\ \hline
		id & INT UNSIGNED & PK & ID \\ \hline
		node\_id & INT UNSIGNED & FK, NOT NULL & 関連付けたノード \\ \hline
		user\_id & INT UNSIGNED & FK, NOT NULL & 地点の登録者 \\ \hline
		type & TINYINT UNSIGNED & NOT NULL & 種別 \\ \hline
		name & VARCHAR(255) & NOT NULL & 名前 \\ \hline
		lng & INT UNSIGNED & NOT NULL & 経度 \\ \hline
		lat & INT UNSIGNED & NOT NULL & 緯度 \\ \hline
		created\_at & DATETIME & & 作成時刻 \\ \hline
		updated\_at & DATETIME & & 更新時刻 \\ \hline
		deleted\_at & DATETIME & & 削除時刻 \\ \hline
	\end{tabular}
\end{table}
\begin{table}[H]
	\centering
	\caption{keywordsテーブル(スポットに関連付けるキーワード)}\label{tb:db-keywords}
	\begin{tabular}{|l|l|l|l|}
		\hline
		\thead{カラム名} & \thead{データ型} & \thead{制約} & \thead{説明} \\ \hline
		id & INT UNSIGNED & PK & ID \\ \hline
		spot\_id & INT UNSIGNED & FK, UK, NOT NULL & 関連付けたスポット \\ \hline
		keyword & VARCHAR(255) & UK, NOT NULL & キーワード \\ \hline
		created\_at & DATETIME & & 作成時刻 \\ \hline
		updated\_at & DATETIME & & 更新時刻 \\ \hline
		deleted\_at & DATETIME & & 削除時刻 \\ \hline
	\end{tabular}
\end{table}
スポットに対する口コミは、reviewsテーブル(\cref{tb:db-reviews})に
保存します。段階的な評価および文章を入力できます。
また、閲覧数も記録され、事業者は閲覧されている口コミを
確認できます。
\begin{table}[H]
	\centering
	\caption{reviewsテーブル(スポットに対する口コミ)}\label{tb:db-reviews}
	\begin{tabular}{|l|l|l|l|}
		\hline
		\thead{カラム名} & \thead{データ型} & \thead{制約} & \thead{説明} \\ \hline
		id & INT UNSIGNED & PK & ID \\ \hline
		spot\_id & INT UNSIGNED & FK, NOT NULL & 関連付けたスポット \\ \hline
		user\_id & INT UNSIGNED & FK, NOT NULL & 投稿者 \\ \hline
		rate & TINYINT UNSIGNED & NOT NULL & 評価 \\ \hline
		comment & TEXT & NOT NULL & 本文 \\ \hline
		views & INT UNSIGNED & NOT NULL & 閲覧数 \\ \hline
		ip\_addr & VARBINARY(16) & NOT NULL & IPアドレス \\ \hline
		port & SMALLINT UNSIGNED & NOT NULL & ポート番号 \\ \hline
		user\_agent & VARCHAR(255) & NOT NULL & User-Agent \\ \hline
		created\_at & DATETIME & & 作成時刻 \\ \hline
		updated\_at & DATETIME & & 更新時刻 \\ \hline
		deleted\_at & DATETIME & & 削除時刻 \\ \hline
	\end{tabular}
\end{table}
事業者が発行したクーポンはcouponsテーブル(\cref{tb:db-coupons})に
保存します。クーポンを手に入れるために行く必要のあるスポットも
記録します。
また、利用者が手に入れたクーポンはuser\_couponsテーブル(\cref{tb:db-user_coupons})に
保存します。クーポンの入手時に確認用キーが生成され、
これを含む二次元コードを事業者に提示することで
正規のクーポンであることを証明します。
\begin{table}[H]
	\centering
	\caption{couponsテーブル(スポットに対して作成されたクーポン)}\label{tb:db-coupons}
	\begin{tabular}{|l|l|l|l|}
		\hline
		\thead{カラム名} & \thead{データ型} & \thead{制約} & \thead{説明} \\ \hline
		id & INT UNSIGNED & PK & ID \\ \hline
		spot\_id & INT UNSIGNED & FK, NOT NULL & 使用できるスポット \\ \hline
		name & VARCHAR(255) & NOT NULL & 名前 \\ \hline
		cond\_spot\_id & INT UNSIGNED & NOT NULL & 行く必要のあるスポット \\ \hline
		created\_at & DATETIME & & 作成時刻 \\ \hline
		updated\_at & DATETIME & & 更新時刻 \\ \hline
		deleted\_at & DATETIME & & 削除時刻 \\ \hline
	\end{tabular}
\end{table}
\begin{table}[H]
	\centering
	\caption{user\_couponsテーブル(利用者が手に入れたクーポン)}\label{tb:db-user_coupons}
	\begin{tabular}{|l|l|l|l|}
		\hline
		\thead{カラム名} & \thead{データ型} & \thead{制約} & \thead{説明} \\ \hline
		id & INT UNSIGNED & PK & ID \\ \hline
		coupon\_id & INT UNSIGNED & FK, UK1, NOT NULL & 関連付けたクーポン \\ \hline
		user\_id & INT UNSIGNED & FK, UK1, NOT NULL & 手に入れた利用者 \\ \hline
		key & BIGINT UNSIGNED & UK2, NOT NULL & 確認用キー \\ \hline
		is\_used & TINYINT UNSIGNED & NOT NULL & 使用されたか \\ \hline
		created\_at & DATETIME & & 作成時刻 \\ \hline
		updated\_at & DATETIME & & 更新時刻 \\ \hline
	\end{tabular}
\end{table}
利用者が手に入れたスタンプはstampsテーブル(\cref{tb:db-stamps})に
保存します。スタンプの画像はファイルシステムに保存するため、
それに対応するテーブルやカラムは存在しません。
\begin{table}[H]
	\centering
	\caption{stampsテーブル(利用者が手に入れたスタンプ)}\label{tb:db-stamps}
	\begin{tabular}{|l|l|l|l|}
		\hline
		\thead{カラム名} & \thead{データ型} & \thead{制約} & \thead{説明} \\ \hline
		id & INT UNSIGNED & PK & ID \\ \hline
		spot\_id & INT UNSIGNED & FK, UK, NOT NULL & 関連付けたスポット \\ \hline
		user\_id & INT UNSIGNED & FK, UK, NOT NULL & 手に入れた利用者 \\ \hline
		ip\_addr & VARBINARY(16) & NOT NULL & IPアドレス \\ \hline
		port & SMALLINT UNSIGNED & NOT NULL & ポート番号 \\ \hline
		user\_agent & VARCHAR(255) & NOT NULL & User-Agent \\ \hline
		created\_at & DATETIME & & 作成時刻 \\ \hline
		updated\_at & DATETIME & & 更新時刻 \\ \hline
	\end{tabular}
\end{table}
利用者の検索クエリは、事業者の分析に
使用するため、queriesテーブル(\cref{tb:db-queries})に
保存されます。経路検索を伴う場合は、その出発点および目的地も
保存されます。
\begin{table}[H]
	\centering
	\caption{queriesテーブル(利用者の検索クエリ)}\label{tb:db-queries}
	\begin{tabular}{|l|l|l|l|}
		\hline
		\thead{カラム名} & \thead{データ型} & \thead{制約} & \thead{説明} \\ \hline
		id & INT UNSIGNED & PK & ID \\ \hline
		user\_id & INT UNSIGNED & FK, NOT NULL & 検索した利用者 \\ \hline
		query & TEXT & NOT NULL & 検索クエリ \\ \hline
		from\_spot\_id & INT UNSIGNED & & 出発地 \\ \hline
		to\_spot\_id & INT UNSIGNED & & 目的地 \\ \hline
		ip\_addr & VARBINARY(16) & NOT NULL & IPアドレス \\ \hline
		port & SMALLINT UNSIGNED & NOT NULL & ポート番号 \\ \hline
		user\_agent & VARCHAR(255) & NOT NULL & User-Agent \\ \hline
		created\_at & DATETIME & & 作成時刻 \\ \hline
		updated\_at & DATETIME & & 更新時刻 \\ \hline
	\end{tabular}
\end{table}
利用者の投稿した写真は、photosテーブル(\cref{tb:db-photos})に
保存されます。地図上に表示するため、写真の経緯度を保存します。
また、画像の拡張子も保存します。拡張子を除いた
ファイル名はIDから生成可能なものとします。
\begin{table}[H]
	\centering
	\caption{photosテーブル(利用者の投稿した写真)}\label{tb:db-photos}
	\begin{tabular}{|l|l|l|l|}
		\hline
		\thead{カラム名} & \thead{データ型} & \thead{制約} & \thead{説明} \\ \hline
		id & INT UNSIGNED & PK & ID \\ \hline
		user\_id & INT UNSIGNED & FK, NOT NULL & 投稿者 \\ \hline
		lng & INT UNSIGNED & NOT NULL & 経度 \\ \hline
		lat & INT UNSIGNED & NOT NULL & 緯度 \\ \hline
		img\_ext & VARCHAR(4) & NOT NULL & 画像の拡張子 \\ \hline
		comment & TEXT & NOT NULL & 写真に付けた文 \\ \hline
		ip\_addr & VARBINARY(16) & NOT NULL & IPアドレス \\ \hline
		port & SMALLINT UNSIGNED & NOT NULL & ポート番号 \\ \hline
		user\_agent & VARCHAR(255) & NOT NULL & User-Agent \\ \hline
		created\_at & DATETIME & & 作成時刻 \\ \hline
		updated\_at & DATETIME & & 更新時刻 \\ \hline
		deleted\_at & DATETIME & & 削除時刻 \\ \hline
	\end{tabular}
\end{table}
問い合わせ機能での管理者と利用者の
会話は、contactsテーブル(\cref{tb:db-contacts})に
保存されます。利用者ごとに会話のスレッドが作成され、
スレッドは他の利用者には閲覧できません。
\begin{table}[H]
	\centering
	\caption{contactsテーブル(問い合わせ機能での会話)}\label{tb:db-contacts}
	\begin{tabular}{|l|l|l|l|}
		\hline
		\thead{カラム名} & \thead{データ型} & \thead{制約} & \thead{説明} \\ \hline
		id & INT UNSIGNED & PK & ID \\ \hline
		user\_id & INT UNSIGNED & FK, NOT NULL & 投稿者 \\ \hline
		thread\_user\_id & INT UNSIGNED & NOT NULL & どの利用者のスレッドか \\ \hline
		comment & TEXT & NOT NULL & 投稿文 \\ \hline
		ip\_addr & VARBINARY(16) & NOT NULL & IPアドレス \\ \hline
		port & SMALLINT UNSIGNED & NOT NULL & ポート番号 \\ \hline
		user\_agent & VARCHAR(255) & NOT NULL & User-Agent \\ \hline
		created\_at & DATETIME & & 作成時刻 \\ \hline
		updated\_at & DATETIME & & 更新時刻 \\ \hline
		deleted\_at & DATETIME & & 削除時刻 \\ \hline
	\end{tabular}
\end{table}
\section{ネットワーク設計}
\begin{figure}[H]
	\centering
	\includegraphics[bb=32.951601 147.563995 568.673983 794.321976,clip,scale=.8]{gaibu-network.pdf}
	\caption{ネットワーク構成図}\label{fig:nw}
\end{figure}

\end{document}
