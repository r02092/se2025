\documentclass{docs}

%--- 基本パッケージ ---%
\usepackage{tikz} % 図の作成

%--- TikZライブラリ ---%
\usetikzlibrary{arrows.meta} % 矢印
\usetikzlibrary{shapes.geometric} % 幾何学図形

%--- 文書情報 ---%
\title{外部設計書}

% アクティビティ図を描画するマクロ
% \adiagram{パーティション名}{縦の長さ}{流れ}{追加する描画命令}
% パーティション名: 上部に表示されるパーティション名を","で区切って記述
% 縦の長さ    : アクションごとに縦に進む長さ
% 本筋のフロー  : パーティション番号
%           0:アクション or 1:分岐
%           アクション名 or 分岐の名前
%           上記の3つを"/"で区切り、それを","で区切って記述
%           作成される各アクションのnodeは
%           "action<通し番号>"という名前で
%           (パーティション番号,-通し番号)の座標に配置される
%           本筋のアクションしか描画できないため、それ以外は
%           以下の引数に描画命令を渡して手動で描く必要がある
% 追加する描画命令: 追加で描画するTikZの形式の命令
%           条件分岐で別れた本筋以外のフローの描画などに使用
\newcommand\adiagram[4]{
	\foreach\t[count=\i from 0]in{#1}{
		\xdef\nps{\i}
	}
	\begin{tikzpicture}[
		action/.style={draw,thick,rounded corners,font=\sffamily,align=center},
		decision/.style={draw,thick,diamond,inner sep=8pt},
		arrow/.style={-Stealth},
		label/.style={fill=white,inner sep=0pt,font=\sffamily,align=center},
		xscale=(\textwidth-1.6pt)/(\nps cm+1cm)
	]
		\foreach\t[count=\i from 0]in{#1}
			\node[font=\bfseries\sffamily]at(\i,1.5){\t};
		\begin{scope}[yscale=#2]
			\node[fill,inner sep=8pt,shape=circle](action0)at(0,0){};
			\xdef\ot{}
			\xdef\ox{0}
			\xdef\oi{0}
			\foreach\x/\e/\t[count=\i from 1]in{#3}{
				\ifnum\e=0
					\node[action](action\i)at(\x,-\i){\t};
				\else
					\node[decision](action\i)at(\x,.25-\i){};
				\fi
				\draw[arrow](action\oi)
				\if"\ot"
					\ifnum\x=\ox--\else-|\fi(action\i);
				\else
					\ifnum\x=\ox--node[label]{\ot}\else-|node[label]{\ot}\fi(action\i);
					\xdef\ot{}
				\fi
				\ifnum\e=1\xdef\ot{\t}\fi
				\xdef\ox{\x}
				\xdef\oi{\i}
			}
			\node[draw,double distance=2pt,thick,fill,inner sep=8pt,shape=circle](end)at(\ox,-1-\oi){};
			\draw[arrow](action\oi)--(end);
		\end{scope}
		\foreach\i in{-.5,...,\nps.5}
			\draw[ultra thick](\i,1)--(\i,-2.5-\oi*#2);
		\begin{scope}[yscale=#2]
			#4
		\end{scope}
	\end{tikzpicture}
}

\begin{document}
\section{はじめに}%%%%%%%%%%%%%%%%%%%%%%%%%%%%%%%%%%%%%%%%%%
本書では、我々がシステム提案書にて提案した「SceneTrip」の
詳細について記述します。まず、本システムの業務の流れを説明し、
次に機能設計、ユーザーインタフェース設計、データベース設計、
ネットワーク設計について順に説明します。

\section{利用の流れ}
本システムは、高知県を訪れる観光客、
特にアニメなどの聖地巡礼を目的とした旅行者を対象として、
観光地や施設の検索、最適なルートの作成などの機能を提供するサービスです。
また、地域事業者向けには、観光データを分析できるAPIの提供や、
システム上での店舗情報掲載機能の提供が可能です。
以下に、本システムの利用の流れを示します。
\begin{figure}[H]
  \centering
  \includegraphics[bb=43.721999 420.785987 549.485983 778.409976,clip,scale=.8]{gaibu-flow.pdf}
  \caption{利用の流れ}\label{fig:flow}
\end{figure}

\section{機能設計}%%%%%%%%%%%%%%%%%%%%%%%%%%%%%%%%%%%%%%%%%%
\subsection{共通機能}
\subsubsection{利用規約}
システムを初めて利用する際に、ユーザーは利用規約に同意する必要があります。
利用規約に同意した後、ユーザーはシステムの各種機能を利用できます。

\subsubsection{ログイン・ログアウト機能}
本システムを利用するユーザーがログインおよびログアウトを行うための機能です。
ユーザーは、ユーザー名とパスワードを設定してアカウント登録をすることでログインができ、
現在設定しているパスワードを再度入力することを条件にアカウント情報を再度設定できます。
さらに、RFC 6238に準拠したTOTPによる二要素認証を設定できます。
ログイン不要で利用できる機能は、観光地・施設への経路検索のみになります。
ログインに成功すると、ユーザーはシステムの各種機能を利用でき、
ログアウトを行うと、ユーザーはシステムから安全に退出します。

\subsection{管理者}
\subsubsection{観光地・施設情報管理}
システム上で観光地や施設の情報を管理できます。
具体的には、観光地や施設の追加、編集、削除などの操作が可能です。

\subsubsection{ユーザー管理}
システムのユーザー情報を管理できます。
具体的には、ユーザー情報の登録、編集、削除などの操作が可能です。

\subsubsection{UGCの監視・管理機能}
UGCの監視・管理ができます。
具体的には、利用者が投稿した口コミやフォトスポット投稿を監視し、
利用規約違反が確認されたアカウントの利用停止や削除を行うことが可能です。

\subsubsection{システム管理}
システム全体の設定や運用を管理できます。
具体的には、システムのパフォーマンス監視、バックアップ、
セキュリティ設定などの操作が可能です。

\subsubsection{データ分析・閲覧機能}
データの閲覧・分析を行うことができます。
具体的には、行政や事業者に提供する観光データや、
ダッシュボードの表示内容を管理者として確認・監視
することが可能です。

\subsubsection{問い合わせ対応機能}
ユーザーからの問い合わせに対応できます。
具体的には、利用者や事業者からの問い合わせを一元管理し、
対応を記録することが可能です。

\subsection{利用者(ログイン不要)}
\subsubsection{観光地検索}
システム上で観光地や施設を検索できます。
検索は、キーワード検索やカテゴリ検索などの方法で行うことができます。
検索結果には、観光地や施設の詳細情報が表示されます。

\subsection{利用者(ログイン必須)}
上記機能に加え以下の機能を利用できます。
\subsubsection{利用者アカウント作成機能}
システムを利用するためのアカウントを作成できます。
作成の際には、ユーザー名、パスワード等の情報を登録します。

\subsubsection{経路作成}
システム上で訪れたい観光地や施設を選択し、
最適な観光ルートを作成できます。
システムは、選択された観光地や施設を考慮して、
効率的なルートを提案します。

\subsubsection{外部地図アプリ連携機能}
作成した観光ルートをGoogle Mapsに連携させて表示できます。
これにより、ルート作成後のナビゲーションやルート案内もGoogle Mapsで利用できます。

\subsubsection{評価機能}
訪れた観光地や施設に対して評価を行うことができます。
評価は、星評価やコメントなどの形式で行うことができます。
評価情報は、他の観光客が観光地や施設を選択する際の参考になります。

\subsubsection{写真投稿・共有機能}
システム上で訪れた観光地や施設の写真を投稿・共有できます。
投稿された写真は、他の利用者が閲覧できるようになります。

\subsubsection{スタンプ機能}
システム上で訪れた観光地や施設のスタンプをつけることができます。

\subsubsection{クーポンの受け取り機能}
システム上で提供されるクーポンを受け取ることができます。
クーポンは、観光地や施設での割引や特典に利用できます。

\subsection{事業者}

\subsubsection{スポット作成機能}
店舗や施設などのスポットを管理する事業者が、
システム上にスポットを作成できます。
作成の際には、名前、カテゴリ、所在地
などの情報を登録します。

%\subsubsection{広告掲載機能}
%地域事業者は、システム上で広告を掲載できます。
%広告は、観光客がシステムを利用する際に表示されます。
%広告掲載により、地域事業者は自社のサービスや商品を
%観光客に効果的にアピールできます。

\subsubsection{事業者情報変更機能}
登録した店舗情報(店舗名、所在地、営業時間、紹介文、店舗写真など)
を変更する機能。

\subsubsection{観光データ分析API}
システムが提供する観光データ分析APIを利用できます。
APIを通じて、観光客の行動データや評価データなどを取得し、
マーケティングやサービス改善に活用できます。

\subsubsection{クーポン発行機能}
システム上でクーポンを発行できます。
発行されたクーポンは、観光客がシステムを通じて受け取ることができます。
事業者は、クーポンの利用者が提示する二次元コードを
用いてクーポンを検証できます。

\section{サブシステム設計}%%%%%%%%%%%%%%%%%%%%%%%%%%%%%%%%%%%

\subsection{共通機能}

\begin{figure}[H]
	\centering
	\adiagram{利用者,システム}{1}{
    0/0/アクセス,
    1/0/利用規約表示,
    0/0/利用規約同意,
    0/0/同意チェックボックスをオン,
    0/0/OKボタンをクリック,
    1/0/会員登録タブを表示
	}{
  }
	\caption{利用規約同意}\label{fig:act-agree_terms}
\end{figure}

\begin{figure}[H]
	\centering
	\adiagram{利用者,システム,データベース}{1}{
    0/0/会員登録タブ選択,
    1/0/ユーザー属性選択画面表示,
    0/0/ユーザー属性選択,
    1/0/会員登録画面表示,
		0/0/必要情報を入力,
		0/0/作成ボタンをクリック,
		1/0/入力値を検証,
		2/0/入力値重複チェック,
    2/0/結果を返す,
    1/1/入力値に不備がない,
		2/0/アカウント情報を保存,
    1/0/作成完了の旨を表示,
		0/0/結果の確認
	}{
    \node[action](decision0)at(1,-13){不備がある旨を表示};
		\draw[arrow](action10)-|(1.45,-13)node[label,pos=.57]{不備あり}|-(decision0);
		\draw[arrow](decision0)--(action13);
  }
	\caption{アカウント作成}\label{fig:act-register_account}
\end{figure}

\begin{figure}[H]
	\centering
	\adiagram{利用者,システム,データベース}{1.5}{
    0/0/アカウント更新タブ選択,
    1/0/アカウント更新画面表示,
		0/0/更新情報入力,
		0/0/更新ボタンをクリック,
		1/0/入力値を検証,
		2/0/入力不備チェック,
    2/0/結果を返す,
    1/1/入力値に不備がない,
		2/0/アカウント情報を保存,
    1/0/更新完了の旨を表示,
		0/0/アカウント編集完了の確認
	}{
    \node[action](decision0)at(1,-11){不備がある旨を表示};
		\draw[arrow](action8)-|(1.45,-8)node[label,pos=.6]{不備あり}|-(decision0);
		\draw[arrow](decision0)--(action11);
  }
	\caption{アカウント情報更新}\label{fig:act-edit_account}
\end{figure}

\begin{figure}[H]
	\centering
	\adiagram{利用者,システム,データベース}{1.5}{
    0/0/ログインボタンをクリック,
    1/0/ログイン画面表示,
    0/0/ログインを選択,
		0/0/必要情報を入力,
		0/0/ログインボタンをクリック,
		1/0/SQLクエリを生成,
		2/0/検索を実行,
    2/0/結果を返す,
    1/1/ユーザー情報が存在する,
		1/0/ログイン成功の旨を表示,
		0/0/結果を確認
	}{
    \node[action](decision0)at(1,-11){存在しない旨を表示};
		\draw[arrow](action9)-|(1.45,-11)node[label,pos=.57]{存在しない}|-(decision0);
		\draw[arrow](decision0)--(action11);
  }
	\caption{ログイン}\label{fig:act-login}
\end{figure}

\begin{figure}[H]
	\centering
  \adiagram{利用者,システム,データベース}{1.5}{
		0/0/ログアウトボタンを\\クリック,
    2/0/セッション情報を削除,
    2/0/結果を返す,
    1/0/結果を表示,
    0/0/結果を確認
      }{
    % 不備がある場合の処理
  }
	\caption{ログアウト}\label{fig:act-logout}
\end{figure}

\begin{figure}[H]
	\centering
  \adiagram{利用者,システム,データベース}{1.2}{
		0/0/お楽しみ機能タブを選択,
    1/0/お楽しみ画面表示,
    0/0/チェックイン画面を\\クリック,
    1/0/カメラ機能を起動,
    0/0/カメラで\\QRコードを読み取る,
    1/0/施設情報を取得,
    1/0/位置情報機能を起動,
    0/0/位置情報の利用を許可,
    1/0/位置情報を取得,
    2/0/現在地情報と\\施設情報を照合,
    1/1/位置情報が一致する,
    2/0/ユーザの\\スタンプ情報を更新,
    1/0/スタンプ取得の表示,
		0/0/結果の確認
      }{
    \node[action](decision0)at(1,-14){取得できない旨を表示};
		\draw[arrow](action11)-|(1.45,-14)node[label,pos=.57]{一致しない}|-(decision0);
		\draw[arrow](decision0)--(action14);
  }
	\caption{スタンプ}\label{fig:act-post_stamp}
\end{figure}

\subsection{管理者}

\begin{figure}[H]
	\centering
  \adiagram{利用者,システム,データベース}{1.5}{
		0/0/追加タブ選択,
    1/0/観光地・施設情報\\追加画面表示,
    0/0/必要情報を入力,
    0/0/追加ボタンをクリック,
    1/0/SQLクエリを生成,
    2/0/追加処理を実行,
    2/0/結果を返す,
    1/1/入力に不備がない,
    1/0/追加完了の旨の表示,
    0/0/結果を確認
      }{
    \node[action](decision0)at(1,-10){追加できない旨を表示};
		\draw[arrow](action8)-|(1.45,-10)node[label,pos=.57]{不備あり}|-(decision0);
		\draw[arrow](decision0)--(action10);
  }
	\caption{観光地・施設情報追加}\label{fig:act-add_spot}
\end{figure}

\begin{figure}[H]
	\centering
  \adiagram{利用者,システム,データベース}{1.5}{
		0/0/編集タブを選択,
    1/0/観光地・施設情報\\編集画面表示,
    0/0/情報を編集,
    0/0/編集ボタンをクリック,
    1/0/SQLクエリを生成,
    2/0/編集処理を実行,
    2/0/結果を返す,
    1/1/編集情報に不備がない,
    1/0/編集完了の旨を表示,
    0/0/結果を確認
      }{
    \node[action](decision0)at(1,-10){編集できない旨を表示};
		\draw[arrow](action8)-|(1.45,-10)node[label,pos=.57]{不備あり}|-(decision0);
		\draw[arrow](decision0)--(action10);
  }
	\caption{観光地・施設情報編集}\label{fig:act-edit_spot}
\end{figure}

\begin{figure}[H]
	\centering
  \adiagram{利用者,システム,データベース}{1.5}{
		0/0/削除タブを選択,
    1/0/観光地・施設情報\\削除画面表示,
    0/0/削除ボタンをクリック,
    0/0/操作の確認画面を表示,
    0/1/削除を確定,
    1/0/SQLクエリを生成,
    2/0/削除を実行,
    2/0/結果を返す,
    1/1/削除が成功する,
    1/0/削除完了の旨を表示,
    0/0/結果を確認
      }{
    \node[action](decision0)at(0,-6){元の画面に戻る};
		\draw[arrow](action5)-|(0.45,-6)node[label,pos=.7]{キャンセルを選択}|-(decision0);
		\draw[arrow](decision0)-|(-0.45,-12)|-(end);
    \node[action](decision1)at(1,-11){削除できない旨を表示};
		\draw[arrow](action9)-|(1.45,-11)node[label,pos=.57]{不備あり}|-(decision1);
		\draw[arrow](decision1)--(action11);
  }
	\caption{観光地・施設情報削除}\label{fig:act-delete_spot}
\end{figure}

\begin{figure}[H]
	\centering
  \adiagram{利用者,システム,データベース}{1.5}{
		0/0/追加タブ選択,
    1/0/ユーザー情報追加画面表示,
    0/0/必要情報を入力,
    0/0/追加ボタンをクリック,
    1/0/SQLクエリを生成,
    2/0/追加処理を実行,
    2/0/結果を返す,
    1/1/入力に不備がない,
    1/0/追加完了の旨を表示,
    0/0/結果を確認
      }{
    \node[action](decision0)at(1,-10){追加できない旨を表示};
		\draw[arrow](action8)-|(1.45,-10)node[label,pos=.57]{不備あり}|-(decision0);
		\draw[arrow](decision0)--(action10);
  }
	\caption{ユーザー情報追加}\label{fig:act-add_user}
\end{figure}

\begin{figure}[H]
	\centering
  \adiagram{利用者,システム,データベース}{1.5}{
		0/0/編集タブを選択,
    1/0/ユーザー情報編集画面表示,
    0/0/情報を編集,
    0/0/編集ボタンをクリック,
    1/0/SQLクエリを生成,
    2/0/編集処理を実行,
    2/0/結果を返す,
    1/1/編集情報に不備がない,
    1/0/編集完了の旨を表示,
    0/0/結果を確認
      }{
    \node[action](decision0)at(1,-10){編集できない旨を表示};
		\draw[arrow](action8)-|(1.45,-10)node[label,pos=.57]{不備あり}|-(decision0);
		\draw[arrow](decision0)--(action10);
  }
	\caption{ユーザー情報編集}\label{fig:act-edit_user}
\end{figure}

\begin{figure}[H]
	\centering
  \adiagram{利用者,システム,データベース}{1.5}{
		0/0/削除タブを選択,
    1/0/ユーザー情報削除画面表示,
    0/0/削除ボタンをクリック,
    1/0/操作の確認画面を表示,
    0/1/削除を確定,
    1/0/SQLクエリを生成,
    2/0/削除を実行,
    2/0/結果を返す,
    1/1/削除が成功する,
    1/0/結果を表示,
    0/0/結果を確認
      }{
    \node[action](decision0)at(0,-6){元の画面に戻る};
		\draw[arrow](action5)-|(0.45,-6)node[label,pos=.7]{キャンセルを選択}|-(decision0);
		\draw[arrow](decision0)-|(-0.45,-12)|-(end);
    \node[action](decision1)at(1,-11){削除できない旨を表示};
		\draw[arrow](action9)-|(1.45,-11)node[label,pos=.57]{不備あり}|-(decision1);
		\draw[arrow](decision1)--(action11);
  }
	\caption{ユーザー情報削除}\label{fig:act-delete_user}
\end{figure}

\begin{figure}[H]
	\centering
  \adiagram{管理者,システム,データベース}{1.5}{
		0/0/UGC管理画面を開く,
    1/0/投稿一覧を表示,
    0/0/投稿を選択,
    0/0/削除ボタンを\\クリック,
    1/0/操作の確認画面を表示,
    0/1/操作を確定,
    1/0/SQLクエリを生成,
    2/0/削除処理を実行,
    2/0/結果を返す,
    1/1/処理が成功する,
    1/0/削除できた旨を表示,
    0/0/結果を確認
      }{
    \node[action](decision0)at(0,-7){元の画面に戻る};
		\draw[arrow](action5)-|(0.45,-7)node[label,pos=.8]{キャンセルを選択}|-(decision0);
		\draw[arrow](decision0)-|(-0.45,-13)|-(end);
    \node[action](decision1)at(1,-12){削除できない旨を表示};
		\draw[arrow](action10)-|(1.45,-12)node[label,pos=.57]{不備あり}|-(decision1);
		\draw[arrow](decision1)--(action12);
  }
	\caption{UGC監視・管理}\label{fig:act-manage_ugc}
\end{figure}

\begin{figure}[H]
	\centering
  \adiagram{管理者,システム,データベース}{1.5}{
		0/0/問い合わせ一覧画面を開く,
    1/0/問い合わせ一覧を表示,
    0/0/返信する問い合わせを選択,
    1/0/問い合わせ詳細と\\返信フォームを表示,
    0/0/返信内容を入力,
    0/0/送信ボタンをクリック,
    1/0/入力値を検証,
    1/1/入力値に不備がない,
    2/0/返信内容を保存,
    1/0/返信完了画面を表示,
    0/0/返信完了を確認
      }{
    \node[action](decision1)at(1,-11){返信できない旨を表示};
		\draw[arrow](action8)-|(1.45,-11)node[label,pos=.57]{不備あり}|-(decision1);
		\draw[arrow](decision1)--(action11);
  }
	\caption{問い合わせ返信}\label{fig:act-reply_inquiry}
\end{figure}

\subsection{利用者}
\subsubsection{ログイン不要機能}

\begin{figure}[H]
	\centering
	\adiagram{利用者,システム,データベース}{1.5}{
		0/0/検索クエリを入力,
		0/0/検索ボタンをクリック,
		1/0/SQLクエリを生成,
		2/0/検索を実行,
		2/0/結果を返す,
		1/1/観光地・施設が存在する,
		1/0/観光地・施設の\\情報を表示,
		0/0/結果を確認
	}{
		\node[action](decision0)at(1,-8){存在しない旨を表示};
		\draw[arrow](action6)-|(1.45,-8)node[label,pos=.57]{存在しない}|-(decision0);
		\draw[arrow](decision0)--(action8);
	}
	\caption{観光地・施設検索}\label{fig:act-spot_search}
\end{figure}

\subsubsection{ログイン必須機能}

\begin{figure}[H]
	\centering
  \adiagram{利用者,システム,データベース}{1.5}{
		0/0/投稿タブを選択,
    1/0/投稿画面表示,
    0/0/必要情報を入力,
    0/0/投稿ボタンをクリック,
    1/0/SQLクエリを生成,
    2/0/入力値を検証,
    1/1/入力値に不備がない,
		2/0/投稿内容を保存,
    1/0/投稿完了を表示,
		0/0/結果の確認
      }{
    \node[action](decision0)at(1,-10){不備がある旨を表示};
		\draw[arrow](action7)-|(1.45,-10)node[label,pos=.57]{不備あり}|-(decision0);
		\draw[arrow](decision0)--(action10);
  }
	\caption{写真投稿・共有}\label{fig:act-post_spot}
\end{figure}

\begin{figure}[H]
	\centering
  \adiagram{利用者,システム,データベース}{1.6}{
		0/0/経路に追加する\\スポットを選択,
    0/0/経路作成ボタンを\\クリック,
    1/0/経路作成画面を表示,
    0/0/経由地などを設定し\\経路を確定,
    1/0/スポット情報を基に\\経路計算クエリを生成,
    2/0/経路情報を検索,
    2/0/結果を返す,
    1/0/最適な経路を計算,
    1/0/計算結果を表示,
    0/0/Google マップで\\開くボタンをクリック,
    1/0/Google マップ連携用\\URLを生成し転送
      }{
  }
	\caption{経路作成}\label{fig:act-create_route}
\end{figure}

\begin{figure}[H]
	\centering
  \adiagram{利用者,システム,データベース}{1.5}{
		0/0/施設詳細画面で\\評価ボタンをクリック,
    1/0/評価入力画面を表示,
    0/0/星評価・コメントを\\入力,
    0/0/投稿ボタンをクリック,
    1/0/入力値を検証,
    1/1/入力値に不備がない,
    1/0/SQLクエリを生成,
		2/0/投稿内容を保存,
    1/0/投稿完了を表示,
		0/0/結果を確認
      }{
    \node[action](decision0)at(1,-10){不備がある旨を表示};
		\draw[arrow](action6)-|(1.45,-10)node[label,pos=.57]{不備あり}|-(decision0);
		\draw[arrow](decision0)--(action10);
  }
	\caption{施設評価}\label{fig:act-post_review}
\end{figure}

\begin{figure}[H]
	\centering
  \adiagram{利用者,システム,AI,データベース}{2.0}{
		0/0/チャットで施設の\\希望を入力・送信,
    1/0/入力内容から\\キーワードや\\意図を抽出,
    1/0/キーワードを\\AIに送信,
    2/0/キーワードに基づき\\検索条件を生成,
    2/0/問い合わせ結果を\\返す,
    1/0/生成された条件で\\DBに施設情報を\\問合せ,
    3/0/施設情報を\\検索・返却,
    1/0/検索結果を\\チャットで表示,
		0/0/提示された\\施設を確認
      }{
  }
	\caption{AIによる観光地・施設推薦}\label{fig:act-chat_spot_recommendation}
\end{figure}

\begin{figure}[H]
	\centering
  \adiagram{利用者,システム,データベース}{1.5}{
		0/0/特定のスタンプを獲得,
    1/0/獲得したスタンプ情報を\\DBに問い合わせ,
    2/0/スタンプ情報を基に\\入手可能なクーポンを検索,
    2/0/結果を返す,
    1/1/入手可能なクーポンが\\存在する場合,
    2/0/クーポン情報を保存,
    1/0/クーポン獲得を通知,
    0/0/獲得したクーポンを確認
      }{
    \node[action](decision0)at(1,-8){不備がある旨を表示};
		\draw[arrow](action5)-|(1.45,-8)node[label,pos=.57]{不備あり}|-(decision0);
		\draw[arrow](decision0)--(action8);
  }
	\caption{クーポン受け取り}\label{fig:act-receive_coupon}
\end{figure}

\begin{figure}[H]
	\centering
  \adiagram{利用者,システム,データベース}{1.5}{
		0/0/問い合わせボタンをクリック,
    1/0/問い合わせ画面表示,
    0/0/必要情報を入力,
    0/0/送信ボタンをクリック,
    1/0/入力値を検証,
    1/1/入力値に不備がない,
    2/0/問い合わせ内容を保存,
    1/0/問い合わせ送信完了\\の旨を表示,
		0/0/問い合わせ送信\\完了の確認
      }{
    \node[action](decision0)at(1,-9){不備がある旨を表示};
		\draw[arrow](action6)-|(1.45,-9)node[label,pos=.57]{不備あり}|-(decision0);
		\draw[arrow](decision0)--(action9);
  }
	\caption{問い合わせ対応}\label{fig:act-inquiry_response}
\end{figure}

\subsection{事業者}
\begin{figure}[H]
  \centering
  \adiagram{利用者,システム,データベース}{1.5}{
    0/0/スポット作成タブを選択,
    1/0/スポット作成画面表示,
    0/0/必要情報を入力,
    0/0/作成ボタンをクリック,
    1/0/入力値を検証,
    1/1/入力値に不備がない,
    2/0/スポット情報を保存,
    1/0/スポット情報を表示,
    0/0/スポット作成完了の確認
      }{
    \node[action](decision0)at(1,-9){不備がある旨を表示};
		\draw[arrow](action6)-|(1.45,-9)node[label,pos=.57]{不備あり}|-(decision0);
		\draw[arrow](decision0)--(action9);
  }
  \caption{スポット作成}\label{fig:act-create_spot}
\end{figure}

\begin{figure}[H]
  \centering
  \adiagram{事業者,システム,データベース}{1.5}{
    0/0/APIリクエストを送信,
    1/0/APIリクエストを受信,
    1/0/認証情報を検証,
    1/1/認証が成功,
    1/0/分析用クエリを生成,
    2/0/クエリを実行し\\データを集計,
    2/0/結果を返す,
    1/0/データを整形し\\APIレスポンスを送信,
    0/0/データを受信
  }{
    \node[action](decision0)at(1,-10){認証失敗の旨を表示};
		\draw[arrow](action4)-|(1.45,-10)node[label,pos=.56]{認証失敗}|-(decision0);
		\draw[arrow](decision0)--(end);
  }
  \caption{観光データ分析API}\label{fig:act-data_analysis_api}
\end{figure}

\begin{figure}[H]
  \centering
  \adiagram{事業者,システム,データベース}{1.5}{
    0/0/クーポン発行タブを選択,
    1/0/クーポン発行画面表示,
    0/0/必要情報を入力,
    0/0/発行ボタンをクリック,
    1/0/入力値を検証,
    1/1/入力値に不備がない,
    2/0/クーポン情報を保存,
    1/0/クーポン発行\\の旨を表示,
    0/0/結果の確認
  }{
    \node[action](decision0)at(1,-9){クーポン発行失敗\\の旨を表示};
		\draw[arrow](action6)-|(1.45,-9)node[label,pos=.57]{不備あり}|-(decision0);
		\draw[arrow](decision0)--(action9);
  }
  \caption{クーポン発行}\label{fig:act-create_coupon}
\end{figure}

\section{ユーザーインタフェース設計}
\newpage
\begin{figure}[H]
	\centering
	\begin{tikzpicture}[scale=.0375]
		\draw(-210,297)--(210,297)--(210,-297)--(-210,-297)--cycle;
		\node[inner sep=0pt,outer sep=0pt]{\includegraphics[scale=.75]{invoice/invoice.pdf}};
	\end{tikzpicture}
	\caption{出力される請求書の様式}\label{fig:invoice}
\end{figure}
\section{データベース設計}
\subsection{ER図}
IE記法によるデータベースのER図を\cref{fig:er}に示します。
\begin{figure}[H]
	\centering
	\includegraphics[bb=16.65246 40.229999 824.542569 584.729982,clip,scale=.55]{gaibu-er.pdf}
	\caption{ER図}\label{fig:er}
\end{figure}
\subsection{テーブルとカラムの詳細}
各テーブルおよびカラムの詳細を示します。
IDについて特筆なきものは、AUTO INCREMENT属性を
指定することで自動で採番を行うものとします。
経緯度については、以下に示す式を用いて度単位の値を変換し、
得られた値を格納します。
\[
	\text{[データベースに格納する値]}=\frac{\text{[元の値]}-\text{[最小値]}}{\text{[最大値]}-\text{[最小値]}}(2^{32}-1)
\]

経路検索のために、システムは
鉄道(軌道を含む)や国道(高速道路を含む)および
それらを接続する主な道路を模したグラフをデータベースに
保持します。nodesテーブル(\cref{tb:db-nodes})は
グラフのノードを、edgesテーブル(\cref{tb:db-edges})は
グラフのエッジをそれぞれ保存するテーブルとなっています。
\begin{table}[H]
	\centering
	\caption{nodesテーブル(グラフのノード)}\label{tb:db-nodes}
	\begin{tabular}{|l|l|l|l|}
		\hline
		\thead{カラム名} & \thead{データ型} & \thead{制約} & \thead{説明} \\ \hline
		id & INT UNSIGNED & PK & ID \\ \hline
		type & TINYINT UNSIGNED & UK, NOT NULL & 種別 \\ \hline
		code & INT UNSIGNED & UK, NOT NULL & コード \\ \hline
		name & VARCHAR(255) & NOT NULL & 地点名 \\ \hline
		lng & INT UNSIGNED & NOT NULL & 経度 \\ \hline
		lat & INT UNSIGNED & NOT NULL & 緯度 \\ \hline
	\end{tabular}
\end{table}
nodesテーブルのIDは、typeカラムおよびcodeカラムの
値からSHA3-224で生成したハッシュ値の先頭4バイトを使用して
採番します。また、typeカラムの値の意味とtypeカラムの
値ごとのcodeカラムの意味を\cref{tb:db-nodes-type}に示します。
\begin{table}[H]
	\centering
	\caption{nodesテーブルのtypeカラムおよびcodeカラムについて}\label{tb:db-nodes-type}
	\begin{tabular}{|r|l|l|}
		\hline
		\thead{type} & \thead{意味} & \thead{codeの値の意味} \\ \hline
		0 & 交差点 &
		\begin{tabular}{l}
			$\text{[JIS X 0402:2020で定められた市区町村コード]}\times 1000$\\
			${}+\text{[独自に定めた市区町村内で一意の値]}$\\
			(市区町村コードはJIS X 0401:1973で定められた\\
			\quad 都道府県コードと併用するものとする)
		\end{tabular}
		\\ \hline
		1 & 鉄道(軌道を含む)の駅 &
		\begin{tabular}{l}
			「緯度経度付き全国沿線・駅データベース」で\\
			定められた駅コード
		\end{tabular}
		\\ \hline
		2 & 高速道路のIC &
		\begin{tabular}{l}
			$\text{[各路線に対し独自に定めた一意の値]}\times 10000$\\
			${}+\text{[IC番号]}\times 10+\text{[IC番号の枝番]}$
		\end{tabular}
		\\ \hline
		3 & 常時観測点 &
		\begin{tabular}{l}
			「国土交通省交通量API」で提供される\\
			常時観測点コード
		\end{tabular}
		\\ \hline
	\end{tabular}
\end{table}
\begin{table}[H]
	\centering
	\caption{edgesテーブル(グラフのエッジ)}\label{tb:db-edges}
	\begin{tabular}{|l|l|l|l|}
		\hline
		\thead{カラム名} & \thead{データ型} & \thead{制約} & \thead{説明} \\ \hline
		id & INT UNSIGNED & PK & ID \\ \hline
		node1\_id & INT UNSIGNED & FK, UK, NOT NULL & 1つ目のノード \\ \hline
		node2\_id & INT UNSIGNED & UK, NOT NULL & 2つ目のノード \\ \hline
		time & INT UNSIGNED & NOT NULL & ノード間の所要時間 \\ \hline
	\end{tabular}
\end{table}
このシステムでは、管理者や事業者を含む全ての利用者を
共通のusersテーブル(\cref{tb:db-users})で
扱い、\cref{tb:db-users-type}に示すようにtypeカラムの値により
それらを区別します。
事業者の場合、住所の登録を要するものとします。
住所は郵便番号およびJIS X 0402:2020で
定められた市区町村コード(JIS X 0401:1973で定められた
都道府県コードと併用するもの)と市区町村に続く住所の文字列の
組合せにより保存します。
また、ログイン時にTOTPによる二要素認証を
行えるようにするためのカラムを設けています。
総当たり攻撃を防ぐため、直近の認証時の時刻と
認証回数を保持するカラムも設けており、
高頻度な認証操作を制限できます。OTPの更新頻度は30秒と
するため、直近の認証時の時刻は、UNIX時刻を30で
割った値として保存し、認証回数も30秒ごとに数え直すものと
します。
\begin{table}[H]
	\centering
	\caption{usersテーブル(管理者や事業者を含む利用者)}\label{tb:db-users}
	\begin{tabular}{|l|l|l|l|}
		\hline
		\thead{カラム名} & \thead{データ型} & \thead{制約} & \thead{説明} \\ \hline
		id & INT UNSIGNED & PK & ID \\ \hline
		login\_name & VARCHAR(255) & UK, NOT NULL & ログイン名 \\ \hline
		password & VARCHAR(255) & NOT NULL & ハッシュ化されたパスワード \\ \hline
		type & TINYINT UNSIGNED & NOT NULL & 種別 \\ \hline
		name & VARCHAR(255) & NOT NULL & 名前 \\ \hline
		points & INT UNSIGNED & NOT NULL & ポイント数 \\ \hline
		num\_plan\_std & INT UNSIGNED & NOT NULL & スタンダードプランの契約数 \\ \hline
		num\_plan\_prm & INT UNSIGNED & NOT NULL & プレミアムプランの契約数 \\ \hline
		postal\_code & INT UNSIGNED & & 郵便番号 \\ \hline
		addr\_city & INT UNSIGNED & & 市区町村コード \\ \hline
		addr\_detail & VARCHAR(255) & & 市区町村名の後に続く住所 \\ \hline
		totp\_secret & BINARY(20) & & TOTPのシークレット \\ \hline
		totp\_iv & BINARY(12) & & TOTPの初期ベクトル \\ \hline
		totp\_tag & BINARY(16) & & TOTPの認証タグ \\ \hline
		totp\_last\_time & INT UNSIGNED & & TOTPの直近の認証時の時刻 \\ \hline
		totp\_counter & TINYINT UNSIGNED & & TOTPの30秒ごとの認証回数 \\ \hline
		created\_at & DATETIME & & 作成時刻 \\ \hline
		updated\_at & DATETIME & & 更新時刻 \\ \hline
		deleted\_at & DATETIME & & 削除時刻 \\ \hline
	\end{tabular}
\end{table}
\begin{table}[H]
	\centering
	\caption{usersテーブルのtypeカラムについて}\label{tb:db-users-type}
	\begin{tabular}{|r|l|}
		\hline
		\thead{type} & \thead{意味} \\ \hline
		0 & 管理者 \\ \hline
		1 & 利用者(事業者を含む) \\ \hline
	\end{tabular}
\end{table}
パスワードのハッシュ化にはPHPの\verb|password_hash()|関数を
使用し、暗号学的ハッシュ関数であるArgon2idを用い、かつ
ソルトを使用することで、安全にパスワードを保存します。

観光地や観光施設、店などのスポットを
保存するspotsテーブル(\cref{tb:db-spots})は、
近隣のグラフのノードに関連付けることで、
経路検索にスポットを使用できるようにします。
また、keywordsテーブル(\cref{tb:db-keywords})
スポットにはキーワードを関連付けることができます。
このキーワードはAI検索時のLLMのシステムプロンプトの構成などに
使用されます。
\begin{table}[H]
	\centering
	\caption{spotsテーブル(観光地や観光施設、店などのスポット)}\label{tb:db-spots}
	\begin{tabular}{|l|l|l|l|}
		\hline
		\thead{カラム名} & \thead{データ型} & \thead{制約} & \thead{説明} \\ \hline
		id & INT UNSIGNED & PK & ID \\ \hline
		node\_id & INT UNSIGNED & FK, NOT NULL & 関連付けたノード \\ \hline
		user\_id & INT UNSIGNED & FK, NOT NULL & 登録者 \\ \hline
		type & TINYINT UNSIGNED & NOT NULL & カテゴリ \\ \hline
		name & VARCHAR(255) & NOT NULL & 名前 \\ \hline
		lng & INT UNSIGNED & NOT NULL & 経度 \\ \hline
		lat & INT UNSIGNED & NOT NULL & 緯度 \\ \hline
		created\_at & DATETIME & & 作成時刻 \\ \hline
		updated\_at & DATETIME & & 更新時刻 \\ \hline
		deleted\_at & DATETIME & & 削除時刻 \\ \hline
	\end{tabular}
\end{table}
\begin{table}[H]
	\centering
	\caption{keywordsテーブル(スポットに関連付けるキーワード)}\label{tb:db-keywords}
	\begin{tabular}{|l|l|l|l|}
		\hline
		\thead{カラム名} & \thead{データ型} & \thead{制約} & \thead{説明} \\ \hline
		id & INT UNSIGNED & PK & ID \\ \hline
		spot\_id & INT UNSIGNED & FK, UK, NOT NULL & 関連付けたスポット \\ \hline
		keyword & VARCHAR(255) & UK, NOT NULL & キーワード \\ \hline
		created\_at & DATETIME & & 作成時刻 \\ \hline
		updated\_at & DATETIME & & 更新時刻 \\ \hline
		deleted\_at & DATETIME & & 削除時刻 \\ \hline
	\end{tabular}
\end{table}
スポットに対する口コミは、reviewsテーブル(\cref{tb:db-reviews})に
保存します。段階的な評価および文章を入力できます。
また、閲覧数も記録され、事業者は閲覧されている口コミを
確認できます。
\begin{table}[H]
	\centering
	\caption{reviewsテーブル(スポットに対する口コミ)}\label{tb:db-reviews}
	\begin{tabular}{|l|l|l|l|}
		\hline
		\thead{カラム名} & \thead{データ型} & \thead{制約} & \thead{説明} \\ \hline
		id & INT UNSIGNED & PK & ID \\ \hline
		spot\_id & INT UNSIGNED & FK, NOT NULL & 関連付けたスポット \\ \hline
		user\_id & INT UNSIGNED & FK, NOT NULL & 投稿者 \\ \hline
		rate & TINYINT UNSIGNED & NOT NULL & 評価 \\ \hline
		comment & TEXT & NOT NULL & 本文 \\ \hline
		views & INT UNSIGNED & NOT NULL & 閲覧数 \\ \hline
		ip\_addr & VARBINARY(16) & NOT NULL & IPアドレス \\ \hline
		port & SMALLINT UNSIGNED & NOT NULL & ポート番号 \\ \hline
		user\_agent & VARCHAR(255) & NOT NULL & User-Agent \\ \hline
		created\_at & DATETIME & & 作成時刻 \\ \hline
		updated\_at & DATETIME & & 更新時刻 \\ \hline
		deleted\_at & DATETIME & & 削除時刻 \\ \hline
	\end{tabular}
\end{table}
事業者が発行したクーポンはcouponsテーブル(\cref{tb:db-coupons})に
保存します。クーポンを手に入れるために行く必要のあるスポットも
記録します。
また、利用者が手に入れたクーポンはuser\_couponsテーブル(\cref{tb:db-user_coupons})に
保存します。クーポンの入手時に確認用キーが生成され、
これを含む二次元コードを事業者に提示することで
正規のクーポンであることを証明します。
\begin{table}[H]
	\centering
	\caption{couponsテーブル(スポットに対して作成されたクーポン)}\label{tb:db-coupons}
	\begin{tabular}{|l|l|l|l|}
		\hline
		\thead{カラム名} & \thead{データ型} & \thead{制約} & \thead{説明} \\ \hline
		id & INT UNSIGNED & PK & ID \\ \hline
		spot\_id & INT UNSIGNED & FK, NOT NULL & 使用できるスポット \\ \hline
		name & VARCHAR(255) & NOT NULL & 名前 \\ \hline
		cond\_spot\_id & INT UNSIGNED & NOT NULL & 行く必要のあるスポット \\ \hline
		created\_at & DATETIME & & 作成時刻 \\ \hline
		updated\_at & DATETIME & & 更新時刻 \\ \hline
		deleted\_at & DATETIME & & 削除時刻 \\ \hline
	\end{tabular}
\end{table}
\begin{table}[H]
	\centering
	\caption{user\_couponsテーブル(利用者が手に入れたクーポン)}\label{tb:db-user_coupons}
	\begin{tabular}{|l|l|l|l|}
		\hline
		\thead{カラム名} & \thead{データ型} & \thead{制約} & \thead{説明} \\ \hline
		id & INT UNSIGNED & PK & ID \\ \hline
		coupon\_id & INT UNSIGNED & FK, UK1, NOT NULL & 関連付けたクーポン \\ \hline
		user\_id & INT UNSIGNED & FK, UK1, NOT NULL & 手に入れた利用者 \\ \hline
		key & BIGINT UNSIGNED & UK2, NOT NULL & 確認用キー \\ \hline
		is\_used & TINYINT UNSIGNED & NOT NULL & 使用されたか \\ \hline
		created\_at & DATETIME & & 作成時刻 \\ \hline
		updated\_at & DATETIME & & 更新時刻 \\ \hline
	\end{tabular}
\end{table}
利用者が手に入れたスタンプはstampsテーブル(\cref{tb:db-stamps})に
保存します。スタンプの画像はファイルシステムに保存するため、
それに対応するテーブルやカラムは存在しません。
\begin{table}[H]
	\centering
	\caption{stampsテーブル(利用者が手に入れたスタンプ)}\label{tb:db-stamps}
	\begin{tabular}{|l|l|l|l|}
		\hline
		\thead{カラム名} & \thead{データ型} & \thead{制約} & \thead{説明} \\ \hline
		id & INT UNSIGNED & PK & ID \\ \hline
		spot\_id & INT UNSIGNED & FK, UK, NOT NULL & 関連付けたスポット \\ \hline
		user\_id & INT UNSIGNED & FK, UK, NOT NULL & 手に入れた利用者 \\ \hline
		ip\_addr & VARBINARY(16) & NOT NULL & IPアドレス \\ \hline
		port & SMALLINT UNSIGNED & NOT NULL & ポート番号 \\ \hline
		user\_agent & VARCHAR(255) & NOT NULL & User-Agent \\ \hline
		created\_at & DATETIME & & 作成時刻 \\ \hline
		updated\_at & DATETIME & & 更新時刻 \\ \hline
	\end{tabular}
\end{table}
利用者の検索クエリは、事業者の分析に
使用するため、queriesテーブル(\cref{tb:db-queries})に
保存されます。経路検索を伴う場合は、その出発点および目的地も
保存されます。
\begin{table}[H]
	\centering
	\caption{queriesテーブル(利用者の検索クエリ)}\label{tb:db-queries}
	\begin{tabular}{|l|l|l|l|}
		\hline
		\thead{カラム名} & \thead{データ型} & \thead{制約} & \thead{説明} \\ \hline
		id & INT UNSIGNED & PK & ID \\ \hline
		user\_id & INT UNSIGNED & FK, NOT NULL & 検索した利用者 \\ \hline
		query & TEXT & NOT NULL & 検索クエリ \\ \hline
		from\_spot\_id & INT UNSIGNED & & 出発地 \\ \hline
		to\_spot\_id & INT UNSIGNED & & 目的地 \\ \hline
		ip\_addr & VARBINARY(16) & NOT NULL & IPアドレス \\ \hline
		port & SMALLINT UNSIGNED & NOT NULL & ポート番号 \\ \hline
		user\_agent & VARCHAR(255) & NOT NULL & User-Agent \\ \hline
		created\_at & DATETIME & & 作成時刻 \\ \hline
		updated\_at & DATETIME & & 更新時刻 \\ \hline
	\end{tabular}
\end{table}
利用者の投稿した写真は、photosテーブル(\cref{tb:db-photos})に
保存されます。地図上に表示するため、写真の経緯度を保存します。
また、画像の拡張子も保存します。拡張子を除いた
ファイル名はIDから生成可能なものとします。
\begin{table}[H]
	\centering
	\caption{photosテーブル(利用者の投稿した写真)}\label{tb:db-photos}
	\begin{tabular}{|l|l|l|l|}
		\hline
		\thead{カラム名} & \thead{データ型} & \thead{制約} & \thead{説明} \\ \hline
		id & INT UNSIGNED & PK & ID \\ \hline
		user\_id & INT UNSIGNED & FK, NOT NULL & 投稿者 \\ \hline
		lng & INT UNSIGNED & NOT NULL & 経度 \\ \hline
		lat & INT UNSIGNED & NOT NULL & 緯度 \\ \hline
		img\_ext & VARCHAR(4) & NOT NULL & 画像の拡張子 \\ \hline
		comment & TEXT & NOT NULL & 写真に付けた文 \\ \hline
		ip\_addr & VARBINARY(16) & NOT NULL & IPアドレス \\ \hline
		port & SMALLINT UNSIGNED & NOT NULL & ポート番号 \\ \hline
		user\_agent & VARCHAR(255) & NOT NULL & User-Agent \\ \hline
		created\_at & DATETIME & & 作成時刻 \\ \hline
		updated\_at & DATETIME & & 更新時刻 \\ \hline
		deleted\_at & DATETIME & & 削除時刻 \\ \hline
	\end{tabular}
\end{table}
問い合わせ機能での管理者と利用者の
会話は、contactsテーブル(\cref{tb:db-contacts})に
保存されます。利用者ごとに会話のスレッドが作成され、
スレッドは他の利用者には閲覧できません。
\begin{table}[H]
	\centering
	\caption{contactsテーブル(問い合わせ機能での会話)}\label{tb:db-contacts}
	\begin{tabular}{|l|l|l|l|}
		\hline
		\thead{カラム名} & \thead{データ型} & \thead{制約} & \thead{説明} \\ \hline
		id & INT UNSIGNED & PK & ID \\ \hline
		user\_id & INT UNSIGNED & FK, NOT NULL & 投稿者 \\ \hline
		thread\_user\_id & INT UNSIGNED & NOT NULL & どの利用者のスレッドか \\ \hline
		comment & TEXT & NOT NULL & 投稿文 \\ \hline
		ip\_addr & VARBINARY(16) & NOT NULL & IPアドレス \\ \hline
		port & SMALLINT UNSIGNED & NOT NULL & ポート番号 \\ \hline
		user\_agent & VARCHAR(255) & NOT NULL & User-Agent \\ \hline
		created\_at & DATETIME & & 作成時刻 \\ \hline
		updated\_at & DATETIME & & 更新時刻 \\ \hline
		deleted\_at & DATETIME & & 削除時刻 \\ \hline
	\end{tabular}
\end{table}
\section{ネットワーク設計}
\begin{figure}[H]
	\centering
	\includegraphics[bb=32.951601 147.563995 568.673983 794.321976,clip,scale=.8]{gaibu-network.pdf}
	\caption{ネットワーク構成図}\label{fig:nw}
\end{figure}

\end{document}
